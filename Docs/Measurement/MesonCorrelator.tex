\subsection{\label{MesonCorrelator}\index{meson}\index{correlator}Meson Correlator}

\subsubsection{\label{MesonWaveFunction}Meson Wave Function}

We need at first construct an observable which is a bound state of two fermions and \textbf{has the same quantum number} as mesons. In short, we want to know
\begin{equation}
\begin{split}
&O(x)=\bar{\psi}(x)\Gamma \psi (x)\\
\end{split}
\end{equation}
where $\Gamma$ is a (product of) gamma matrix.

\subsubsection{\label{MesonCorrelatorSub}Meson Correlator}

The correlator is defined as
\begin{equation}
\begin{split}
&C(x,y)=\langle \bar{O}(x)O(y)\rangle
\end{split}
\end{equation}
where
\begin{equation}
\begin{split}
&\langle W \rangle = \frac{1}{Z}\int \mathcal{D}[U,\bar{\psi},\psi] W \exp \left(-S\right)\\
&Z=\int \mathcal{D}[U,\bar{\psi},\psi] \exp \left(-S\right),\;\;S=S_G+S_{pf}\\
\end{split}
\end{equation}

\begin{itemize}
  \item iso-triplet
\end{itemize}

Denote the variables as $C_T$ and $O_T$.

We need to calculate (green variables are constant)
\begin{equation}
\begin{split}
&C_T(n,m)=\langle \bar{\psi} ^{f_1} (n)\textcolor[rgb]{0,0.5,0}{\Gamma}\psi ^{f_2} (n)\bar{\psi} ^{f_2} (m)\textcolor[rgb]{0,0.5,0}{\Gamma}\psi ^{f_1}(m)\rangle \\
&=\sum _{{a,b,c}_i}\textcolor[rgb]{0,0.5,0}{\Gamma _{a_1,b_1}}\textcolor[rgb]{0,0.5,0}{\Gamma _{a_2,b_2}}\langle \bar{\psi} ^{f_1} _{a_1,c_1} (n)\psi ^{f_2} _{b_1,c_1} (n)\bar{\psi} ^{f_2} _{a_2,c_2}(m)\psi ^{f_1}_{b_2,c_2}(m)\rangle\\
\end{split}
\end{equation}

Note that, they are all Grassman numbers (exchange three times will introduce a minus sign), and they can be averaged according to different fields, so
\begin{equation}
\begin{split}
&C_T(n,m)=-\sum _{{a,b,c}_i}\textcolor[rgb]{0,0.5,0}{\Gamma _{a_1,b_1}}\textcolor[rgb]{0,0.5,0}{\Gamma _{a_2,b_2}}\langle \psi ^{f_2} _{b_1,c_1} (n)\bar{\psi} ^{f_2} _{a_2,c_2}(m)\rangle _{f_1}\langle \psi ^{f_1}_{b_2,c_2}(m)\bar{\psi} ^{f_1} _{a_1,c_1} (n)\rangle _{f_2}\\
\end{split}
\end{equation}

Using the Wick theorem for Grassman numbers ($f$ is flavour index, $c$ is color index, $a,b$ are spinor index).
\begin{equation}
\begin{split}
&\langle \ldots \rangle = \frac{1}{Z_f}\int \mathcal{D}\left[\psi\right]\ldots \exp \left(\textcolor[rgb]{1,0,0}{-}\sum _{l,m}^N\bar{\psi}_lM_{lm}\psi_m\right).\\
&\langle \psi _{i_1}\ldots \psi _{i_n} \bar{\psi}_{j_1}\ldots \bar{\psi}_{j_n}\rangle = \sum _{P}{\rm sign}(P)\prod ^N_n \left(M^{-1}\right)_{i_n,j_{P_n}}.\\
&\langle \psi^f (n)_{a,c_1}\bar{\psi}^f_{b,c_2}(m)\rangle = -D_{f,a,b,c_1,c_2}^{-1}(n,m).\\
\end{split}
\end{equation}
Then we can multiply gamma matrix back
\begin{equation}
\begin{split}
&C_T(n,m)=-{\rm tr}_{c,s} \left[\Gamma D^{-1}_{f_1}(n,m)\Gamma D^{-1}_{f_2}(m,n)\right]
\end{split}
\end{equation}

The trace is for both color and spinor space.

\begin{itemize}
  \item iso-singlet
\end{itemize}

Denote the variables as $C_S$ and $O_S$.

\subsubsection{\label{MonteCarloInMonteCarlo}Monte Carlo in Monte Carlo}

Trace estimator.