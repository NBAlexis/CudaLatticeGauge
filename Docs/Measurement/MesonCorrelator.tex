\subsection{\label{MesonCorrelator}\index{meson}\index{correlator}Meson Correlator}

\subsubsection{\label{MesonWaveFunction}Meson Wave Function}

We need at first construct an observable which is a bound state of two fermions and \textbf{has the same quantum number} as mesons. In short, we want to know
\begin{equation}
\begin{split}
&O(x)=\bar{\psi}(x)\Gamma \psi (x)\\
\end{split}
\end{equation}
where $\Gamma$ is a (product of) gamma matrix.

\subsubsection{\label{MesonCorrelatorSub}Meson Correlator}

The correlator is defined as
\begin{equation}
\begin{split}
&C(x,y)=\langle \bar{O}(x)O(y)\rangle
\end{split}
\end{equation}
where
\begin{equation}
\begin{split}
&\langle W \rangle = \frac{1}{Z}\int \mathcal{D}[U,\bar{\psi},\psi] W \exp \left(-S\right)\\
&Z=\int \mathcal{D}[U,\bar{\psi},\psi] \exp \left(-S\right),\;\;S=S_G+S_{pf}\\
\end{split}
\end{equation}

\begin{itemize}
  \item iso-triplet
\end{itemize}

Denote the variables as $C_T$ and $O_T$.

We need to calculate (green variables are constant)
\begin{equation}
\begin{split}
&C_T(n,m)=\langle \bar{\psi} ^{f_1} (n)\textcolor[rgb]{0,0.5,0}{\Gamma}\psi ^{f_2} (n)\bar{\psi} ^{f_2} (m)\textcolor[rgb]{0,0.5,0}{\Gamma}\psi ^{f_1}(m)\rangle \\
&=\sum _{{a,b,c}_i}\textcolor[rgb]{0,0.5,0}{\Gamma _{a_1,b_1}}\textcolor[rgb]{0,0.5,0}{\Gamma _{a_2,b_2}}\langle \bar{\psi} ^{f_1} _{a_1,c_1} (n)\psi ^{f_2} _{b_1,c_1} (n)\bar{\psi} ^{f_2} _{a_2,c_2}(m)\psi ^{f_1}_{b_2,c_2}(m)\rangle\\
\end{split}
\end{equation}

Note that, they are all Grassman numbers (exchange three times will introduce a minus sign), and they can be averaged according to different fields, so
\begin{equation}
\begin{split}
&C_T(n,m)=-\sum _{{a,b,c}_i}\textcolor[rgb]{0,0.5,0}{\Gamma _{a_1,b_1}}\textcolor[rgb]{0,0.5,0}{\Gamma _{a_2,b_2}}\langle \psi ^{f_2} _{b_1,c_1} (n)\bar{\psi} ^{f_2} _{a_2,c_2}(m)\rangle _{f_1}\langle \psi ^{f_1}_{b_2,c_2}(m)\bar{\psi} ^{f_1} _{a_1,c_1} (n)\rangle _{f_2}\\
\end{split}
\end{equation}

Using the Wick theorem for Grassman numbers ($f$ is flavour index, $c$ is color index, $a,b$ are spinor index).
\begin{equation}
\begin{split}
&\langle \ldots \rangle = \frac{1}{Z_f}\int \mathcal{D}\left[\psi\right]\ldots \exp \left(\textcolor[rgb]{1,0,0}{-}\sum _{l,m}^N\bar{\psi}_lM_{lm}\psi_m\right).\\
&\langle \psi _{i_1}\ldots \psi _{i_n} \bar{\psi}_{j_1}\ldots \bar{\psi}_{j_n}\rangle = \sum _{P}{\rm sign}(P)\prod ^N_n \left(M^{-1}\right)_{i_n,j_{P_n}}.\\
&\langle \psi^f (n)_{a,c_1}\bar{\psi}^f_{b,c_2}(m)\rangle = -D_{f,a,b,c_1,c_2}^{-1}(n,m).\\
\end{split}
\end{equation}
Then we can multiply gamma matrix back
\begin{equation}
\begin{split}
&C_T(n,m)=-{\rm tr}_{c,s} \left[\Gamma D^{-1}_{f_1}(n,m)\Gamma D^{-1}_{f_2}(m,n)\right]
\end{split}
\end{equation}

The trace is for both color and spinor space.

\begin{itemize}
  \item iso-singlet
\end{itemize}

Denote the variables as $C_S$ and $O_S$.

\subsubsection{\label{Source}\index{source}Sources}

\begin{itemize}
  \item Fourier transform
\end{itemize}

Usually, one need to know the observable in momentum space, which is
\begin{equation}
\begin{split}
&\tilde{C}({\bf p},n_t;{\bf 0}, 0)\equiv \frac{1}{\sqrt{\Lambda _3}}\sum _{{\bf n}\in \Lambda _3}\exp (-ia {\bf n}\cdot {\bf p})C({\bf n},n_t;{\bf 0},0)\\
\end{split}
\label{eq.mesoncorrelator.Fourier}
\end{equation}
where $\Lambda _3$ denotes the spatial lattice.

For hadron spectroscopy,
\begin{equation}
\begin{split}
&\tilde{C}({\bf p},n_t;{\bf 0}, 0)\propto \exp\left(-an_tE_0({\bf p})\right)\times \left(1+\mathcal{O}\left(e^{-an_t\Delta E}\right)\right)
\end{split}
\end{equation}
where $E_0({\bf p})$ is the ground state energy (dissipative relation?) and $\Delta E$ is the energy gap between ground state and the lowest excitation, and
\begin{equation}
\begin{split}
&E_0({\bf p})=\sqrt{m_H^2+|{\bf p}|^2}\times \left(1+\mathcal{O}\left(a|{\bf p}|\right)\right)\\
\end{split}
\end{equation}

For zero momentum, we find $m_H$. That is why the lattice at $t$-dir is usually larger than the spatial directions.

From Eq.~(\ref{eq.mesoncorrelator.Fourier}), we only need to calculate $C(n,0)$ for all $n$. That is a \index{point source}\textbf{point source}.

Using $\{\gamma _{\mu},\gamma _5\}=0$ and $\gamma _5^2=1$, $\{\gamma _{\mu},\gamma _5\}=0$, so
\begin{equation}
\begin{split}
&\left(\Gamma D^{-1}(n,m)\Gamma D^{-1}(m,n)\right)=\left(\Gamma D^{-1}(n,m) \Gamma \gamma _5 \left(D^{-1}(n,m)\right)^{\dagger}\gamma _5\right)\\
&{\rm tr}_{c,s}\left[\Gamma D^{-1}(n,m) \Gamma \gamma _5 \left(D^{-1}(n,m)\right)^{\dagger}\gamma _5\right]={\rm tr}_{c,s}\left[\gamma _5\Gamma D^{-1}(n,m) \Gamma \gamma _5 \left(D^{-1}(n,m)\right)^{\dagger}\right]\\
&=\textcolor[rgb]{0,0,1}{\pm} {\rm tr}_{c,s}\left[\Gamma' D^{-1}(n,m) \Gamma' \left(D^{-1}(n,m)\right)^{\dagger}\right]\\
&=\textcolor[rgb]{0,0.5,0}{\pm} {\rm tr}_{c,s}\left[\Gamma'^{\dagger} D^{-1}(n,m) \Gamma' \left(D^{-1}(n,m)\right)^{\dagger}\right]\\
\end{split}
\end{equation}
where $\Gamma ' = \Gamma \gamma _5$ and $\textcolor[rgb]{0,0,1}{\pm}$ come from $\gamma _5\Gamma=\pm \Gamma \gamma _5$, and $\textcolor[rgb]{0,0.5,0}{\pm}$ come from both $\gamma _5\Gamma=\pm \Gamma \gamma _5$ and $\Gamma ^{\dagger}=\pm \Gamma ^{\dagger}$. Note that it is in fact a \textbf{real} number because
\begin{equation}
\begin{split}
&{\rm tr}_{c,s}\left[\Gamma'^{\dagger} D^{-1}(n,m) \Gamma' \left(D^{-1}(n,m)\right)^{\dagger}\right]={\rm tr}_{c,s}\left[D^{-1}(n,m) \Gamma' \left(D^{-1}(n,m)\right)^{\dagger}\Gamma'^{\dagger} \right]\\
&\left({\rm tr}_{c,s}\left[\Gamma'^{\dagger} D^{-1}(n,m) \Gamma' \left(D^{-1}(n,m)\right)^{\dagger}\right]\right)^*={\rm tr}_{c,s}\left[\left(D^{-1}(n,m) \Gamma' \left(D^{-1}(n,m)\right)^{\dagger}\Gamma'^{\dagger} \right)^{\dagger}\right]\\
&={\rm tr}_{c,s}\left[\Gamma' D^{-1}(n,m) {\Gamma'}^{\dagger} \left(D^{-1}(n,m)\right)^{\dagger}\right]={\rm tr}_{c,s}\left[{\Gamma'}^{\dagger} D^{-1}(n,m) \Gamma' \left(D^{-1}(n,m)\right)^{\dagger}\right]
\end{split}
\end{equation}
With point source, we need only to calculate $D^{-1}(n,0)$, which is a $12\times 12=144$ elements matrix field on each site, with the matrix element
\begin{equation}
\begin{split}
&D^{-1}(n,m_0)_{c_1,c_2,s_1,s_2}=\sum _{m,c_3,s_3}D^{-1}(n,m)_{c_1,c_3,s_1,s_3}\left(S(m_0,c_2,s_2;m,c_3,s_3)\right)\\
&D^{-1}(n,m_0)_{:,c_2,:,s_2}=D^{-1} \phi ^{S}_{m_0,c_2,s_2}\\
\end{split}
\end{equation}

In the last line, $:,c_2,:,s_2$ denote one column of the $12\times 12$ matrix, and $\phi ^{S}_{m_0,c_2,s_2}$ is \index{pseudofermions}pseudo-fermion field with only one none-zero element (the \index{point source}\textbf{point source} at $m_0$, in our case, $m_0=({\bf 0},0)$)
\begin{equation}
\begin{split}
&\phi ^{S}_{m_0,c_2,s_2}(m)_{c,s}=\delta (m-m_0)\delta (c-c_2)\delta (s-s_2)\\
\end{split}
\end{equation}
In matrix form it is
\begin{equation}
\begin{split}
&\left(\begin{array}{c} D^{-1}_{1,cs} \\ D^{-1}_{2,cs} \\ D^{-1}_{3,cs} \\ \ldots \\ D^{-1}_{10,cs}\\ D^{-1}_{11,cs} \\ D^{-1}_{12,cs_2}\end{array}\right)=\left(\begin{array}{ccccccc}
D^{-1}_{1,1} & D^{-1}_{1,2} & \ldots & \textcolor[rgb]{1,0,0}{D^{-1}_{1,cs}} & \ldots &  D^{-1}_{1,11} & D^{-1}_{1,12} \\
D^{-1}_{2,1} & D^{-1}_{2,2} & \ldots & \textcolor[rgb]{1,0,0}{D^{-1}_{2,cs}} & \ldots &  D^{-1}_{2,11} & D^{-1}_{2,12} \\
D^{-1}_{3,1} & D^{-1}_{3,2} & \ldots & \textcolor[rgb]{1,0,0}{D^{-1}_{3,cs}} & \ldots &  D^{-1}_{3,11} & D^{-1}_{3,12} \\
\ldots & \ldots & \ldots & \ldots & \ldots &  \ldots & \ldots \\
D^{-1}_{10,1} & D^{-1}_{0,2} & \ldots & \textcolor[rgb]{1,0,0}{D^{-1}_{10,cs}} & \ldots &  D^{-1}_{10,11} & D^{-1}_{10,12} \\
D^{-1}_{11,1} & D^{-1}_{11,2} & \ldots & \textcolor[rgb]{1,0,0}{D^{-1}_{11,cs}} & \ldots &  D^{-1}_{11,11} & D^{-1}_{11,12} \\
D^{-1}_{12,1} & D^{-1}_{12,2} & \ldots & \textcolor[rgb]{1,0,0}{D^{-1}_{12,cs}} & \ldots &  D^{-1}_{12,11} & D^{-1}_{12,12} \\
 \end{array}\right)\left(\begin{array}{c} 0\\ \ldots \\ 0 \\ 1 _{idx=cs}\\ 0 \\ \ldots \\ 0 \end{array}\right)
\end{split}
\label{eq.mesoncorrelator.matrix1}
\end{equation}

So, we need $12$ point sources to fill the $12\times 12$ matrix. Now, for each site, we can calculate the trace, and obtain a \textbf{real} field defined on sites
\begin{equation}
\begin{split}
&\Phi (n) = {\rm tr}_{c,s}\left(\Gamma D^{-1}\Gamma D^{-1}\right)
\end{split}
\end{equation}

The final step is to sum the spatial lattice with the weight $e^{-ia{\bf n}\cdot {\bf p}}$ for each $n_t$, note that, the result should be calculated for each (assume periodic boundary condition for spatial directions)
\begin{equation}
\begin{split}
&{\bf p}\in \left\{(p_1,p_2,p_3)|p_i=\frac{2\pi}{a N_i}k_i, k_i=-\frac{N_i}{2}-1,\ldots \frac{N_i}{2}\right\}
\end{split}
\end{equation}
where $N_i$ is the length of the lattice at $i$ direction.

Therefor, $\tilde{C}({\bf p})$ is a complex field defined on spatial \index{reciprocal space}\textbf{reciprocal space}.

For spectroscopy, we need only the data for ${\bf p}=0$, because when $\Delta E\ll 1$
\begin{equation}
\begin{split}
&\tilde{C}(n_t)\equiv \tilde{C}({\bf 0},n_t;{\bf 0},0)\propto \exp \left(-a n_t m_H\right).
\end{split}
\end{equation}

Note that, for $k_i=0$, we need \textbf{even} number of length at spatial directions.

\begin{itemize}
  \item Detail of implementation
\end{itemize}

We can write $D^{-1}$ in the form of $4\times 4$ matrices, with elements as $3\times 3$ matrices, as
\begin{equation}
\begin{split}
&D^{-1}=\left(\begin{array}{cccc}
U_{11} & U_{12} & U_{13} & U_{14} \\
U_{21} & U_{22} & U_{23} & U_{24} \\
U_{31} & U_{32} & U_{33} & U_{34} \\
U_{41} & U_{42} & U_{43} & U_{44} \\
\end{array}\right)
\end{split}
\end{equation}
If our pseudo-fermion field is organized as
\begin{equation}
\begin{split}
&D^{-1}\phi^S\equiv \phi ^{s\times 3+c}(n)=\left(d_0,d_1,d_2,d_3\right), d_s=\left(v_0,v_1,v_2\right)\\
\end{split}
\end{equation}
Using Eq.~(\ref{eq.mesoncorrelator.matrix1}), one have
\begin{equation}
\begin{split}
&U_{ij}=\left(\begin{array}{ccc}
\phi ^{j\times 3 + 0} (d_i,v_0) & \phi ^{j\times 3 + 1}(d_i,v_0) & \phi ^{j\times 3 + 2}(d_i,v_0)  \\
\phi ^{j\times 3 + 0} (d_i,v_1) & \phi ^{j\times 3 + 1}(d_i,v_1) & \phi ^{j\times 3 + 2}(d_i,v_1)  \\
\phi ^{j\times 3 + 0} (d_i,v_2) & \phi ^{j\times 3 + 1}(d_i,v_2) & \phi ^{j\times 3 + 2}(d_i,v_2)  \\
\end{array}\right),\\
&U_{ij}^T=\left(\begin{array}{ccc}
\phi ^{j\times 3 + 0} (d_i,v_0) & \phi ^{j\times 3 + 0}(d_i,v_1) & \phi ^{j\times 3 + 0}(d_i,v_2)  \\
\phi ^{j\times 3 + 1} (d_i,v_0) & \phi ^{j\times 3 + 1}(d_i,v_1) & \phi ^{j\times 3 + 1}(d_i,v_2)  \\
\phi ^{j\times 3 + 2} (d_i,v_0) & \phi ^{j\times 3 + 2}(d_i,v_1) & \phi ^{j\times 3 + 2}(d_i,v_2)  \\
\end{array}\right)\\
\end{split}
\end{equation}
The $\Gamma$ intersect between $D^{-1}$ and $\left(D^{-1}\right)^{\dagger}$ is a permutation of rows in spinor space, for example
\begin{equation}
\begin{split}
&\gamma _1=\left(\begin{array}{cccc}
0 & 0 & 0 & c_1 \\
0 & 0 & c_2 & 0 \\
0 & c_3 & 0 & 0 \\
c_4 & 0 & 0 & 0 \\
\end{array}\right),\;\;\begin{array}{cccc} p(1)=4,&p(2)=3,&p(3)=2,&p(4)=1 \\ c_1=-i, & c_2=-i, &c_3=i, &c_4=i \end{array}
\end{split}
\end{equation}
where $c_i$ are coefficients and $c_i\in \mathbb{Z}_4$ group. $p(i)=j$ denote that the none-zero of the $i$-th row is $j$-element, (also, the none-zero of the $i$-th column is $j$-element, because $\gamma _{\mu}^{\dagger}=\gamma _{\mu}$.).

So one have such that
\begin{equation}
\begin{split}
&\Gamma ' D^{-1}=\left(\begin{array}{cccc}
c_1U_{p(1)1} & c_1U_{p(1)2} & c_1U_{p(1)3} & c_1U_{p(1)4} \\
c_2U_{p(2)1} & c_2U_{p(2)2} & c_2U_{p(2)3} & c_2U_{p(2)4} \\
c_3U_{p(3)1} & c_3U_{p(3)2} & c_3U_{p(3)3} & c_3U_{p(3)4} \\
c_4U_{p(4)1} & c_4U_{p(4)2} & c_4U_{p(4)3} & c_4U_{p(4)4} \\
\end{array}\right),\\
&\Gamma ' D^{-1} {\Gamma'}^{\dagger}=\left(\begin{array}{cccc}
c_1c^*_1U_{p(1)p(1)} & c_1c^*_2U_{p(1)p(2)} & c_1c^*_3U_{p(1)p(3)} & c_1c^*_4U_{p(1)p(4)} \\
c_2c^*_1U_{p(2)p(1)} & c_2c^*_2U_{p(2)p(2)} & c_2c^*_3U_{p(2)p(3)} & c_2c^*_4U_{p(2)p(4)} \\
c_3c^*_1U_{p(3)p(1)} & c_3c^*_2U_{p(3)p(2)} & c_3c^*_3U_{p(3)p(3)} & c_3c^*_4U_{p(3)p(4)} \\
c_4c^*_1U_{p(4)p(1)} & c_4c^*_2U_{p(4)p(2)} & c_4c^*_3U_{p(4)p(3)} & c_4c^*_4U_{p(4)p(4)} \\
\end{array}\right)\\
\end{split}
\end{equation}

Finally we have (Note $U$ is not a $SU(3)$ matrix)
\begin{equation}
\begin{split}
&{\rm tr}_{c,s}\left[\Gamma ' D^{-1}{\Gamma '}^{\dagger}\left(D^{-1}\right)^{\dagger}\right] = \sum _{ij}c_ic_j^* {\rm tr}_c\left[U_{p(i)p(j)}U^{\dagger}_{ij}\right]=\sum _{ij}c_ic_j^* {\rm tr}_c\left[U^{\dagger}_{ij}U_{p(i)p(j)}\right]
\end{split}
\end{equation}

This can be further simplified, note that the result should be a real number, so for $i\neq j$, if ${\rm tr}\left[U^{\dagger}_{p(i)p(j)}U_{ij}\right]$ is present, so must be ${\rm tr}\left[U^{\dagger}_{ij}U_{p(i)p(j)}\right]$ with the same sign. This is guaranteed by symmetric matrix, i.e. if $p(i)=a$, one must have $p(a)=i$, and also $c_ic^*_j=c_{p(i)}c_{p(j)}^*$ as shown below.

To prove $c_ic^*_j=c_{p(i)}c_{p(j)}^*$, we need to consider:

\begin{enumerate}
  \item $\Gamma ^{\dagger}=\pm \Gamma$ and $p(i)=i$. In this case, $c_i=c_{p(i)}$, and $c_ic^*_j=c_{p(i)}c_{p(j)}^*$ is straightforward.
  \item $\Gamma ^{\dagger}=\Gamma$ and $p(i)\neq i$. In this case, $c_i=c_{p(i)}^*$, so $c_ic^*_j=c^*_{p(i)}c_{p(j)}$. Note that, $c_i$ are either all real or all imaginary, so $c_ic^*_j=c^*_{p(i)}c_{p(j)}=c_{p(i)}c_{p(j)}^*$.
  \item $\Gamma ^{\dagger}=-\Gamma$ and $p(i)\neq i$. In this case, $c_i=-c_{p(i)}^*$, so $c_ic^*_j=c^*_{p(i)}c_{p(j)}=c_{p(i)}c_{p(j)}^*$.
\end{enumerate}

So, we have two cases, one for $p(1)=1$, and one for $p(1)\neq 1$
\begin{equation}
\begin{split}
&{\rm tr}_{c,s}\left[\Gamma ' D^{-1}{\Gamma '}^{\dagger}\left(D^{-1}\right)^{\dagger}\right]\\
& = \left\{\begin{array}{cc}
2\sum _{i>1,j>i}c_ic_j^* {\rm Retr}_c\left[U^{\dagger}_{ij}U_{p(i)p(j)}\right]+\sum _{i=1,2,3,4}{\rm tr}_c\left[U^{\dagger}_{ii}U_{ii}\right] & p(i)=i\\
2\sum _{i>1,j>i}c_ic_j^* {\rm Retr}_c\left[U^{\dagger}_{ij}U_{p(i)p(j)}\right]+2\sum _{i=1,k}{\rm Retr}_c\left[U^{\dagger}_{ii}U_{p(i)p(i)}\right] & p(1)\neq 1,k\\
\end{array}\right.
\end{split}
\end{equation}


\subsubsection{\label{Gauge_Smearing}\index{gauge smearing}Gauge smearing}

In HMC with fermions, the computer power is consumed mainly in solving the $D^{-1}$. The small eigenvalues of the $D$ operator is the main reason to slow down the solver, which is the so called \index{low mode}\textbf{low mode} or \index{exceptional configurations}\textbf{exceptional configurations}.

Gauge smearing (\index{gauge smoothing}gauge smoothing) is one of the method to ease the problem by replacing the original configuration with a gauge equivalent but easier configuration. There are several different smearing methods. In CLGLib, only two are implemented.

\begin{itemize}
  \item \index{APE smearing}APE
\end{itemize}

It use
\begin{equation}
\begin{split}
&U_{\mu}'=\mathcal{P}\left((1-\alpha)U_{\mu} + \frac{\alpha}{6}\Sigma _{\mu}\right)
\end{split}
\end{equation}
where $\Sigma _{\mu}$ is the \index{staple}staple, (see Eq.~(\ref{eq.plaqutteEnergy.staple})). In CLGLib, staples are cached. After smoothing, $\mathcal{P}$ is a projection project to result to $SU(3)$ and can be approximated as
\begin{algorithm}[H]
\begin{algorithmic}
\State $U =U / \sqrt{tr(U^{\dagger}U / 3)}$
\For{$i=0$ to $r$}
\Comment {iterate r times}
    \State{$x = U \left(\frac{3}{2}-\frac{1}{2}U^{\dagger}U\right)$}
    \State{$U = \left(1-\frac{i}{3}{\rm Im}(\det(x))\right)x$}
\EndFor

\Return $U$
\end{algorithmic}
\caption{$\mathcal{P}(U)$ approximately}
\end{algorithm}

Usually, iterate for $4$ times, it can archive $\mathcal{\alpha}$ accuracy.

\begin{itemize}
  \item \index{stout}\index{APE stout}APE stout
\end{itemize}

In this approach, it construct a $SU(3)$ candidate directly by the staples. (Therefor, no need to project). Using

\begin{equation}
\begin{split}
&\Omega _{\mu}=\rho _{\mu} \Sigma _{\mu} U_{\mu}^{\dagger},\;Q_{\mu}=\left\{\Omega_{\mu}\right\}_{TA},\;U_{\mu}'=\exp (Q_{\mu}) U_{\mu}\\
\end{split}
\end{equation}
where $\rho_{\mu}$ is usually set to be $\rho_{1,2,3}=\rho,\rho_{4}=0$. Note that, there is no sum over $\mu$ in the above equation. Also, note that, $\exp$ is not accurate unless $\rho$ is small enough, however, one can iterate the smearing for a few sub-steps.

