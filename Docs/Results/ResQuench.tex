\subsection{\label{resquench}Phase transition quenched}

\begin{table}
\begin{center}
\begin{tabular}{c|c|c|c|c}
    \hline    
$\beta$ & $T$ & $L_{\rm bare}$ & $L_{\rm ren}$ & $\chi _P$\\
& (MeV) & & & $(a^3)$ \\
\hline
$5.45$ & $113.4(3)$ & $0.00770(4)$   & $0.049(5)$  & $0.0277(4)$ \\
\hline
$5.5$  & $123.7(3)$ & $0.00791(4)$   & $0.050(2)$  & $0.0291(5)$ \\
\hline
$5.55$ & $135.6(2)$ & $0.00812(5)$   & $0.054(1)$  & $0.0318(5)$ \\
\hline
$5.6$  & $151.8(2)$ & $0.00869(6)$   & $0.061(1)$  & $0.0361(7)$ \\
\hline
$5.65$ & $170.4(2)$ & $0.00966(9)$   & $0.070(1)$  & $0.0444(12)$ \\ 
\hline
$5.7$  & $192.5(2)$ & $0.01065(12)$  & $0.080(1)$  & $0.0432(18)$ \\
\hline
$5.75$ & $217.8(2)$ & $0.01309(26)$  & $0.100(2)$  & $0.0804(45)$ \\
\hline
$5.8$  & $242.5(2)$ & $0.01826(82)$  & $0.139(6)$  & $0.1658(220)$ \\
\hline
$5.85$ & $270.4(3)$ & $0.02808(293)$ & $0.215(22)$ & $0.4412(1177)$ \\
\hline
$\textcolor[rgb]{1,0,0}{5.9}$  & $297.6(3)$ & $0.05979(314)$ & $0.451(24)$ & $\textcolor[rgb]{1,0,0}{0.8096(1522)}$ \\
\hline
$5.95$ & $325.5(3)$ & $0.08117(151)$ & $0.602(11)$ & $0.4689(736)$ \\
\hline
$6.0$  & $356.9(4)$ & $0.09258(149)$ & $0.674(11)$ & $0.4229(612)$ \\
\hline
$6.05$ & $385.2(5)$ & $0.10293(124)$ & $0.731(9)$  & $0.3955(455)$ \\
\hline
\end{tabular}
\caption{\label{tab.res.quenchphasetransition}Result of phase transition, quenched. critical temperature is $297.6\;{\rm MeV}$.}
\end{center}
\end{table}

\begin{table}
\begin{center}
\begin{tabular}{c|c|c|c|c|c}
    \hline    
$\beta$ & $T$ & $\tau _{ind}$ & $L_{\rm bare}$ & $L_{\rm ren}$ & $\chi _P/T^3$\\
& (MeV) & & & \\
\hline
$5.6$ & $151.6(1)$ & $  0.7835$ & $0.00869(1)$   & $0.0607(14)$ & $7.70\pm 0.04$ \\
\hline
$5.7$ & $192.2(1)$ & $  3.5950$ & $0.01100(4)$   & $0.0794(4)$  & $12.55\pm 0.14$ \\
\hline
$5.8$ & $241.4(1)$ & $ 38.7015$ & $0.01736(22)$  & $0.1290(17)$ & $35.77\pm 1.48$ \\
\hline
\textcolor[rgb]{0,0,1}{$5.9$} & $295.0(1)$ & $233.7995$ & $0.05411(134)$ & $0.3981(98)$ & \textcolor[rgb]{0,0,1}{$213.31\pm 12.73$} \\
\hline
$6.0$ & $350.3(1)$ & $ 53.4871$ & $0.09317(44)$  & $0.6617(31)$ & $99.26\pm 4.44$ \\ 
\hline
$6.1$ & $410.9(2)$ & $ 32.3169$ & $0.11347(31)$  & $0.7732(22)$ & $85.84\pm 2.92$ \\
\hline
$6.2$ & $478.7(2)$ & $ 29.1558$ & $0.12892(30)$  & $0.8408(20)$ & $85.17\pm 2.56$ \\
\hline
$6.3$ & $531.7(3)$ & $ 24.4408$ & $0.14375(27)$  & $0.8826(17)$ & $81.71\pm 2.26$ \\
\hline
$6.4$ & $595.7(4)$ & $ 21.3246$ & $0.15759(25)$  & $0.9198(15)$ & $82.23\pm 2.14$ \\
\hline
$6.5$ & $643.9(4)$ & $ 30.4303$ & $0.16978(31)$  & $0.9358(17)$ & $89.41\pm 3.54$ \\
\hline
\end{tabular}
\caption{\label{tab.res.quenchphasetransition2}Result of phase transition, quenched. critical temperature is $295.0\;{\rm MeV}$.}
\end{center}
\end{table}


The first set of results are obtained by using $12^3 \times 6$ with torus periodic boundary.
$L_{\rm bare}$ is calculated with autocorrelation, $S=1.5$, however, when calculate $L_{\rm ren}$, the $c(\beta)$ is introduced without autocorrelation.
$2000+(100+9900)\times 13$ configurations are generated.
The results are in Table.~\ref{tab.res.quenchphasetransition}.

The first set of results are obtained by using $2000+(100+149900)\times 10$.
The results are in Table.~\ref{tab.res.quenchphasetransition2}.
