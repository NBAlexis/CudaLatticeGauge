\subsection{\label{sec:ConstantAcceleration}Constant Acceleration}

\subsubsection{\label{sec:discretizationAcceleration}Discretization}

The metric of constant acceleration $g$ at ${\bf z}$ direction is
\begin{equation}
\begin{split}
&g_{\mu\nu}=\left(\begin{array}{cccc} 1-g^2 t^2 & 0 & 0 & g t \\ 0 & -1 & 0 & 0 \\ 0 & 0 & -1 & 0 \\ g t & 0 & 0 & -1 \end{array}\right),\;\;\sqrt{-\det (g_{\mu\nu})}=i\\
\end{split}
\end{equation}
The analytical extension is simply add $-2$ to $g_{00}$, \begin{equation}
\begin{split}
&g_{\mu\nu}=\left(\begin{array}{cccc} -1-g^2 t^2 & 0 & 0 & g t \\ 0 & -1 & 0 & 0 \\ 0 & 0 & -1 & 0 \\ g t & 0 & 0 & -1 \end{array}\right),\;\;\sqrt{-\det (g_{\mu\nu})}=1\\
\end{split}
\end{equation}
Then
\begin{equation}
\begin{split}
&\mathcal{L}=\frac{1}{2g_{YM}^2}\left(\sum _{ijkl=0}^3 \delta_{ij}\delta_{kl}{\rm tr}[F_{ik}F_{jl}]+2g^2 t^2 {\rm tr}\left[F_{01}^2+F_{02}^2\right]\right.\\
&\left.+4g t\left({\rm tr}[F_{01}F_{13}]+{\rm tr}[F_{02}F_{23}]\right)\right).
\end{split}
\end{equation}
And
\begin{equation}
\begin{split}
&i\int d^4 x \sqrt{-\det (g_{\mu\nu})}\mathcal{L}=-\frac{1}{2g_{YM}^2}\int d^4x \left(\sum _{ijkl=0}^3 \delta_{ij}\delta_{kl}{\rm tr}[F_{ik}F_{jl}]+2g^2 t^2 {\rm tr}\left[F_{01}^2+F_{02}^2\right]\right.\\
&\left.+4g t\left({\rm tr}[F_{01}F_{13}]+{\rm tr}[F_{02}F_{23}]\right)\right).
\end{split}
\end{equation}
Let $\exp (i\int d^4 x \mathcal{L}) \to \exp (-S_F)$, it is
\begin{equation}
\begin{split}
&S_G=\frac{1}{2g_{YM}^2}\int d^4x \left(\sum _{ijkl=0}^3 \delta_{ij}\delta_{kl}{\rm tr}[F_{ik}F_{jl}]+2g^2 t^2 {\rm tr}\left[F_{01}^2+F_{02}^2\right]\right.\\
&\left.+4g t\left({\rm tr}[F_{01}F_{13}]+{\rm tr}[F_{02}F_{23}]\right)\right).
\end{split}
\end{equation}

The discretized version is then
\begin{equation}
\begin{split}
&\sum _{ijkl=0}^3 \delta_{ij}\delta_{kl}{\rm tr}[F_{ik}F_{jl}] = 2\sum _{\mu>\nu}{\rm tr}[F_{\mu\nu}^2]\\
&{\rm Retr}[1-U_{\mu\nu}]={\rm Retr}[1-\bar{U}_{\mu\nu}]=2{\rm tr}[F_{\mu\nu}^2]\\
&{\rm Retr}V_{\mu\nu\rho}]=4{\rm tr}[F_{\mu\nu}F_{\nu\rho}]\\
&S_G= a^4 \frac{\beta}{N_c}\sum _n \left(\sum _{\mu>\nu}{\rm Retr}[1-U_{\mu\nu}]+g^2 t^2 \left({\rm Retr}[1-\bar{U}_{14}]+{\rm Retr}[1-\bar{U}_{24}]\right)\right.\\
&\left.+g t\left({\rm Retr}[V_{413}]+{\rm Retr}[V_{423}]\right)\right).\\
\end{split}
\end{equation}
Note that ${\rm Retr}[1-U_{\mu\nu}]={\rm Retr}[1-U_{\nu\mu}]$, ${\rm Retr}[V_{\mu\nu\rho}]={\rm Retr}[V_{\rho\nu\mu}]$

For the fermions, we choose
\begin{equation}
\begin{split}
&e_0=i(1,0,0,g t), \;\; e_1=i(0,1,0,0),\\
&e_2=i(0,0,1,0),\;\; e_3=i(0,0,0,1).\\
\end{split}
\end{equation}
one can verify $\delta _{ab}=e_a^{\mu}e_b^{\nu}g^E_{\mu\nu}$.

The only non-zero derivative of metric is $\partial g_{11} / \partial x_1=-2g^2 t$, $\partial g_{14} / \partial x_1 = g$, so the non-zero Christoffel connection are \begin{equation}
\begin{split}
&\Gamma _{1,14}=\frac{1}{2}\left(\frac{\partial g^{11}}{\partial x_4}+\frac{\partial g^{14}}{\partial x_1}-\frac{\partial g^{14}}{\partial x_1}\right)\\
&\Gamma _{4,11}=\frac{1}{2}\left(\frac{\partial g^{41}}{\partial x_1}+\frac{\partial g^{41}}{\partial x_1}-\frac{\partial g^{11}}{\partial x_4}\right)\\
&\Gamma _{1,41}=\frac{1}{2}\left(\frac{\partial g^{14}}{\partial x_1}+\frac{\partial g^{11}}{\partial x_4}-\frac{\partial g^{41}}{\partial x_1}\right)\\
&\Gamma _{1,11}=\frac{1}{2}\left(\frac{\partial g^{11}}{\partial x_1}\right)\\
&\\
&\Gamma _{1,14}=\frac{1}{2}\left(\frac{\partial g^{14}}{\partial x_1}-\frac{\partial g^{14}}{\partial x_1}\right)=0\\
&\Gamma _{4,11}=\frac{1}{2}\left(\frac{\partial g^{41}}{\partial x_1}+\frac{\partial g^{41}}{\partial x_1}\right)=g\\
&\Gamma _{1,41}=\frac{1}{2}\left(\frac{\partial g^{14}}{\partial x_1}-\frac{\partial g^{41}}{\partial x_1}\right)=0\\
&\Gamma _{1,11}=\frac{1}{2}\left(\frac{\partial g^{11}}{\partial x_1}\right)=-g^2t\\
\end{split}
\end{equation}
Then $\Gamma ^i_{jk}=g_{i l}\Gamma _{l,jk}$, $jk$ can only be $11$.
\begin{equation}
\begin{split}
&\Gamma ^1_{11}=\sum _n g_{1 n}\Gamma _{n,11}=g_{11}\Gamma _{1,11}+g_{14}\Gamma _{4,11}=-1\times (-g^2t) + (-g t)\times g =0\\
\end{split}
\end{equation}
The only non-zero is $\Gamma ^4_{11}$.
\begin{equation}
\begin{split}
&w_{\mu ij}=g_{\alpha \beta}e_i^{\alpha}(\partial _{\mu}e_j^{\beta}+\Gamma ^{\beta}_{\mu\nu}e_j^{\nu})\\
\end{split}
\end{equation}
The $\mu$ of $w$ can only be $1$.
\begin{equation}
\begin{split}
&w_{1 ij}=g_{\alpha \beta}e_i^{\alpha}(\partial _{t}e_j^{\beta}+\Gamma ^{\beta}_{1\nu}e_j^{\nu})\\
&=g_{\alpha 4}e_i^{\alpha}(\partial _{t}e_j^{z}+\Gamma ^{4}_{11}e_j^{t})\\
&=g_{1 4}e_i^{t}(\partial _{t}e_j^{z}+\Gamma ^{4}_{11}e_j^{t})+g_{4 4}e_i^{z}(\partial _{t}e_j^{z}+\Gamma ^{4}_{11}e_j^{t})\\
&=g_{1 4}e_i^{t}(\partial _{t}e_j^{z}-ge_j^{t})+g_{4 4}e_i^{z}(\partial _{t}e_j^{z}-ge_j^{t})\\
\end{split}
\end{equation}
Just look at $\partial _{t}e_j^{z}-ge_j^{t}$, for any $j$, it is zero. So $w=0$.

So the result is just
\begin{equation}
\begin{split}
&D_g=\left[i\gamma ^{\mu}\left((\partial _{\mu}+ieA_{\mu})-\frac{i}{4}\sigma ^{ij}w_{\mu ij}\right)-m\right]
\end{split}
\end{equation}
with $\gamma ^{\mu}=\gamma ^i e_i^{\mu}$ and $w=0$. It is
\begin{equation}
\begin{split}
&D_g=\left[-\sum _i \gamma ^i\left(\partial _i+ieA_i\right)-gt \gamma _0\left(\partial _z+ieA_z\right) -m\right]
\end{split}
\end{equation}
As a summary the Wick rotation is
\begin{equation}
\begin{split}
&-S_F=i\int d^4 x \sqrt{-g_{\alpha \beta}}\bar{\psi } D_a \psi\\
&S_F=\int d^4 x_E \bar{\psi } \left[\gamma _1 (\partial _x+ieA_x)+\gamma _2(\partial _y+ieA_y)+\gamma _3 (\partial _z+ieA_z)+\gamma _4 (\partial _{\tau}+ieA_{\tau})+m\right.\\
&\left.+gt \gamma _4 \left(\partial _z+ieA_z\right)\right] \psi\\
\end{split}
\end{equation}
The first line is Wilson-Dirac operator, satisfying $\gamma _5$ - hermiticity, the last line is discretized as
\begin{equation}
\begin{split}
&\left(\gamma _4\frac{U_{\mu}(n)\delta _{n,n+\mu}-U_{-\mu}(n)\delta _{n,n-\mu}}{2a}\right)^{\dagger}\\
&=\left(\frac{\delta _{n,n+\mu}^{\dagger}U^{\dagger}_{\mu}(n)-\delta _{n,n-\mu} ^{\dagger}U^{\dagger}_{-\mu}(n)}{2a}\right)\gamma _4^{\dagger}\\
&=\left(\frac{\delta _{n+\mu,n}U_{-\mu}(n + \mu)-\delta _{n-\mu,n} U_{\mu}(n-\mu)}{2a}\right)\gamma _4\\
\end{split}
\end{equation}
Do the shift then resemble, it is
\begin{equation}
\begin{split}
&=\left(\frac{\delta _{n,n-\mu}U_{-\mu}(n)-\delta _{n,n+\mu} U_{\mu}(n)}{2a}\right)\gamma _4\\
&=\gamma _4\left(\frac{\delta _{n,n-\mu}U_{-\mu}(n)-\delta _{n,n+\mu} U_{\mu}(n)}{2a}\right)=-\left(\gamma _4\frac{U_{\mu}(n)\delta _{n,n+\mu}-U_{-\mu}(n)\delta _{n,n-\mu}}{2a}\right)\\
\end{split}
\end{equation}
So it is anti-hermiticity, note that $\gamma _5 \gamma _4 \gamma _5 = - \gamma _4$, it is then $\gamma _5$ - hermiticity.

\subsubsection{\label{sec:forceAcceleration}Force}

\begin{itemize}
  \item Gauge force
\end{itemize}

The $\bar{U}_{\mu\nu}$ term:

\begin{equation}
\begin{split}
&\sum _n f(n){\rm Retr}[1-\bar{U}_{\mu, \nu}(n)]=\sum _n\frac{f(n)+f(n+a\mu)+f(n+a\nu)+f(n+a\mu+a\nu)}{4}{\rm Retr}[1-U_{\mu\nu}(n)]\\
\end{split}
\end{equation}

So
\begin{equation}
\begin{split}
&g^2 t^2 {\rm Retr}[1-\bar{U}_{4,1}(n)]=g^2\frac{t^2 + (t+1)^2 + t^2 + (t+1)^2}{4}{\rm Retr}[1-U_{4,1}(n)]=\frac{g^2(2t^2+2t+1)}{2}{\rm Retr}[1-U_{4,1}(n)]\\
&g^2 t^2 {\rm Retr}[1-\bar{U}_{4,2}(n)]=\frac{g^2(2t^2+2t+1)}{2}{\rm Retr}[1-U_{4,2}(n)]\\
\end{split}
\end{equation}
Then
\begin{equation}
\begin{split}
&g(n)=\frac{g^2(2t^2+2t+1)}{2}\\
&\sum _ng(n){\rm Retr}[1-U_{\mu,\nu}(n)]=N\times N_c-\sum _n {\rm Retr}\left[U_{\mu}(n)\Sigma ^{\dagger}_{\mu}(n)\right]\\
&\Sigma _{\mu}(n,\nu)=g(n)U_{\nu}(n)U_{\mu}(n+a\nu)U^{-1}_{\nu}(n+a\mu)+g(n-a\nu)U^{-1}_{\nu}(n-a\nu)U_{\mu}(n-a\nu)U_{\nu}(n+a\mu-a\nu)\\
\end{split}
\end{equation}
Note that there is no sum over $\nu$, and $\nu=1$ or $\nu=2$.

\textcolor[rgb]{1,0,0}{\textbf{Note that, if we put rotation and constant acceleration together, $f(n)$ function will not touch the Dirichlet boundary condition, while the $g(n)$ function will. So although $g(n)$ and $g(n-a\nu)$ have the same form $\frac{g^2(2t^2+2t+1)}{2}$, we will not combine them, because it is possible $g(n)=0$ for Dirichlet boundary.}}

Finally
\begin{equation}
\begin{split}
&F_{\mu=1,2}(n)=-\frac{\beta}{2N_c}\left\{U_{\mu}(n)\Sigma _{\mu}^{\dagger}(n,4)\right\}_{TA},\;\;F_3=0\\
&F_{4}(n)=-\frac{\beta}{2N_c}\left\{U_{4}(n)\sum _{i=1,2}\Sigma _{4}^{\dagger}(n,i)\right\}_{TA}\\
\end{split}
\end{equation}

The $V_{\mu\nu\rho}$ force is derived in the `Rotation' chapter.

\subsubsection{\label{sec:AnotherAcceleration}Rigid Acceleration}

Another metric of constant acceleration $g$ at ${\bf z}$ direction is
\begin{equation}
\begin{split}
&g_{\mu\nu}=\left(\begin{array}{cccc} \left(1 + gz\right)^2 & 0 & 0 & 0 \\ 0 & -1 & 0 & 0 \\ 0 & 0 & -1 & 0 \\ 0 & 0 & 0 & -1 \end{array}\right),\;\;\sqrt{-\det (g_{\mu\nu})}=(1+gz)\\
\end{split}
\end{equation}
The tangent space Wick rotation yields, 
\begin{equation}
\begin{split}
&g_{\mu\nu}=\left(\begin{array}{cccc} -\left(1 + gz\right)^2 & 0 & 0 & 0 \\ 0 & -1 & 0 & 0 \\ 0 & 0 & -1 & 0 \\ 0 & 0 & 0 & -1 \end{array}\right),\;\;\sqrt{-\det (g_{\mu\nu})}=i(1+gz)\\
\end{split}
\end{equation}
Then
\begin{equation}
\begin{split}
&S_G=\frac{1}{2g_{YM}^2}(1+g z)2\left(F_{12}^2+F_{13}^2+F_{23}^2+(1+g z)^2F_{01}^2+(1+g z)^2F_{02}^2+(1+g z)^2F_{03}^2\right)\\
&=\frac{\beta}{N_c}(1+g z){\rm Retr}\left[3-\bar{U}_{12}-\bar{U}_{13}-\bar{U}_{23}+(1+g z)^2\left(3-\bar{U}_{41}-\bar{U}_{42}-\bar{U}_{43}\right)\right]\\
\end{split}
\end{equation}
Since every plaquette is related to coordinate, we use $\bar{U}$ for every coordinate.


The Dirac equation becomes
\begin{equation}
\begin{split}
&D=(1+gz)\left[\frac{1}{1+gz}\gamma ^4D_4+\gamma ^iD_i+i\frac{g}{2(1+gz)}\gamma _4 \sigma ^{43}+m\right]\\
&=\left[\gamma ^4D_4+(1+gz)\gamma ^iD_i+i\frac{g}{2}\gamma _4 \sigma ^{43}+m(1+gz)\right]\\
\end{split}
\end{equation}

As usual, the $\gamma _4$ term is realized as chemical potential $\mu = \frac{ig}{2}\sigma ^{43}$,
\begin{equation}
\begin{split}
&D(n|m)=(1+gz)-\kappa (1+gz)\sum _{i = 1,2,3} T_i \\
&-\kappa \left(e^{\frac{i}{2}ag \sigma ^{12} } (1-\gamma^E _4) U_{\tau}(n)\delta _{n+\tau, m} + e^{-\frac{i}{2}ag \sigma ^{12} } (1+\gamma^E _4) U_{-\tau}(n)\delta _{n-\tau, m}\right),\\
T_{i}&\equiv (1-\gamma _i^E)U_{i}(n)\delta _{n+i,m}+(1+\gamma _i^E)U_{-i}(n)\delta _{n-i,m},\\
\end{split}
\end{equation}

The detail of the last term, we start from
\begin{equation}
\begin{split}
&\gamma ^4D_4+i\frac{g}{2}\gamma _4 \sigma ^{43}\\
\end{split}
\end{equation}
The $D_{\mu}$ in Wilson-Dirac operator is
\begin{equation}
\begin{split}
&\left(\partial _{\mu} + iA_{\mu}(n)\right)\psi(n)\approx  \frac{U_{\mu}(n)\psi(n+a\mu)-U_{-\mu}(n)\psi(n-a\mu)}{2a}\\
& -\frac{U_{\mu}(n)\psi(n+a\mu)+U_{-\mu}(n)\psi(n-a\mu)-2\psi (n)}{2a} \to 0 \\
& \gamma _{\mu}  \left(\partial _{\mu} + iA_{\mu}(n)\right)\psi(n) \\
&\approx \gamma _{\mu}  \frac{U_{\mu}(n)\psi(n+a\mu)-U_{-\mu}(n)\psi(n-a\mu)}{2a} -\frac{U_{\mu}(n)\psi(n+a\mu)+U_{-\mu}(n)\psi(n-a\mu)-2\psi (n)}{2a}\\
&=\frac{\psi}{a} -   \frac{(1+\gamma _{\mu})U_{\mu}(n)\psi(n+a\mu)+(1-\gamma _{\mu})U_{-\mu}(n)\psi(n-a\mu)}{2a}
\end{split}
\end{equation}
The $4$ $\frac{\psi}{a}$ combined with $m$ is $\frac{am\textcolor[rgb]{0,0,1}{+4}}{a}$, this is why the $\kappa$ has this $4$.

Now, we have
\begin{equation}
\begin{split}
&D=m(1+gz)+3(1+gz)\frac{1}{a} + \frac{1}{a} - (1+g z)\sum _{\mu = x,y,z}\frac{(1+\gamma _{\mu})U_{\mu}(n)\psi(n+a\mu)+(1-\gamma _{\mu})U_{-\mu}(n)\psi(n-a\mu)}{2a}\\
&- \frac{(1+\gamma _{t})U_{t}(n)\psi(n+at)+(1-\gamma _{t})U_{-t}(n)\psi(n-at)}{2a}\textcolor[rgb]{1,0,0}{-}i\frac{g}{2} \sigma ^{43}\gamma _4
\end{split}
\end{equation}
The `-' of last term is from anti-commutation. Then
\begin{equation}
\begin{split}
&D=(1+gz)(m+4\frac{1}{a}) - \frac{gz}{a} - (1+g z)\sum _{\mu = x,y,z}\frac{(1+\gamma _{\mu})U_{\mu}(n)\psi(n+a\mu)+(1-\gamma _{\mu})U_{-\mu}(n)\psi(n-a\mu)}{2a}\\
&- \frac{(1+\gamma _{t})U_{t}(n)\psi(n+at)+(1-\gamma _{t})U_{-t}(n)\psi(n-at)}{2a}\textcolor[rgb]{1,0,0}{+}i\frac{g}{2} \sigma ^{\textcolor[rgb]{1,0,0}{34}}\gamma _4
\end{split}
\end{equation}
Still let $\kappa = \frac{1}{2am + 8}$
\begin{equation}
\begin{split}
&2a\kappa D=(1+gz) - 2gz\kappa - \kappa(1+g z)\sum _{\mu = x,y,z}\left((1+\gamma _{\mu})U_{\mu}(n)\psi(n+a\mu)+(1-\gamma _{\mu})U_{-\mu}(n)\psi(n-a\mu)\right)\\
&- \kappa\left((1+\gamma _{t})U_{t}(n)\psi(n+at)+(1-\gamma _{t})U_{-t}(n)\psi(n-at)\right)+ia\kappa g \sigma ^{34}\gamma _4
\end{split}
\end{equation}

Still use
\begin{equation}
\begin{split}
& -\frac{U_{\mu}(n)\psi(n+a\mu)+U_{-\mu}(n)\psi(n-a\mu)-2\psi (n)}{2a} \to 0 \\
& \psi (n) \to \frac{U_{\mu}(n)\psi(n+a\mu)+U_{-\mu}(n)\psi(n-a\mu)}{2}
\end{split}
\end{equation}
The last term is
\begin{equation}
\begin{split}
&\kappa \gamma _4 ia g \sigma ^{43} \psi (n)\\
&-\approx \kappa \left( \frac{ -iag \sigma ^{34}}{2}\right) ( \gamma _4 U_{\mu}(n) \psi (n+\tau) + \gamma _4 U_{-\mu}(n) \psi (n-\tau)) \\
&\approx -\kappa \left( \frac{ -iag \sigma ^{34}}{2}\right)  ( \gamma _4 U_{\mu}(n) \psi (n+\tau) + \gamma _4 U_{-\mu}(n) \psi (n-\tau) \textcolor[rgb]{0.8,0.3,0}{- U_{\mu}(n)\psi (n+\tau) + U_{-\mu}(n)\psi (n-\tau)}) \\
&=-\kappa\left( \frac{ -iag \sigma ^{34}}{2}\right) \left((\gamma _4-1)U_{\mu}(n)\psi (n+\tau) +(\gamma _4+1) U_{-\mu}(n) \psi (n-\tau)\right)\\
&= -\kappa \left(\frac{-iag \sigma ^{34}}{2} (\gamma _4-1)U_{\mu}(n)\psi (n+\tau) + \frac{-iag \sigma ^{34}}{2} (\gamma _4+1) U_{-\mu}(n) \psi (n-\tau)\right)\\
&= -\kappa \left(\frac{iag \sigma ^{34}}{2} (1-\gamma _4)U_{\mu}(n)\psi (n+\tau) + \frac{-iag \sigma ^{34}}{2} (1+\gamma _4) U_{-\mu}(n) \psi (n-\tau)\right)\\
\end{split}
\end{equation}
Then $D_t +  \gamma _4 i \frac{g}{2} \sigma ^{43} \psi (n)$ is
\begin{equation}
\begin{split}
&= -\kappa \left(\left(1+\frac{iag \sigma ^{34}}{2}\right) (1-\gamma _4)U_{\mu}(n)\psi (n+\tau) + \left(1-\frac{iag \sigma ^{34}}{2}\right) (1+\gamma _4) U_{-\mu}(n) \psi (n-\tau)\right)\\
&\approx -\kappa \left(e^{\frac{iag \sigma ^{34}}{2}} (1-\gamma _4)U_{\mu}(n)\psi (n+\tau) + e^{-\frac{iag \sigma ^{34}}{2}} (1+\gamma _4) U_{-\mu}(n) \psi (n-\tau)\right)\\
\end{split}
\end{equation}

Finally
\begin{equation}
\begin{split}
&D=1+gz- 2gz\kappa - \kappa(1+g z)\sum _{\mu = x,y,z}\left((1+\gamma _{\mu})U_{\mu}(n)\psi(n+a\mu)+(1-\gamma _{\mu})U_{-\mu}(n)\psi(n-a\mu)\right)\\
&-\kappa \left(e^{\frac{iag \sigma ^{34}}{2}} (1-\gamma _4)U_{\mu}(n)\psi (n+\tau) + e^{-\frac{iag \sigma ^{34}}{2}} (1+\gamma _4) U_{-\mu}(n) \psi (n-\tau)\right)\\
\end{split}
\end{equation}

To implement it, note $(\sigma ^{34})^2=1$
\begin{equation}
\begin{split}
&e^{\pm i x \sigma ^{34}} =\cos (x \sigma ^{34}) \pm i \sin (x\sigma ^{34}) = \cos (x) \pm i \sigma ^{34}\sin (x)\\
\end{split}
\end{equation}

\textbf{\textcolor[rgb]{1,0,0}{We have sign problem here. There is an $\mathcal{O}(a)$ sign problem: $f(n)(1-\gamma _{\mu})U_{\mu}(n)\delta _{n,n+\mu}+f(n+\mu)(1+\gamma _{\mu})U_{\mu}^{-1}(n)\delta _{n+\mu,n}$ such that $f(n)\neq f(n+\mu)$. There is another chemical potential like sign problem, $i\gamma _4 \sigma ^{34}$ is NOT $\gamma _5$ -hermitian.}}

\subsubsection{\label{sec:ForceRigidAcc}Force Terms}

The force term of gauge field is obtained by
\begin{equation}
\begin{split}
&\sum _n f(n){\rm Retr}[1-\bar{U}_{\mu, \nu}(n)]=\sum _n\frac{f(n)+f(n+a\mu)+f(n+a\nu)+f(n+a\mu+a\nu)}{4}{\rm Retr}[1-U_{\mu\nu}(n)]\\
\end{split}
\end{equation}
and
\begin{equation}
\begin{split}
&\Sigma _{\mu}(n,\nu)=g(n)U_{\nu}(n)U_{\mu}(n+a\nu)U^{-1}_{\nu}(n+a\mu)+g(n-a\nu)U^{-1}_{\nu}(n-a\nu)U_{\mu}(n-a\nu)U_{\nu}(n+a\mu-a\nu)\\
\end{split}
\end{equation}

The force of Fermion field is
\begin{equation}
\begin{split}
&\frac{\partial D}{\partial \omega _{\mu}^a}=-i\kappa M_a,\\
&\frac{\partial}{\partial \omega_a} \left(\phi ^{\dagger}\left(\hat{D}\hat{D}^{\dagger}\right)^{-1}\phi\right)=-2\kappa{\rm Im}\left[\left(\phi _1 ^{\dagger} M \phi _2\right)\right]\\
&\phi _1=\left(\left(\hat{D}\hat{D}^{\dagger}\right)^{-1}\phi\right),\;\;
 \phi _2=D^{-1}\phi,\\
\end{split}
\end{equation}
And use
\begin{equation}
\begin{split}
&D(U_{\mu}(n))=f(n)(1-\gamma _{\mu})U_{\mu}(n)\delta _{n,n+\mu}+f(n)(1+\gamma _{\mu})U_{-\mu}(n)\delta _{n,n-\mu}\\
&=f(n)(1-\gamma _{\mu})U_{\mu}(n)\delta _{n,n+\mu}+f(n)(1+\gamma _{\mu})U_{\mu}^{-1}(n-\mu)\delta _{n,n-\mu}\\
&=f(n)(1-\gamma _{\mu})U_{\mu}(n)\delta _{n,n+\mu}+f(n+\mu)(1+\gamma _{\mu})U_{\mu}^{-1}(n)\delta _{n+\mu,n}\\
\end{split}
\end{equation}