\subsection{\label{sec:DataAnalyse}Data Analyse}

We write the data analyse code based on Ref.~\cite{autocorrelation} in Mathematica.

\subsubsection{What is autocorrelation\index{autocorrelation}}

The problem to address is that, the configurations generated are not statistically independent. So one need to take the relations between configurations into account.

To consider the relation between two sets, correlation functions are used, assuming two sets $a_{\alpha,\beta}$, assume $a_{\alpha,\beta}-\bar{a}_{\alpha,\beta}$ is a normal distribution.
\begin{equation}
\begin{split}
&\langle (a_{\alpha}-\bar{a}_{\alpha})(a_{\beta}-\bar{a}_{\beta})\rangle = \frac{1}{N^2}\sum _{i,j}\Gamma _{\alpha\beta}(j-i)
\end{split}
\end{equation}
and
\begin{equation}
\begin{split}
&C_{\alpha\beta}=\sum _{t=-\infty}^{\infty}\Gamma _{\alpha\beta}(t)
\end{split}
\end{equation}

\textbf{Note that, $C_{\alpha\alpha}(0)=N\langle \delta _{\alpha}^2\rangle$ is the standard error.}

For one single observable, one can define a correlation of a set of itself with delayed Markov time as
\begin{equation}
\begin{split}
&\tau _{\alpha}=\frac{1}{2\Gamma _{\alpha\alpha}(0)} \sum _{t=-\infty}^{\infty}\Gamma _{\alpha\alpha}(t)
\end{split}
\end{equation}
For a purely exponential behaviour, $\Gamma _{\alpha\beta}(t)\sim \exp (-|t|/\tau)$. Generally, we can estimate $2\tau _{\alpha}$ as an interval such that two configurations are effectively independent~\cite{autocorrelation}.

Here, $\Gamma _{\alpha\beta}(t)$ is \textbf{autocorrelation\index{autocorrelation}}.

\subsubsection{How to calculate autocorrelation, and how to use it to obtain the interval}

Considering, we have already obtained a set of measurements by using configurations generated with Markov chain $\{a_{\alpha}^{i,r}\}$, where $\alpha$ indicating different observables, $r=1\to R$ indicating different replicas (usually, different replicas are obtained by running multi-times starting from same parameters, or running parallelly starting from same parameters), and $i=1\to N_r$ is index of each value in the replica.

Assume we measured $a^{i,r}_{\alpha}$, and want to obtain $F=f(a_{\alpha})$.

In Ref.~\cite{autocorrelation}, a biased estimator is used such that
\begin{equation}
\begin{split}
&N=\sum _r^RN_r,\\
&\bar{a}^r_{\alpha}=\frac{1}{N_r}\sum _i a_{\alpha}^{i,r}\\
&\bar{\bar{a}}_{\alpha}=\frac{1}{N} \sum^R _r N_r \bar{a}^r_{\alpha}\\
&\bar{F}=\frac{1}{N} \sum^R _r N_rf(\bar{a}^r_{\alpha}),\\
&\bar{\bar{F}}=f(\bar{\bar{a}}_{\alpha})\\
\end{split}
\end{equation}
and
\begin{equation}
\begin{split}
&F_{mean}=\left\{\begin{array}{cc}\bar{\bar{F}}, & R=1 \\ \frac{R\bar{\bar{F}}-\bar{F}}{R-1} & R\geq 2\end{array}\right.
\end{split}
\end{equation}

The error is related to a correlation
\begin{equation}
\begin{split}
&\bar{\bar{\Gamma}} _{\alpha \beta}(t)=\frac{1}{N-Rt}\sum _{r=1}^R \sum _{i=1}^{N_r-t} \left(a^{i,r}_{\alpha}-\bar{\bar{a}}_{\alpha}\right)\left(a^{i+t,r}_{\beta}-\bar{\bar{a}}_{\beta}\right)
\end{split}
\end{equation}
at first we need to project it onto single variable, for this purpose, we need to calculate gradient as
\begin{equation}
\begin{split}
&h_{\alpha}=\sqrt{\frac{\Gamma _{\alpha\alpha}(0)}{N}},\\
&\bar{\bar{f}}_{\alpha}\approx \frac{1}{2h_{\alpha}}\left(f(\bar{\bar{a}}_1,\bar{\bar{a}}_2,\ldots, \bar{\bar{a}}_{\alpha}+h_{\alpha},\ldots)-f(\bar{\bar{a}}_1,\bar{\bar{a}}_2,\ldots, \bar{\bar{a}}_{\alpha}-h_{\alpha},\ldots)\right)
\end{split}
\end{equation}
and
\begin{equation}
\begin{split}
&\bar{\bar{\Gamma}} _F(t)=\sum _{\alpha\beta}\bar{\bar{f}}_{\alpha}\bar{\bar{f}}_{\beta}\bar{\bar{\Gamma}} _{\alpha\beta}(t)\\
\end{split}
\end{equation}
then the sum from $t=-\infty \to \infty$ is approximated as
\begin{equation}
\begin{split}
&\bar{\bar{C}} _F(W)=\bar{\bar{\Gamma}} _F(0)+2\sum _{t=1}^W\bar{\bar{\Gamma}} _F(t)\\
\end{split}
\end{equation}
By definition, if window $W$ is known, then
\begin{equation}
\begin{split}
&\bar{\bar{\tau}}_{int} (W) = \frac{\bar{\bar{C}} _F(W)}{2\bar{\bar{C}} _F(0)}
\end{split}
\end{equation}

To get $\tau$ one need to use a factor $S$ to fit the exponential, assuming
\begin{equation}
\begin{split}
&2\bar{\bar{\tau}} _{int} (W) = \sum _{t=-\infty}^{\infty}\exp\left(-\frac{S|t|}{\bar{\bar{\tau}} (W)}\right)
\end{split}
\end{equation}
with $S$ as a constant usually $S=1\to 2$.

Then one can let $\bar{\bar{\tau}} (W)= S\tau _{int}(W)$, and calculate
\begin{equation}
\begin{split}
&g(W)=\exp \left(-\frac{W}{\bar{\bar{\tau}} (W)}\right)-\frac{\bar{\bar{\tau}} (W)}{\sqrt{WN}}
\end{split}
\end{equation}
and fix $W$ as the first index such that $g(W)<0$ change sign.

Once $W$ is obtained, one can calculate $2\bar{\bar{\tau}}_{int}(W)$ which is the Markov time separation such that two configurations can be considered as independent. Also, the error estimate is
\begin{equation}
\begin{split}
&\delta _F^2 = \frac{\bar{\bar{C}}(W)}{N}
\end{split}
\end{equation}

