\subsection{\label{sec:RotatingFrame}Rotating Frame}

We follow Ref.~\cite{rotation}.

The matrix element can be written as
\begin{equation}
\begin{split}
&\mathcal{M}=\int \mathcal{D}(A_{\mu}\psi)\exp \left(i\int d^4x \mathcal{L}\right)
\end{split}
\end{equation}
with
\begin{equation}
\begin{split}
&\mathcal{L}=\bar{\psi}\left(i\slashed {D}-m\right)\psi-\frac{1}{4}\left(F^a_{\mu\nu}\right)^2\\
&D_{\mu}=\partial _{\mu}\textcolor[rgb]{1,0,0}{+}ig_{YM}\sum _a T_aA^a_{\mu}\\
\end{split}
\end{equation}
The first few steps are as usual, defining
\begin{equation}
\begin{split}
&A_{\mu}=g_{YM}\sum _a T_aA^a_{\mu},\;\;F_{\mu\nu}=\partial _{\mu}A_{\nu}-\partial _{\nu}A_{\mu}+i[A_{\mu},A_{\nu}]\\
\end{split}
\end{equation}
using ${\rm tr}[T_iT_j]=\frac{1}{2}\delta _{ij}$
\begin{equation}
\begin{split}
&F_{\mu\nu}=g_{YM}\sum _aT^a F^a_{\mu\nu}\\
&\frac{1}{4}\left(F^a_{\mu\nu}\right)^2=\frac{1}{2g_{YM}^2}{\rm tr}\left[F_{\mu\nu}^2\right]\\
\end{split}
\end{equation}
and
\begin{equation}
\begin{split}
&\mathcal{L}=\bar{\psi}\left(i\slashed {D}-m\right)\psi-\frac{1}{2g_{YM}^2}{\rm tr}\left[F_{\mu\nu}^2\right]\\
&D_{\mu}=\partial _{\mu}+iA_{\mu}\\
\end{split}
\end{equation}

For rotational frame, the metric and frame can be defined as
\begin{equation}
\begin{split}
&h_{\mu\nu}=\left(\begin{array}{cccc} 1 & 0 & 0 & 0 \\ 0 & -1 & 0 & 0 \\ 0 & 0 & -1 & 0 \\ 0 & 0 & 0 & -1 \end{array}\right)\\
&g_{\mu\nu}=\left(\begin{array}{cccc} 1-r^2\Omega^2 & \textcolor[rgb]{1,0,0}{+}y\Omega & \textcolor[rgb]{1,0,0}{-}x\Omega & 0 \\ y\Omega & -1 & 0 & 0 \\ -x\Omega & 0 & -1 & 0 \\ 0 & 0 & 0 & -1 \end{array}\right),\;\;\sqrt{-g_{\mu\nu}}=1\\
&e_0=(1,y\Omega,-x\Omega,0)\\
&e_1=(0,1,0,0)\\
&e_2=(0,0,1,0)\\
&e_3=(0,0,0,1)\\
\end{split}
\end{equation}

\subsubsection{\label{sec:RotatingGaugeAction}The rotating gauge action}

Considering the case of pure gauge, the action can be written as
\begin{equation}
\begin{split}
&\mathcal{L}_G=-\sqrt{\det (-g_{\alpha \beta})}\frac{1}{2g_{YM}^2}g^{\mu\nu}g^{\rho\sigma }{\rm tr}[F_{\mu\rho}F_{\nu\sigma}]\\
&=-\frac{1}{2g_{YM}^2}\left(\sum _{ijkl=0}^3 h_{ij}h_{kl}{\rm tr}[F_{ik}F_{jl}]+2\Omega^2 {\rm tr}\left[(x F_{01}+y F_{02})^2+r^2 F_{03}^2\right]\right.\\
&\left.-4\Omega\left(x {\rm tr}[F_{01}F_{12}]+y {\rm tr}[F_{02}F_{12}]+y {\rm tr}[F_{03}F_{13}]-x {\rm tr}[F_{03}F_{23}]\right)\right)
\end{split}
\end{equation}

\begin{itemize}
  \item Wick rotation of gauge action
\end{itemize}

The Wick rotation
\begin{equation}
\begin{split}
&t\to -i\tau,\;\;\Omega \to i\Omega ,\;\;A_{\mu}\to (iA_0,A_1,A_2,A_3),\;\;F_{0i}\to iF_{0i}\\
\end{split}
\end{equation}
and substitute $x=(t,x,y,z)\to x_E=(x,y,z,\tau)$. After Wick rotation, we are expecting
\begin{equation}
\begin{split}
&\exp(-S_G)=\exp (i\int d^4 x \mathcal{L}_G)
\end{split}
\end{equation}
the result is
\begin{equation}
\begin{split}
&-S_G=i\int d^4 x\mathcal{L}_G\\
&S_G=\int d^4 x_E \frac{1}{2g_{YM}^2}\left(\sum _{ij=1}^4 {\rm tr}[F_{ij}F_{ij}]+2\Omega^2 {\rm tr}\left[(x F_{14}+y F_{24})^2+r^2 F_{34}^2\right]\right.\\
&\left.+4\Omega\left(x {\rm tr}[F_{14}F_{12}]+y {\rm tr}[F_{24}F_{12}]+y {\rm tr}[F_{34}F_{13}]-x {\rm tr}[F_{34}F_{23}]\right)\right)
\end{split}
\end{equation}

Therefor $S_G$ is real. The $\sum _{ij=1}^4 {\rm tr}[F_{ij}F_{ij}]$ is the gauge action in rest frame.

\begin{itemize}
  \item Discretization of gauge action
\end{itemize}

The discretized version can be derived using compact gauge group
\begin{equation}
\begin{split}
&U_{\mu}(x)=\exp (i a A_{\mu}(x)),\;\;U_{-\mu}(x)=U_{\mu}^{-1}(x-\mu).
\end{split}
\end{equation}

As usual,
\begin{equation}
\begin{split}
&U_{\mu, \nu}(n)\equiv U_{\mu}(n)U_{\nu}(n+a\mu)U^{-1}_{\mu}(n+a\nu)U^{-1}_{\nu}(n)=\exp (ia^2 F_{\mu\nu}+\mathcal{O}(a^3)).\\
&{\rm Re}\left[U_{\mu \nu}(n)\right]=\mathbb{I}_{N_c\times N_c}-\frac{a^4}{2}F_{\mu\nu}^2+\mathcal{O}(a^6)\\
&\frac{1}{2g_{YM}^2}\sum _{\mu\neq \nu}{\rm tr}[F_{\mu\nu}^2]=\frac{1}{a^4g_{YM}^2}\sum _{\mu\neq \nu}{\rm Retr}\left[1-U_{\mu \nu}(n)\right]=\frac{2}{a^4g_{YM}^2}\sum _{\mu> \nu}{\rm Retr}\left[1-U_{\mu \nu}(n)\right]
\end{split}
\end{equation}

For those plaquette with coordinate as coefficients, we use the average of plaquette
\begin{equation}
\begin{split}
&\bar{U}_{\mu, \nu}(n)\equiv \frac{1}{4}\left(U_{\mu, \nu}(n)+U_{-\mu ,\nu}(n)+U_{\mu, -\nu}(n)+U_{-\mu, -\nu}(n)\right)\\
&{\rm Retr}[\bar{U}_{\mu, \nu}(n)]=N_c-\frac{a^4}{2}{\rm tr}[F_{\mu\nu}^2]+\mathcal{O}(a^6),\;\;\frac{1}{2g_{YM}^2}{\rm tr}[F_{\mu\nu}^2]=\frac{2}{a^4g_{YM}^2}\frac{1}{2}{\rm Retr}[1-\bar{U}_{\mu, \nu}(n)]\\
\end{split}
\end{equation}

The remaining are those has the form ${\rm tr}[F_{ab}F_{bc}]$, using
\begin{equation}
\begin{split}
&U_{\mu\nu}^{-1}(n)=U_{\nu}(n)U_{\mu}(n+a\nu)U^{-1}_{\nu}(n+a\mu)U^{-1}_{\mu}(n)=U_{\nu\mu}(n)=\exp (-ia^2 F_{\mu\nu})+\mathcal{O}(a^6)\\
&U_{a,b}(n)(U_{b,c}(n)-U_{c,b}(n))=\exp (-ia^2 (F_{ab}+F_{bc}))-\exp (-ia^2 (F_{ab}-F_{bc}))\\
&\frac{1}{2}{\rm Re}\left[U_{a,b}(n)(U_{c,b}(n)-U_{b,c}(n))\right]=a^4F_{ab}F_{bc}+\mathcal{O}(a^6)\\
\end{split}
\end{equation}
Note that $b$ is not summed.

We can have a more symmetric form to use
\begin{equation}
\begin{split}
&U_{c,b}(n)=U_{b,-c}(n)+\mathcal{O}(a^4)\\
\end{split}
\end{equation}
then it is a chair-type.
\begin{equation}
\begin{split}
&\frac{1}{2}{\rm Re}\left[U_{a,b}(n)(U_{b,-c}(n)-U_{b,c}(n))\right]=a^4F_{ab}F_{bc}+\mathcal{O}(a^6)\\
\end{split}
\end{equation}
Similarly, we can use the average of chairs, and define
\begin{equation}
\begin{split}
&V_{\mu\nu\sigma}=\frac{1}{8}\left((U_{\mu,\nu}-U_{-\mu,\nu})(U_{\nu,\sigma}-U_{\nu,-\sigma})+(U_{\mu,-\nu}-U_{-\mu,-\nu})(U_{-\nu,\sigma}-U_{-\nu,-\sigma})\right)\\
&{\rm Retr}[V_{\mu\nu\rho}]=-a^4{\rm tr}\left[F_{\mu\nu}F_{\nu\rho}\right]+\mathcal{O}(a^6),\;\;\frac{1}{2g_{YM}^2}{\rm tr}\left[F_{\mu\nu}F_{\nu\rho}\right]=-\frac{2}{a^4g_{YM}^2}\frac{1}{4}{\rm Retr}[V_{\mu\nu\rho}]\\
\end{split}
\end{equation}

\begin{itemize}
  \item Finial result of gauge action
\end{itemize}

The discretized and Wick rotated gauge action is
\begin{equation}
\begin{split}
&S_G=\frac{2}{a^4g_{YM}^2}\sum _{n}\left(\sum _{\mu >\nu}{\rm Retr}[1-U_{\mu\nu}(n)]+\Omega\left(x{\rm Retr}[V_{412}+V_{432}]-y{\rm Retr}[V_{421}+V_{431}]\right)\right.\\
&\left.+\Omega^2(x^2{\rm Retr}[1-\bar{U}_{14}(n)]+y^2{\rm Retr}[1-\bar{U}_{24}(n)]+r^2{\rm Retr}[1-\bar{U}_{34}(n)]+xy{\rm Retr}[V_{142}]\right)\\
&=\frac{\beta}{N_c}\sum _{n}\left(\sum _{\mu >\nu}{\rm Retr}[1-U_{\mu\nu}(n)]+\Omega\left(x{\rm Retr}[V_{412}+V_{432}]-y{\rm Retr}[V_{421}+V_{431}]\right)\right.\\
&\left.+\Omega^2(x^2{\rm Retr}[1-\bar{U}_{14}(n)]+y^2{\rm Retr}[1-\bar{U}_{24}(n)]+r^2{\rm Retr}[1-\bar{U}_{34}(n)]+xy{\rm Retr}[V_{142}]\right)\\
\end{split}
\end{equation}
with $\frac{\beta}{N_c} \equiv \frac{2}{a^4g_{YM}^2}$.

\subsubsection{\label{sec:RotatingFermionAction}Rotating Fermion action}

\begin{equation}
\begin{split}
&D_R=\left[i\gamma ^{\mu}\left((\partial _{\mu}+ieA_{\mu})-\frac{i}{4}\sigma ^{ij}w_{\mu ij}\right)-m\right]
\end{split}
\end{equation}

with
\begin{equation}
\begin{split}
&w_{\mu ij}=g_{\alpha \beta}e_i^{\alpha}(\partial _{\mu}e_j^{\beta}+\Gamma ^{\beta}_{\mu\nu}e_j^{\nu})\\
&\Gamma ^{\beta}_{\mu\nu}=\frac{1}{2}g^{\beta \alpha}\left(\frac{\partial g _{\alpha \mu}}{\partial x ^{\nu}}+\frac{\partial g _{\alpha \nu}}{\partial x ^{\mu}}-\frac{\partial g_{\mu\nu}}{\partial x^{\alpha}}\right)\\
&\sigma ^{ij}=\frac{i}{2}[\gamma ^i,\gamma ^j]\\
\end{split}
\end{equation}
so
\begin{equation}
\begin{split}
&\frac{i}{4}\sigma ^{ij}w_{\mu ij}=\left(\frac{i}{2}\Omega \sigma ^{12},0,0,0\right)\\
\end{split}
\end{equation}
and
\begin{equation}
\begin{split}
&D_R=\left[i\gamma ^x (\partial _x+ieA_x)+i\gamma ^y (\partial _y+ieA_y)+i\gamma ^z (\partial _z+ieA_z)+i\gamma ^t (\partial _t+ieA_t-\frac{i}{2}\Omega \sigma ^{12})-m\right]\\
\end{split}
\end{equation}

using $\gamma ^{\mu}=\gamma ^i e_i^{\mu}$, it is
\begin{equation}
\begin{split}
&D_R=\left[i(\gamma ^1+y\Omega \gamma^0) (\partial _x+ieA_x)+i(\gamma ^2-x\Omega \gamma^0) (\partial _y+ieA_y)+i\gamma ^3 (\partial _z+ieA_z)\right.\\
&\left.+i\gamma ^0 (\partial _t+ieA_t\textcolor[rgb]{1,0,0}{-}\frac{i}{2}\Omega \sigma ^{12})-m\right]\\
\end{split}
\end{equation}

\begin{itemize}
  \item The Wick rotation of Fermion action
\end{itemize}

The Wick rotation
\begin{equation}
\begin{split}
&t\to -i\tau,\;\;\gamma _i^{M}\to i\gamma _i^{E},\;\;\gamma _4=\gamma _0,\;\;\gamma _5=\gamma _1\gamma _2\gamma _3\gamma _4,\;\;A_t\to i A_{\tau},\;\;\partial _t\to i\partial _{\tau},\;\;\Omega \to i\Omega
\end{split}
\end{equation}
where the superscript of gamma matrix stands for Minkowski or Euclidian. So
\begin{equation}
\begin{split}
&D_R=-\left[(\gamma _1+y\Omega \gamma_4) (\partial _x+ieA_x)+(\gamma _2-x\Omega \gamma_4) (\partial _y+ieA_y)+\gamma _3 (\partial _z+ieA_z)\right.\\
&\left.+\gamma _4 (\partial _{\tau}+ieA_{\tau}-\frac{i}{2}\Omega \sigma ^{12})+m\right]\\
&=-\left[\gamma _1 (\partial _x+ieA_x)+\gamma _2(\partial _y+ieA_y)+\gamma _3 (\partial _z+ieA_z)+\gamma _4 (\partial _{\tau}+ieA_{\tau})+m\right.\\
&\left.+\gamma _4 \left(y\Omega(\partial _x+ieA_x)-x\Omega(\partial _y+ieA_y)\right)\textcolor[rgb]{1,0,0}{-}\frac{i}{2}\gamma _4\Omega \sigma ^{12}\right]\\
\end{split}
\end{equation}
And
\begin{equation}
\begin{split}
&-S_F=i\int d^4 x \sqrt{-g_{\alpha \beta}}\bar{\psi } D_R \psi\\
&S_F=\int d^4 x_E \bar{\psi } \left[\gamma _1 (\partial _x+ieA_x)+\gamma _2(\partial _y+ieA_y)+\gamma _3 (\partial _z+ieA_z)+\gamma _4 (\partial _{\tau}+ieA_{\tau})+m\right.\\
&\left.+\gamma _4 \left(y\Omega(\partial _x+ieA_x)-x\Omega(\partial _y+ieA_y)\right)-\frac{i}{2}\gamma _4\Omega \sigma ^{12}\right] \psi\\
\end{split}
\end{equation}

\begin{itemize}
  \item The discretization of Fermion action
\end{itemize}

The naive discretization yields
\begin{equation}
\begin{split}
&\partial _{\mu}\psi(n) =\frac{\psi (n+a\mu)-\psi (n-a\mu)}{2a}\\
\end{split}
\end{equation}
and
\begin{equation}
\begin{split}
&U_{\mu}(n)=\exp(iaA_{\mu})\approx 1+iaA_{\mu}(n),\;\;U^{-1}_{\mu}(n)\approx 1-iaA_{\mu}(n)\\
&iA_{\mu}(n)=\frac{2iaA_{\mu}(n)}{2a}\approx \frac{(U_{\mu}(n)-1)-(U^{-1}_{\mu}(n)-1)}{2a}\approx \frac{(U_{\mu}(n)-1)-(U_{-\mu}(n)-1)}{2a}\\
&iA_{\mu}(n)\psi(n)= \frac{(U_{\mu}(n)-1)\psi(n+a\mu)-(U_{-\mu}(n)-1)\psi(n-a\mu)}{2a}+\mathcal{O}(a)\\
\end{split}
\end{equation}
Therefor
\begin{equation}
\begin{split}
&(\partial _{\mu}+iA_{\mu})\psi(n)= \frac{U_{\mu}(n)\psi(n+a\mu)-U_{-\mu}(n)\psi(n-a\mu)}{2a}+\mathcal{O}(a)\\
\end{split}
\end{equation}

The Wilson term is
\begin{equation}
\begin{split}
&W\psi (n)= -\sum _{\mu}\frac{U_{\mu}(n)\psi(n+a\mu)+U_{-\mu}(n)\psi(n-a\mu)-2\psi(n)}{2a}\\
\end{split}
\end{equation}
considering the rotation, we add a modified Wilson term as
\begin{equation}
\begin{split}
&W_R\psi (n)= -\sum _{\mu}\frac{U_{\mu}(n)\psi(n+a\mu)+U_{-\mu}(n)\psi(n-a\mu)-2\psi(n)}{2a}\\
&-y\Omega \frac{U_{x}(n)\psi(n+ax)+U_{-x}(n)\psi(n-ax)-2\psi(n)}{2a}\\
&+x\Omega \frac{U_{y}(n)\psi(n+ay)+U_{-y}(n)\psi(n-ay)-2\psi(n)}{2a}
\end{split}
\end{equation}
Similar as the Wilson term, the last two terms also decouples when approaching the continuum limit. And the Wilson-Dirac operator becomes
\begin{equation}
\begin{split}
&S_F=\sum _{n,m}\bar{\psi }(n) D_W(n|m) \psi(m)\\
&D_W(n|m)=\left(m+\frac{4}{a}+\frac{y\Omega}{a}-\frac{x\Omega}{a}\right)\delta _{n,m}-\sum _{\mu}\frac{(1-\gamma _{\mu})U_{\mu}(n)\delta _{n+a\mu,m}+(1+\gamma _{\mu})U_{-\mu}(n)\delta _{n-a\mu,m}}{2a}\\
&-y\Omega \frac{(1-\gamma _4)U_{x}(n)\delta _{n+ax,m}+(1+\gamma _4)U_{-x}(n)\delta _{n-ax,m}}{2a}\\
&+x\Omega \frac{(1-\gamma _4)U_{y}(n)\delta _{n+ay,m}+(1+\gamma _4)U_{-y}(n)\delta _{n-ay,m}}{2a} -\frac{i}{2}\gamma _4\Omega \sigma ^{12}\delta _{n,m}\\
\end{split}
\end{equation}
Note that,
\begin{equation}
\begin{split}
&\left(U(1)\delta _{n+1,m,n=1,m=2}\right)^{\dagger}=U^{\dagger}(1)\delta _{n,m+1,n=2,m=1}=U^{\dagger}(m)\delta _{n,m+1}\;or\;U^{\dagger}(n-1)\delta _{n-1,m}\\
&\left(U_{\mu}(n)\delta _{n+a\mu,m}\right)^{\dagger}=U_{\mu}^{-1}(n-a\mu)\delta _{n,m+a\mu}=U_{-\mu}(n)\delta _{n-a\mu,m}\\
\end{split}
\end{equation}
Let's check whether the periodic condition for gauge field or infinite lattice volume is necessary
\begin{equation}
\begin{split}
&\sum _{n=1,2,3,m=1,2,3}\left(U_{\mu}(n)\delta _{n+1,m}+U_{-\mu}(n)\delta _{n-1,m}\right)^{\dagger}=\left(U_{\mu}(1)\right)^{\dagger}+\left(U_{\mu}(2)\right)^{\dagger}+\left(U_{-\mu}(2)\right)^{\dagger}+\left(U_{-\mu}(3)\right)^{\dagger}\\
&=U_{\mu}^{-1}(2-1)+U^{-1}_{\mu}(3-1)+\left(U^{-1}_{\mu}(1)\right)^{\dagger}+\left(U^{-1}_{\mu}(2)\right)^{\dagger}\\
&=U_{-\mu}(2)+U_{-\mu}(3)+U_{\mu}(1)+U_{\mu}(2)\\
&=\sum _{n=1,2,3,m=1,2,3}\left(U_{\mu}(n)\delta _{n+1,m}+U_{-\mu}(n)\delta _{n-1,m}\right)
\end{split}
\end{equation}
So we can conclude \textcolor[rgb]{0,0,1}{The $\gamma _5$-hermiticity is kept with open(Dirichlet) boundary condition and finite volume.}

\textcolor[rgb]{0,0,1}{Both the naive discretization and the Wilson term satisfy the $\gamma _5$-hermiticity (separately).}
\begin{equation}
\begin{split}
&\gamma _5 \gamma _{\nu}\gamma _5=-\gamma _{\nu},\;\;\gamma _{\mu}^{\dagger}=\gamma _{\nu},\;\;\gamma _5^2=1\\
&\sum _{n,m}\left(\gamma _{\nu}U_{\mu}(n)\delta _{n+a\mu,m}-\gamma _{\nu}U_{-\mu}(n)\delta _{n,m+a\mu}\right)^{\dagger}=\sum _{n,m}\left(\gamma _5\gamma _{\nu}\gamma _5U_{\mu}(n)\delta _{n+a\mu,m}-\gamma _5\gamma _{\nu}\gamma _5U_{-\mu}(n)\delta _{n-a\mu,m}\right)\\
&\sum _{n,m}\left(U_{\mu}(n)\delta _{n+a\mu,m}+U_{-\mu}(n)\delta _{n,m+a\mu}\right)^{\dagger}=\sum _{n,m}\left(\gamma _5^2U_{\mu}(n)\delta _{n+a\mu,m}+\gamma _5^2U_{-\mu}(n)\delta _{n-a\mu,m}\right)\\
\end{split}
\end{equation}
Apart from that
\begin{equation}
\begin{split}
&\left(\frac{i}{2}\gamma _4\Omega \sigma ^{12}\delta _{n,m}\right)^{\dagger}=\gamma _5 \frac{i}{2}\gamma _4\Omega \sigma ^{12}\delta _{n,m} \gamma _5\\
\end{split}
\end{equation}
Therefor, the new Wilson-Dirac operator is also $\gamma _5$-hermiticity.

\begin{itemize}
  \item The doubler problem
\end{itemize}

Note that the naive action and Wilson term both satisfy the $\gamma _5$-hermiticity, the traditional Wilson term will also lead to a $\gamma _5$-hermiticity fermion action, with
\begin{equation}
\begin{split}
&D_W(n|m)=\left(m+\frac{4}{a}\right)\delta _{n,m}-\sum _{\mu}\frac{(1-\gamma _{\mu})U_{\mu}(n)\delta _{n+a\mu,m}+(1+\gamma _{\mu})U_{-\mu}(n)\delta _{n-a\mu,m}}{2a}\\
&+y\Omega \frac{\gamma _4U_{x}(n)\delta _{n+ax,m}-\gamma _4U_{-x}(n)\delta _{n-ax,m}}{2a}-x\Omega \frac{\gamma _4U_{y}(n)\delta _{n+ay,m}-\gamma _4U_{-y}(n)\delta _{n-ay,m}}{2a} -\frac{i}{2}\gamma _4\Omega \sigma ^{12}\delta _{n,m}\\
\end{split}
\end{equation}
This action also does not suffer from the doubler problem.

\begin{itemize}
  \item The final action of fermions
\end{itemize}

As usual, we define the hopping parameter as $\kappa = \frac{1}{2am+8}$, then rescale the fermion field, the action is
\begin{equation}
\begin{split}
&S_F=\sum _{n,m}\bar{\psi }(n) D_W(n|m) \psi(m)\\
&D_W(n|m)=\left(1+2\kappa(y-x)\Omega\right)\delta _{n,m}-\kappa\sum _{\mu}\left((1-\gamma _{\mu})U_{\mu}(n)\delta _{n+a\mu,m}+(1+\gamma _{\mu})U_{-\mu}(n)\delta _{n-a\mu,m}\right)\\
&-\kappa y\Omega \left((1-\gamma _4)U_{x}(n)\delta _{n+ax,m}+(1+\gamma _4)U_{-x}(n)\delta _{n-ax,m}\right)\\
&+\kappa x\Omega \left((1-\gamma _4)U_{y}(n)\delta _{n+ay,m}+(1+\gamma _4)U_{-y}(n)\delta _{n-ay,m}\right) -\kappa i\gamma _4a\Omega \sigma ^{12}\delta _{n,m}\\
\end{split}
\end{equation}

\subsubsection{\label{sec:ExponentialChemicalPotential}The exponential chemical potential}

On the other hand, the $i\kappa \gamma _4 \hat{\Omega} \sigma ^{12}$ term can also be modified. The $\sigma ^{12}$ term can be considered as a chemical potential ($\bar{\psi}\gamma _0 \psi$ and then do the Wick rotation $\gamma _0\to \gamma _4$. The sign is after Wick rotation and relative to the mass term)
\begin{equation}
\begin{split}
&\mu  \bar{\psi} \gamma _4 \psi,\;\;\mu = \textcolor[rgb]{1,0,0}{-}\frac{i\Omega}{2}\sigma ^{12}
\end{split}
\end{equation}
and discritized as
\begin{equation}
\begin{split}
&D_{\tau}+\mu  \bar{\psi} \gamma _4 \psi \to -\kappa \left(e^{\mu a}(1-\gamma _4)U_{\tau}(n)\delta _{n,n+t} + e^{-\mu a}(1+\gamma _4) U_{-\tau}(n)\delta _{n-t,n}\right)\\
&=-\kappa \left(e^{\textcolor[rgb]{1,0,0}{-}\frac{ia\Omega \sigma ^{12}}{2}}(1-\gamma _4)U_{\tau}(n)\delta _{n,n+t} + e^{\textcolor[rgb]{1,0,0}{+}\frac{ia\Omega \sigma ^{12}}{2}}(1+\gamma _4) U_{-\tau}(n)\delta _{n-t,n}\right)
\end{split}
\end{equation}
\textcolor[rgb]{1,0,0}{It looks not satisfy the $\gamma _5$ -hermiticity.} However, using $\left(\sigma ^{12}\right)^2=1$, it is in fact
\begin{equation}
\begin{split}
&D_{\tau}+\mu  \bar{\psi} \gamma _4 \psi \to -\kappa \left[\left(\cos(\frac{a\Omega}{2})-i\sin(\frac{a\Omega}{2})\sigma ^{12}\right)(1-\gamma _4)U_{\tau}(n)\delta _{n,n+t} \right.\\
&\left.+\left(\cos(\frac{a\Omega}{2})+i\sin(\frac{a\Omega}{2})\sigma ^{12}\right)(1+\gamma _4) U_{-\tau}(n)\delta _{n-t,n}\right]
\end{split}
\end{equation}
The $1$ in $\textcolor[rgb]{0,0,1}{1}\pm \frac{ia\Omega \sigma ^{12}}{2}$ is the usual $D_{\tau}$. So the additional term is in fact
\textcolor[rgb]{0,0,1}{
\begin{equation}
\begin{split}
&-\kappa \left[\left(\cos(\frac{a\Omega}{2})-1-i\sin(\frac{a\Omega}{2})\sigma ^{12}\right)(1-\gamma _4)U_{\tau}(n)\delta _{n,n+t} \right.\\
&\left.+\left(\cos(\frac{a\Omega}{2})-1+i\sin(\frac{a\Omega}{2})\sigma ^{12}\right)(1+\gamma _4) U_{-\tau}(n)\delta _{n-t,n}\right]
\end{split}
\end{equation}
}
In the case of $a \ll 1$, it is approximately
\begin{equation}
\begin{split}
&-\kappa \left[\left(-i\frac{a\Omega}{2}\sigma ^{12}\right)(1-\gamma _4)U_{\tau}(n)\delta _{n,n+t} \right.\\
&\left.+\left(i\frac{a\Omega}{2}\sigma ^{12}\right)(1+\gamma _4) U_{-\tau}(n)\delta _{n-t,n}\right]
\end{split}
\end{equation}
The $U_{\tau}$ is in fact added to keep gauge symmetry, originally it is
\begin{equation}
\begin{split}
&-\kappa \frac{ia\Omega \sigma ^{12}}{2}\left((\gamma _4-1)\delta _{n,n+t} +(\gamma _4+1) \delta _{n-t,n}\right)\\
&\approx -\kappa \frac{ia\Omega \sigma ^{12}}{2}\left((\gamma _4-1)\delta _{n,n} +(\gamma _4+1) \delta _{n,n}\right)\\
&=-\kappa \gamma _4 ia\Omega \sigma ^{12}
\end{split}
\end{equation}
which go back to the $\sigma ^{12}$ term.

We still check the $\gamma _5$-hermiticity, using
\begin{equation}
\begin{split}
&\sum _{n=1,2,3,m=1,2,3}\left(U_{\mu}(n)\delta _{n+1,m}+U_{-\mu}(n)\delta _{n-1,m}\right)^{\dagger}=\sum _{n=1,2,3,m=1,2,3}\left(U_{\mu}(n)\delta _{n+1,m}+U_{-\mu}(n)\delta _{n-1,m}\right)\\
&\gamma _5 \gamma _4 \gamma _5 = -\gamma _4^{\dagger}\\
&\gamma _5 \frac{ia\Omega \sigma ^{12}}{2} \gamma _5 = -\left(\frac{ia\Omega \sigma ^{12}}{2}\right)^{\dagger}\\
\end{split}
\end{equation}
It is $\gamma _5$-hermite.

\subsubsection{\label{sec:TheFinalActionOfRotation}The final action of rotation}

Defining $\frac{\beta}{N_c} \equiv \frac{2}{a^4g_{YM}^2}$, $\kappa \equiv \frac{1}{2am+8}$, $\hat{\mu}\equiv \frac{\mu}{a}$, $\hat{\Omega}\equiv a\Omega$, (here $\mu=x,y,z,t$ is the coordinate) we have
\begin{equation}
\begin{split}
&Z=\exp (-S_G-S_F)\\
\end{split}
\end{equation}
with
\textcolor[rgb]{0,0,0.8}{
\begin{equation}
\begin{split}
&S_G=\frac{\beta}{N_c}\sum _{n}\left(\sum _{\mu >\nu}{\rm Retr}[1-U_{\mu\nu}(n)]+\hat{\Omega}\left(\hat{x}{\rm Retr}[V_{412}+V_{432}]-\hat{y}{\rm Retr}[V_{421}+V_{431}]\right)\right.\\
&\left.+\hat{\Omega}^2(\hat{x}^2{\rm Retr}[1-\bar{U}_{14}(n)]+\hat{y}^2{\rm Retr}[1-\bar{U}_{24}(n)]+(\hat{x}^2+\hat{y}^2){\rm Retr}[1-\bar{U}_{34}(n)]+\hat{x}\hat{y}{\rm Retr}[V_{142}]\right)\\
&U_{\mu, \nu}(n)\equiv U_{\mu}(n)U_{\nu}(n+a\hat{\mu})U^{-1}_{\mu}(n+a\hat{\nu})U^{-1}_{\nu}(n)\\
&\bar{U}_{\mu, \nu}(n)\equiv \frac{1}{4}\left(U_{\mu, \nu}(n)+U_{-\mu ,\nu}(n)+U_{\mu, -\nu}(n)+U_{-\mu, -\nu}(n)\right)\\
&V_{\mu\nu\sigma}(n)=\frac{1}{8}\left((U_{\mu,\nu}(n)-U_{-\mu,\nu}(n))(U_{\nu,\sigma}(n)-U_{\nu,-\sigma}(n))\right.\\
&\left.+(U_{\mu,-\nu}(n)-U_{-\mu,-\nu}(n))(U_{-\nu,\sigma}(n)-U_{-\nu,-\sigma}(n))\right)\\
\end{split}
\end{equation}
}
and
\textcolor[rgb]{0,0,0.8}{
\begin{equation}
\begin{split}
&S_F=\sum _{n,m}\bar{\psi }(n) D(n|m) \psi(m)\\
&D(n|m)=\left(1+2\kappa(\hat{y}-\hat{x})\hat{\Omega}-i\kappa \gamma _4 \hat{\Omega} \sigma ^{12}\right)\delta _{n,m}\\
&-\kappa\sum _{\mu}\left((1-\gamma _{\mu})U_{\mu}(n)\delta _{n+a\hat{\mu},m}+(1+\gamma _{\mu})U_{-\mu}(n)\delta _{n-a\hat{\mu},m}\right)\\
&-\kappa \hat{y}\hat{\Omega} \left((1-\gamma _4)U_{x}(n)\delta _{n+a\hat{x},m}+(1+\gamma _4)U_{-x}(n)\delta _{n-a\hat{x},m}\right)\\
&+\kappa \hat{x}\hat{\Omega} \left((1-\gamma _4)U_{y}(n)\delta _{n+a\hat{y},m}+(1+\gamma _4)U_{-y}(n)\delta _{n-a\hat{y},m}\right)\\
\end{split}
\end{equation}
}
such that $\gamma _5 D \gamma _5=D^{\dagger}$.

or (As in Ref.~\cite{rotation}, the naive discretization is used.)
\textcolor[rgb]{0,0,0.8}{
\begin{equation}
\begin{split}
&S_F=\sum _{n,m}\bar{\psi }(n) D(n|m) \psi(m)\\
&D(n|m)=\left(1-i\kappa \gamma _4 \hat{\Omega} \sigma ^{12}\right)\delta _{n,m}\\
&-\kappa\sum _{\mu}\left((1-\gamma _{\mu})U_{\mu}(n)\delta _{n+a\hat{\mu},m}+(1+\gamma _{\mu})U_{-\mu}(n)\delta _{n-a\hat{\mu},m}\right)\\
&-\kappa \hat{y}\hat{\Omega} \left((-\gamma _4)U_{x}(n)\delta _{n+a\hat{x},m}+(+\gamma _4)U_{-x}(n)\delta _{n-a\hat{x},m}\right)\\
&+\kappa \hat{x}\hat{\Omega} \left((-\gamma _4)U_{y}(n)\delta _{n+a\hat{y},m}+(+\gamma _4)U_{-y}(n)\delta _{n-a\hat{y},m}\right)\\
\end{split}
\end{equation}
}

If using the exponential spin coupling term it is
\textcolor[rgb]{0,0,0.8}{
\begin{equation}
\begin{split}
&S_F=\sum _{n,m}\bar{\psi }(n) D(n|m) \psi(m)\\
&D(n|m)=\delta _{n,m}-\kappa\sum _{\mu}\left((1-\gamma _{\mu})U_{\mu}(n)\delta _{n+a\hat{\mu},m}+(1+\gamma _{\mu})U_{-\mu}(n)\delta _{n-a\hat{\mu},m}\right)\\
&-\kappa \hat{y}\hat{\Omega} \left((-\gamma _4)U_{x}(n)\delta _{n+a\hat{x},m}+(+\gamma _4)U_{-x}(n)\delta _{n-a\hat{x},m}\right)\\
&+\kappa \hat{x}\hat{\Omega} \left((-\gamma _4)U_{y}(n)\delta _{n+a\hat{y},m}+(+\gamma _4)U_{-y}(n)\delta _{n-a\hat{y},m}\right)\\
&-\kappa \frac{ia\Omega \sigma ^{12}}{2}\left((\gamma _4-1)U_{\tau}(n)\delta _{n,n+t} +(\gamma _4+1) U_{-\tau}(n)\delta _{n-t,n}\right)\\
\end{split}
\end{equation}
}


\subsubsection{\label{sec:ForceFromGaugeAction}The force from gauge action}

\begin{equation}
\begin{split}
&U_{\mu, \nu}(n)= U_{\mu}(n)U_{\nu}(n+a\mu)U^{-1}_{\mu}(n+a\nu)U^{-1}_{\nu}(n).\;\;U^{\dagger}_{\mu, \nu}(n)=U^{-1}_{\mu, \nu}(n)\\
&U^{-1}_{\mu, \nu}(n)=U_{\nu}(n)U_{\mu}(n+a\nu)U^{-1}_{\nu}(n+a\mu)U^{-1}_{\mu}(n)=U_{\nu, \mu}(n).\\
&{\rm tr}[U_{\nu, \mu}(n)]= {\rm tr}\left[U_{\nu}(n)U_{\mu}(n+a\nu)U^{-1}_{\nu}(n+a\mu)U^{-1}_{\mu}(n)\right].\\
&{\rm tr}[U_{-\mu, \nu}(n)]= {\rm tr}\left[U_{\nu}(n-a\mu)U^{-1}_{-\mu}(n+a\nu+a\mu-a\mu)U^{-1}_{\nu}(n)U_{-\mu}(n)\right]\\
&={\rm tr}\left[U_{\nu}(n-a\mu)U_{\mu}(n+a\nu-a\mu)U^{-1}_{\nu}(n)U^{-1}_{\mu}(n-a\mu)\right]={\rm tr}[U_{\nu, \mu}(n-a\mu)]={\rm tr}[U^{\dagger}_{\mu, \nu}(n-a\mu)]\\
&{\rm tr}[U_{\mu, -\nu}(n)]={\rm tr}[U^{\dagger}_{-\nu, \mu}(n)]=({\rm tr}[U_{-\nu, \mu}(n)])^*=({\rm tr}[U_{\mu, \nu}(n-a\nu)])^*={\rm tr}[U^{\dagger}_{\mu, \nu}(n-a\nu)]\\
&{\rm tr}[U_{-\mu, -\nu}(n)]={\rm tr}[U_{\mu, \nu}(n-a\mu-a\nu)]
\end{split}
\end{equation}
so
\begin{equation}
\begin{split}
&{\rm Retr}[\bar{U}_{\mu, \nu}(n)]=\frac{1}{4}{\rm Retr}[U_{\mu\nu}(n)+U_{\mu\nu}(n-a\mu)+U_{\mu\nu}(n-a\nu)+U_{\mu\nu}(n-a\mu-a\nu)]\\
\end{split}
\end{equation}
and
\begin{equation}
\begin{split}
&\sum _n f(n){\rm Retr}[1-\bar{U}_{\mu, \nu}(n)]=\sum _n\frac{f(n)+f(n+a\mu)+f(n+a\nu)+f(n+a\mu+a\nu)}{4}{\rm Retr}[1-U_{\mu\nu}(n)]\\
\end{split}
\end{equation}
so (\textbf{Note, this is for infinite lattice size, the boundary condition should be considered})
\begin{equation}
\begin{split}
&\sum _n \hat{\Omega}^2\hat{x}^2{\rm Retr}[1-\bar{U}_{1, 4}(n)]=\sum _n\hat{\Omega}^2\frac{2\hat{x}^2+2\hat{x}+1}{2}{\rm Retr}[1-U_{1,4}(n)]\\
&\sum _n \hat{\Omega}^2\hat{y}^2{\rm Retr}[1-\bar{U}_{2, 4}(n)]=\sum _n\hat{\Omega}^2\frac{2\hat{y}^2+2\hat{y}+1}{2}{\rm Retr}[1-U_{2,4}(n)]\\
&\sum _n \hat{\Omega}^2(\hat{x}^2+\hat{y}^2){\rm Retr}[1-\bar{U}_{3, 4}(n)]=\sum _n\hat{\Omega}^2(\hat{x}^2+\hat{y}^2){\rm Retr}[1-U_{3,4}(n)]\\
\end{split}
\end{equation}
Using
\begin{equation}
\begin{split}
&{\rm Retr}[U_{\mu,\nu}(n-a\nu)]={\rm Retr}[U_{\mu}(n-a\nu)U_{\nu}(n+a\mu-a\nu)U^{-1}_{\mu}(n)U^{-1}_{\nu}(n-a\nu)]\\
&={\rm Retr}[U_{\nu}(n-a\nu)U_{\mu}(n)U^{-1}_{\nu}(n+a\mu-a\nu)U^{-1}_{\mu}(n-a\nu)]\\
&={\rm Retr}[U_{\mu}(n)U^{-1}_{\nu}(n+a\mu-a\nu)U^{-1}_{\mu}(n-a\nu)U_{\nu}(n-a\nu)]\\
\end{split}
\end{equation}

\begin{equation}
\begin{split}
&\sum _ng(n){\rm Retr}[1-U_{\mu,\nu}(n)]=N\times N_c-\sum _n {\rm Retr}\left[U_{\mu}(n)\Sigma ^{\dagger}_{\mu}(n)\right]\\
&\Sigma _{\mu,i}(n,\nu)=g_i(n)U_{\nu}(n)U_{\mu}(n+a\nu)U^{-1}_{\nu}(n+a\mu)+g_i(n-a\nu)U^{-1}_{\nu}(n-a\nu)U_{\mu}(n-a\nu)U_{\nu}(n+a\mu-a\nu)\\
\end{split}
\end{equation}
The product is just same as the definition of staples. However, there are two differences, (i) there is a coefficient function for each term of the sum. (ii) there is no sum over $\nu$.

The following is usual, with the new definition of the staple, one have
\begin{equation}
\begin{split}
&F_{\mu}(n)=-\frac{\beta}{2N_c}\left\{U_{\mu}(n)\Sigma _{\mu,i}^{\dagger}(n,\nu)\right\}_{TA}
\end{split}
\end{equation}
with $i=1,2,3$ and (\textbf{Note, this is for infinite lattice size, the boundary condition should be considered})
\begin{equation}
\begin{split}
&g_1(n)=\frac{\Omega^2(2x^2+2x+1)}{2},\;\;g_2(n)=\frac{\Omega ^2 (2y^2+2y+1)}{2},\;\;g_3(n)=\Omega ^2 (x^2+y^2),\\
&F_{\mu=1,2,3}(n)=-\frac{\beta}{2N_c}\left\{U_{\mu}(n)\Sigma _{\mu,\mu}^{\dagger}(n,4)\right\}_{TA}\\
&F_{4}(n)=-\frac{\beta}{2N_c}\left\{U_{4}(n)\sum _{i=1,2,3}\Sigma _{4,i}^{\dagger}(n,i)\right\}_{TA}\\
\end{split}
\end{equation}

Now we consider the force of $V$
\begin{equation}
\begin{split}
&V_{\mu\nu\sigma}=\frac{1}{8}\left(U_{\mu,\nu}U_{\nu,\sigma}+U_{-\mu,\nu}U_{\nu,-\sigma}+U_{\mu,-\nu}U_{-\nu,\sigma}+U_{-\mu,-\nu}U_{-\nu,-\sigma}\right.\\
&\left.-U_{\mu,\nu}U_{\nu,-\sigma}-U_{-\mu,\nu}U_{\nu,\sigma}-U_{\mu,-\nu}U_{-\nu,-\sigma}-U_{-\mu,-\nu}U_{-\nu,\sigma}\right)\\
\end{split}
\end{equation}
Using
\begin{equation}
\begin{split}
&{\rm Retr}[U_{\mu,\nu}U_{\nu,\sigma}]={\rm Retr}[U_{\sigma,\nu}U_{\nu,\mu}],\;\;{\rm Retr}[U_{\mu,\nu}U_{\nu,-\sigma}]={\rm Retr}[U_{-\sigma,\nu}U_{\nu,\mu}]
\end{split}
\end{equation}
one have
\begin{equation}
\begin{split}
&{\rm Retr}[V_{\mu\nu\rho}]={\rm Retr}[V_{\rho\nu\mu}]
\end{split}
\end{equation}
So we only need to calculate $\frac{\partial }{\partial \omega _{\mu}}V_{\mu\nu\rho}$ and $\frac{\partial }{\partial \omega _{\nu}}V_{\mu\nu\rho}$.

One can find
\begin{equation}
\begin{split}
&\sum _n{\rm Retr}[g(n)V_{\mu\nu\rho}(n)]\to S[U_{\mu}(n)]={\rm Retr}[U_{\mu}(n)M(n)]
\end{split}
\end{equation}
with
\begin{equation}
\begin{split}
&M(n)=\frac{1}{8}\left((g(n)+g(n+a\nu))U_{\nu}(n+a\mu)U_{\mu}^{-1}(n+a\nu)U_{\rho}(n+a\nu)U_{\nu}^{-1}(n+a\rho)U_{\rho}^{-1}(n)\right.\\
&\left.+(g(n)+g(n-a\nu))U^{-1}_{\nu}(n+a\mu-a\nu)U_{\mu}^{-1}(n-a\nu)U_{\rho}(n-a\nu)U_{\nu}(n-a\nu+a\rho)U_{\rho}^{-1}(n)\right.\\
&\left.+(g(n+a\mu-a\nu)+g(n+a\mu))U^{-1}_{\rho}(n+a\mu-a\rho)U_{\nu}^{-1}(n+a\mu-a\rho-a\nu)\right.\\
&\left.\times U_{\rho}(n+a\mu-a\rho-a\nu)U^{-1}_{\mu}(n-a\nu)U_{\nu}(n-a\nu)\right.\\
&\left.+(g(n+a\mu+a\nu)+g(n+a\mu))U^{-1}_{\rho}(n+a\mu-a\rho)U_{\nu}(n+a\mu-a\rho)\right.\\
&\left.\times U_{\rho}(n+a\mu-a\rho+a\nu)U_{\mu}^{-1}(n+a\nu)U_{\nu}^{-1}(n)\right.\\
&\left.-(g(n+a\mu)+g(n+a\mu+a\nu))U_{\rho}(n+a\mu)U_{\nu}(n+a\mu+a\rho)U_{\rho}^{-1}(n+a\mu+a\nu)U_{\mu}^{-1}(n+a\nu)U_{\nu}^{-1}(n)\right.\\
&\left.-(g(n+a\mu)+g(n+a\mu-a\nu))U_{\rho}(n+a\mu)U_{\nu}^{-1}(n+a\mu+a\rho-a\nu)\right.\\
&\left.\times U_{\rho}^{-1}(n+a\mu-a\nu)U_{\mu}^{-1}(n-a\nu)U_{\nu}(n-a\nu)\right.\\
&\left.-(g(n)+g(n+a\nu))U_{\nu}(n+a\mu)U_{\mu}^{-1}(n+a\nu)U_{\rho}^{-1}(n+a\nu-a\rho)U_{\nu}^{-1}(n-a\rho)U_{\rho}(n-a\rho)\right.\\
&\left.-(g(n)+g(n-a\nu))U_{\nu}^{-1}(n+a\mu-a\nu)U_{\mu}^{-1}(n-a\nu)U_{\rho}^{-1}(n-a\nu-a\rho)U_{\nu}(n-a\nu-a\rho)U_{\rho}(n-a\rho)\right)
\end{split}
\end{equation}
It can be simplified as
\begin{equation}
\begin{split}
&M(n)=\frac{1}{8}\left((g(n)+g(n+a\nu))U_{\nu}(n+a\mu)U_{\mu}^{-1}(n+a\nu)S_1\right.\\
&\left.+(g(n)+g(n-a\nu))U^{-1}_{\nu}(n+a\mu-a\nu)U_{\mu}^{-1}(n-a\nu)S_2\right.\\
&\left.+(g(n+a\mu)+g(n+a\mu+a\nu))S_3U_{\mu}^{-1}(n+a\nu)U_{\nu}^{-1}(n)\right.\\
&\left.+(g(n+a\mu)+g(n+a\mu-a\nu))S_4U^{-1}_{\mu}(n-a\nu)U_{\nu}(n-a\nu)\right)\\
&S_1=U_{\rho}(n+a\nu)U_{\nu}^{-1}(n+a\rho)U_{\rho}^{-1}(n)-U_{\rho}^{-1}(n+a\nu-a\rho)U_{\nu}^{-1}(n-a\rho)U_{\rho}(n-a\rho)\\
&S_2=U_{\rho}(n-a\nu)U_{\nu}(n-a\nu+a\rho)U_{\rho}^{-1}(n)-U_{\rho}^{-1}(n-a\nu-a\rho)U_{\nu}(n-a\nu-a\rho)U_{\rho}(n-a\rho)\\
&S_3=U^{-1}_{\rho}(n+a\mu-a\rho)U_{\nu}(n+a\mu-a\rho)U_{\rho}(n+a\mu-a\rho+a\nu)\\
&-U_{\rho}(n+a\mu)U_{\nu}(n+a\mu+a\rho)U_{\rho}^{-1}(n+a\mu+a\nu)\\
&S_4=U^{-1}_{\rho}(n+a\mu-a\rho)U_{\nu}^{-1}(n+a\mu-a\rho-a\nu)U_{\rho}(n+a\mu-a\rho-a\nu)\\
&-U_{\rho}(n+a\mu)U_{\nu}^{-1}(n+a\mu+a\rho-a\nu)U_{\rho}^{-1}(n+a\mu-a\nu)
\end{split}
\end{equation}
Similarly, one also have
\begin{equation}
\begin{split}
&\sum _n{\rm Retr}[g(n)V_{\mu\nu\rho}(n)]\to S[U_{\nu}(n)]={\rm Retr}[U_{\nu}(n)N(n)]
\end{split}
\end{equation}
where
\begin{equation}
\begin{split}
&N(n)=\frac{1}{8}\left\{(g(n+a\mu)+g(n+a\nu+a\mu))U_{\mu}(n+a\nu)T_1U_{\mu}^{-1}(n)\right.\\
&\left.+(g(n-a\mu)+g(n+a\nu-a\mu))U_{\mu}^{-1}(n+a\nu-a\mu)T_2U_{\mu}(n-a\mu)\right.\\
&\left.+(g(n+a\rho)+g(n+a\nu+a\rho))U_{\rho}(n+a\nu)T_3U_{\rho}^{-1}(n)\right.\\
&\left.+(g(n-a\rho)+g(n+a\nu-a\rho))U_{\rho}^{-1}(n+a\nu-a\rho)T_4U_{\rho}(n-a\rho)\right\},\\
&T_1=U_{\rho}^{-1}(n+a\nu+a\mu-a\rho)U_{\nu}^{-1}(n+a\mu-a\rho)U_{\rho}(n+a\mu-a\rho)\\
&-U_{\rho}(n+a\nu+a\mu)U_{\nu}^{-1}(n+a\mu+a\rho)U_{\rho}^{-1}(n+a\mu),\\
&T_2=U_{\rho}(n+a\nu-a\mu)U_{\nu}^{-1}(n-a\mu+a\rho)U_{\rho}^{-1}(n-a\mu)\\
&-U_{\rho}^{-1}(n+a\nu-a\mu-a\rho)U_{\nu}^{-1}(n-a\mu-a\rho)U_{\rho}(n-a\mu-a\rho),\\
&T_3=U_{\mu}^{-1}(n+a\nu+a\rho-a\mu)U_{\nu}^{-1}(n+a\rho-a\mu)U_{\mu}(n+a\rho-a\mu)\\
&-U_{\mu}(n+a\nu+a\rho)U_{\nu}^{-1}(n+a\mu+a\rho)U_{\mu}^{-1}(n+a\rho),\\
&T_4=U_{\mu}(n+a\nu-a\rho)U_{\nu}^{-1}(n-a\rho+a\mu)U_{\mu}^{-1}(n-a\rho)\\
&-U_{\mu}^{-1}(n+a\nu-a\mu-a\rho)U_{\nu}^{-1}(n-a\mu-a\rho)U_{\mu}(n-a\mu-a\rho),\\
\end{split}
\end{equation}

One can further reduce the dagger operation by defining
\begin{equation}
\begin{split}
&S[U_{\mu}(n)]={\rm Retr}[U_{\mu}(n)M^{\dagger}(n)],\;\; S[U_{\nu}(n)]={\rm Retr}[U_{\nu}(n)N^{\dagger}(n)]
\end{split}
\end{equation}
with
\textcolor[rgb]{0,0,0.8}{
\begin{equation}
\begin{split}
&M(n)=\frac{1}{8}\left((g(n)+g(n+a\nu))S_1U_{\mu}(n+a\nu)U^{-1}_{\nu}(n+a\mu)\right.\\
&\left.+(g(n)+g(n-a\nu))S_2U_{\mu}(n-a\nu)U_{\nu}(n+a\mu-a\nu)\right.\\
&\left.+(g(n+a\mu)+g(n+a\mu+a\nu))U_{\nu}(n)U_{\mu}(n+a\nu)S_3\right.\\
&\left.+(g(n+a\mu)+g(n+a\mu-a\nu))U^{-1}_{\nu}(n-a\nu)U_{\mu}(n-a\nu)S_4\right)\\
&S_1=U_{\rho}(n)U_{\nu}(n+a\rho)U^{-1}_{\rho}(n+a\nu)-U^{-1}_{\rho}(n-a\rho)U_{\nu}(n-a\rho)U_{\rho}(n+a\nu-a\rho)\\
&S_2=U_{\rho}(n)U^{-1}_{\nu}(n-a\nu+a\rho)U^{-1}_{\rho}(n-a\nu)-U^{-1}_{\rho}(n-a\rho)U^{-1}_{\nu}(n-a\nu-a\rho)U_{\rho}(n-a\nu-a\rho)\\
&S_3=U^{-1}_{\rho}(n+a\mu-a\rho+a\nu)U^{-1}_{\nu}(n+a\mu-a\rho)U_{\rho}(n+a\mu-a\rho)\\
&-U_{\rho}(n+a\mu+a\nu)U^{-1}_{\nu}(n+a\mu+a\rho)U^{-1}_{\rho}(n+a\mu)\\
&S_4=U^{-1}_{\rho}(n+a\mu-a\rho-a\nu)U_{\nu}(n+a\mu-a\rho-a\nu)U_{\rho}(n+a\mu-a\rho)\\
&-U_{\rho}(n+a\mu-a\nu)U_{\nu}(n+a\mu+a\rho-a\nu)U^{-1}_{\rho}(n+a\mu)
\end{split}
\end{equation}
}
and
\textcolor[rgb]{0,0,0.8}{
\begin{equation}
\begin{split}
&N(n)=\frac{1}{8}(N(\mu,\rho)(n)+N(\rho,\mu)(n))\\
&N(\mu,\rho)(n)=\left\{(g(n+a\mu)+g(n+a\nu+a\mu))U_{\mu}(n)T_1U^{-1}_{\mu}(n+a\nu)\right.\\
&\left.+(g(n-a\mu)+g(n+a\nu-a\mu))U^{-1}_{\mu}(n-a\mu)T_2U_{\mu}(n+a\nu-a\mu)\right\},\\
&T_1=U^{-1}_{\rho}(n+a\mu-a\rho)U_{\nu}(n+a\mu-a\rho)U_{\rho}(n+a\nu+a\mu-a\rho)\\
&-U_{\rho}(n+a\mu)U_{\nu}(n+a\mu+a\rho)U^{-1}_{\rho}(n+a\nu+a\mu),\\
&T_2=U_{\rho}(n-a\mu)U_{\nu}(n-a\mu+a\rho)U^{-1}_{\rho}(n+a\nu-a\mu)\\
&-U^{-1}_{\rho}(n-a\mu-a\rho)U_{\nu}(n-a\mu-a\rho)U_{\rho}(n+a\nu-a\mu-a\rho),\\
\end{split}
\end{equation}
}

\textbf{Note, instead of $S_G[U_{\mu}]=-U_{\mu} \Sigma ^{\dagger}_{\mu}$, here it is $S_G[U_{\mu}]=\textcolor[rgb]{1,0,0}{+}U_{\mu} M^{\dagger} _{\mu}$ and $S_G[U_{\mu}]=\textcolor[rgb]{1,0,0}{+}U_{\mu} N^{\dagger} _{\mu}$.}

\subsubsection{\label{sec:TheForceFromFermionAction}The force from fermion action}

The first step is to shift the second term to factorize $U_{\mu}(n)$ out
\begin{equation}
\begin{split}
&\sum _{m,n}g(n)(1+\gamma _{\mu})U_{-\mu}(n)\delta _{n-a\mu,m}=\sum _{m,n}g(n)(1+\gamma _{\mu})U_{\mu}^{-1}(n-a\mu)\delta _{n-a\mu,m}\\
&=\sum _{m,n}g(n+a\mu)(1+\gamma _{\mu})U_{\mu}^{-1}(n)\delta _{n,m}\\
\end{split}
\end{equation}
It is more convenient to split the $D$ operator as (note that for $yU_x$, $g(n)=y$, and $g(n)=g(n+x)$, $xU_y$ is similar. Also, note that, it is also true for open boundary)
\begin{equation}
\begin{split}
&M^a=\left\{(1-\gamma _{\mu})T^aU_{\mu}\delta _{x_L,(x+\mu)_R}-(1+\gamma _{\mu})U_{\mu}^{-1}T^a\delta _{(x+\mu)_L,x_R}\right\}\\
&-y\Omega\delta _{\mu,x}\left\{(1-\gamma _4)T^aU_{\mu}\delta _{x_L,(x+\mu)_R}-(1+\gamma _4)U_{\mu}^{-1}T^a\delta _{(x+\mu)_L,x_R}\right\}\\
&+x\Omega\delta _{\mu,y}\left\{(1-\gamma _4)T^aU_{\mu}\delta _{x_L,(x+\mu)_R}-(1+\gamma _4)U_{\mu}^{-1}T^a\delta _{(x+\mu)_L,x_R}\right\}\\
&F_{pf}=2i\kappa \sum _a {\rm Im}\left[\left(\phi _1 ^{\dagger} M_a \phi _2\right)\right] T_a\\
\end{split}
\end{equation}
with
\begin{equation}
\begin{split}
&\phi _1=\left(\left(\hat{D}\hat{D}^{\dagger}\right)^{-1}\phi\right),\;\;\phi _2=\hat{D}^{\dagger}\phi _1,\\
\end{split}
\end{equation}
similarly, with
\begin{equation}
\begin{split}
&\phi_{L1}(n)=\phi _1(n),\;\;\phi _{R1}(n)=\left\{(1-\gamma _{\mu})+\left(x\Omega \delta _{\mu,y}-y\Omega \delta_{\mu,x}\right)(1-\gamma _4)\right\}\phi _2(n+\mu),\\
&\phi_{L2}(n)=\phi _1(n+\mu),\;\;\phi _{R1}(n)=\left\{(1+\gamma _{\mu})+\left(x\Omega \delta _{\mu,y}-y\Omega \delta_{\mu,x}\right)(1+\gamma _4)\right\}\phi _2(n),\\
\end{split}
\end{equation}
\begin{equation}
\begin{split}
&F^{pf}_{\mu}(n)=\kappa \left.\left\{U_{\mu}(n)\left(\phi _{R1}\phi _{L1}^{\dagger}+\phi _{R2}\phi _{L2}^{\dagger}\right)\right\}\right|_{TA}
\end{split}
\end{equation}

Note that \textbf{Both the force from gauge and fermion actions are kept anti-hermitian traceless.}

\subsubsection{\label{sec:AngularMomentum}The angular momentum}

The angular momentum operator is defined as
\begin{equation}
\begin{split}
&J\equiv \left.\frac{\delta \mathcal{L}}{\delta \Omega}\right|_{\Omega = 0}\\
&=J_G+J_{FL}+J_{FS}\\
&J_G=\frac{\beta}{N_c}\sum _n \left(\hat{x}{\rm Retr}[V_{412}(n)+V_{432}(n)]-\hat{y}{\rm Retr}[V_{421}(n)+V_{431}(n)]\right)\\
&J_{FL}=\bar{\psi}\left\{-\kappa \hat{y}\left((-\gamma _4)U_{x}(n)\delta _{n+a\hat{x},m}+(+\gamma _4)U_{-x}(n)\delta _{n-a\hat{x},m}\right)\right.\\
&+\left.\hat{x}\left((-\gamma _4)U_{y}(n)\delta _{n+a\hat{y},m}+(+\gamma _4)U_{-y}(n)\delta _{n-a\hat{y},m}\right)\right\}\\
&=\textcolor[rgb]{1,0,0}{-}\kappa \bar{\psi}\gamma _4 (\hat{y}D_x-xD_y)\psi,\\
&J_{FS}=-i\kappa \bar{\psi} \gamma _4 \sigma ^{12} \psi.\\
\end{split}
\end{equation}

The result is derived as $\delta \mathcal{L}/\delta \hat{\Omega}$, therefor, the result has unit as $a^{-3}$.

The measurement of $\langle J_G \rangle$ is straightforward. The measurement of $J_{FL}$ and $J_{FS}$ are inertia mass densities of quark-antiquark pairs. So, for the $u-\bar{u}$ pair. On the other hand, $J$ is \textbf{NOT} a local operator. $\langle J_F(n|m) \rangle$ can be written as (in the spinor space, where $a,b$ are spinor indices)
\begin{equation}
\begin{split}
&\langle J_F(n|m) \rangle=\langle \bar{u}(n) O(n|m) u(m) \rangle\\
&=\sum _{a,b} O_{a,b}(n|m) \langle \bar{u}_a(n)  u_b(m) \rangle = -\sum _{a,b} O_{a,b}(n|m) \langle   u_b(m) \bar{u}_a(n)\rangle \\
&=-\sum _{a,b} O_{a,b}(n|m) D^{-1}(m|n)_{b,a}=-{\rm tr}_{c,s}\left[O(n|m) D^{-1}(m|n)\right]\\
\end{split}
\end{equation}
So the definition of local angular momentum density should be
\begin{equation}
\begin{split}
&\langle J_F(n) \rangle=-\sum _{a,b,m} O_{a,b}(n|m) D^{-1}(m|n)_{b,a}=-{\rm tr}_{c,s,m}\left[O(n|m) D^{-1}(m|n)\right]\\
\end{split}
\end{equation}

So, we need to calculate $\sum _n O(m_0|n)D^{-1}(n|m_0)$ as a matrix in color and spinor space. It can be done by introduce the source
\begin{equation}
\begin{split}
&\phi ^{S}_{m_0,c_2,s_2}(m)_{c,s}=\delta (m-m_0)\delta (c-c_2)\delta (s-s_2)\\
\end{split}
\end{equation}
and $D^{-1}(n|m_0)_{c,s}$ as a vector $\vec{v}(n)$ can be written as
\begin{equation}
\begin{split}
&\left(\begin{array}{c} D^{-1}_{1,cs}(n) \\ D^{-1}_{2,cs}(n) \\ D^{-1}_{3,cs}(n) \\ \ldots \\ D^{-1}_{10,cs}(n)\\ D^{-1}_{11,cs}(n) \\ D^{-1}_{12,cs_2}(n)\end{array}\right)=\left(\begin{array}{ccccccc}
D^{-1}_{1,1} & D^{-1}_{1,2} & \ldots & \textcolor[rgb]{1,0,0}{D^{-1}_{1,cs}(n|m_0)} & \ldots &  D^{-1}_{1,11} & D^{-1}_{1,12} \\
D^{-1}_{2,1} & D^{-1}_{2,2} & \ldots & \textcolor[rgb]{1,0,0}{D^{-1}_{2,cs}(n|m_0)} & \ldots &  D^{-1}_{2,11} & D^{-1}_{2,12} \\
D^{-1}_{3,1} & D^{-1}_{3,2} & \ldots & \textcolor[rgb]{1,0,0}{D^{-1}_{3,cs}(n|m_0)} & \ldots &  D^{-1}_{3,11} & D^{-1}_{3,12} \\
\ldots & \ldots & \ldots & \ldots & \ldots &  \ldots & \ldots \\
D^{-1}_{10,1} & D^{-1}_{0,2} & \ldots & \textcolor[rgb]{1,0,0}{D^{-1}_{10,cs}(n|m_0)} & \ldots &  D^{-1}_{10,11} & D^{-1}_{10,12} \\
D^{-1}_{11,1} & D^{-1}_{11,2} & \ldots & \textcolor[rgb]{1,0,0}{D^{-1}_{11,cs}(n|m_0)} & \ldots &  D^{-1}_{11,11} & D^{-1}_{11,12} \\
D^{-1}_{12,1} & D^{-1}_{12,2} & \ldots & \textcolor[rgb]{1,0,0}{D^{-1}_{12,cs}(n|m_0)} & \ldots &  D^{-1}_{12,11} & D^{-1}_{12,12} \\
 \end{array}\right)\left(\begin{array}{c} 0\\ \ldots \\ 0 \\ 1 _{idx=cs,x=m_0}\\ 0 \\ \ldots \\ 0 \end{array}\right)
\end{split}
\end{equation}
and $\sum _n O(m|n)D^{-1}(n|m_0)_{c,s}$ as a vector $\vec{v}(m)$ can be written as
\begin{equation}
\begin{split}
&\left(\begin{array}{c} OD^{-1}_{1,cs}(m) \\ OD^{-1}_{2,cs}(m)  \\ \ldots  \\ OD^{-1}_{11,cs}(m) \\ OD^{-1}_{12,cs_2}(m) \end{array}\right)=\left(\begin{array}{ccccc}
O_{1,1} & O_{1,2} & \ldots &  O_{1,11} & O_{1,12} \\
O_{2,1} & O_{2,2} & \ldots &  O_{2,11} & O_{2,12} \\
O_{3,1} & O_{3,2} & \ldots &  O_{3,11} & O_{3,12} \\
\ldots & \ldots & \ldots &  \ldots & \ldots \\
O_{10,1} & O_{0,2} & \ldots & O_{10,11} & O_{10,12} \\
O_{11,1} & O_{11,2} & \ldots & O_{11,11} & O_{11,12} \\
O_{12,1} & O_{12,2} & \ldots &  O_{12,11} & O_{12,12} \\
 \end{array}\right)\left(\begin{array}{c} D^{-1}_{1,cs} \\ D^{-1}_{2,cs} \\ \ldots \\ D^{-1}_{11,cs} \\ D^{-1}_{12,cs_2}\end{array}\right)
\end{split}
\end{equation}
And the trace is just
\begin{equation}
\begin{split}
&\sum _{i=1}^{12}(OD^{-1})_{i,i}(n)
\end{split}
\end{equation}

\subsubsection{\label{sec:RotatingCurrentDensityChargeDensity}The Current density and Charge density}

In the case of exponential $\sigma ^{12}$ term, it is interesting to also measure
\begin{equation}
\begin{split}
J_{12}=-i\kappa \langle \bar{\psi} \gamma _4 \sigma ^{12} \psi \rangle
\end{split}
\end{equation}
Also the currents defined as
\begin{equation}
\begin{split}
J_{\mu}=\langle \bar{\psi} \gamma _{\mu} \psi \rangle
\end{split}
\end{equation}
is measured, such that
\begin{equation}
\begin{split}
&J_x=\langle \bar{\psi} (\gamma _1+y\Omega \gamma _4) \psi \rangle\\
&J_y=\langle \bar{\psi} (\gamma _2-x\Omega \gamma _4) \psi \rangle\\
&J_z=\langle \bar{\psi} \gamma _3 \psi \rangle\\
&J_{\tau}=\langle \bar{\psi} \gamma _4 \psi \rangle\\
\end{split}
\end{equation}
we also measure the
\begin{equation}
\begin{split}
&J_1=\langle \bar{\psi} \gamma _1 \psi \rangle\\
&J_2=\langle \bar{\psi} \gamma _2 \psi \rangle\\
\end{split}
\end{equation}
and the chiral charge density
\begin{equation}
\begin{split}
&n_5=a^3\langle \bar{\psi} \gamma _4\gamma _5\psi \rangle
\end{split}
\end{equation}

\subsubsection{\label{sec:RotatingTopologicalDensity}The Topological Density}

The topological charge is defined as (\textbf{\textcolor[rgb]{1,0,0}{This might has to be modified in the rotating frame!}})
\begin{equation}
\begin{split}
&Q = \frac{1}{32\pi^2}a^4 \sum _n \epsilon _{\mu\nu\rho\sigma}{\rm tr}\left[C_{\mu\nu}(n)C_{\rho\sigma}(n)\right]\\
&C_{\mu\nu}(n)={\rm Im}\left[U_{\mu\nu}(n)\right]
\end{split}
\end{equation}
Another definition is
\begin{equation}
\begin{split}
&Q = \frac{1}{32\pi^2}a^4 \sum _n \epsilon _{\mu\nu\rho\sigma}{\rm tr}\left[C^{clover}_{\mu\nu}(n)C^{clover}_{\rho\sigma}(n)\right]\\
&C^{clover}_{\mu\nu}(n)=\frac{1}{4}{\rm Im}\left[U_{\mu,\nu}(n)+U_{\nu,-\mu}(n)+U_{-\mu,-\nu}(n)+U_{-\nu,\mu}(n)\right]
\end{split}
\end{equation}

Note for both $C_{\mu\nu}$ and $C_{\mu\nu}^{clover}$, one have $C_{\mu\nu}=-C_{\nu\mu}$, alone with $\epsilon _{\mu\nu\rho\sigma}$, it doubles the term. Therefor
\begin{equation}
\begin{split}
&\sum \epsilon _{\mu\nu\rho\sigma}{\rm tr}[C_{\mu\nu}(n)C_{\rho\sigma}(n)]=8 \left({\rm tr}[C_{12}(n)C_{34}(n)]-{\rm tr}[C_{13}(n)C_{24}(n)]+{\rm tr}[C_{14}(n)C_{23}(n)]\right)
\end{split}
\end{equation}

\subsubsection{\label{sec:RotatingPolyakovLoop}The Polyakov loop}

Polyakov loop is measured straight forwardly.

\subsubsection{\label{sec:RotatingChiralCondensate}The Chiral Condensate}

The Chiral condensate can be calculate by Grassman number integral
\begin{equation}
\begin{split}
&\langle \bar{u}u\rangle = {\rm tr}[D_u^{-1}]
\end{split}
\end{equation}
We are using two degenerate fermions, so
\begin{equation}
\begin{split}
&\langle \bar{\psi}\psi\rangle = {\rm tr}[D^{-1}]=a^{-4}\frac{1}{m+\frac{4}{a}}{\rm tr}\left[\hat{D}^{-1}\right]=a^{-4}\times 2a\kappa {\rm tr}\left[\hat{D}^{-1}\right]=2a^{-3}\kappa {\rm tr}\left[\hat{D}^{-1}\right]
\end{split}
\end{equation}

\clearpage
