\subsection{\label{cuda}cuda}

\subsubsection{\label{cudathreads}blocks and threads}

\subsubsection{\label{device member function}device member function}

According to \href{https://stackoverflow.com/questions/53781421/cuda-the-member-field-with-device-ptr-and-device-member-function-to-visit-it-i}{https://stackoverflow.com/questions/53781421/cuda-the-member-field-with-device-ptr-and-device-member-function-to-visit-it-i}

To call device member function, the content of the class should be on device.

\begin{itemize}
  \item First, new a instance of the class.
  \item Then, create a device memory using cudaMalloc.
  \item Copy the content to the device memory
\end{itemize}

In other words, it will work as

\begin{lstlisting}
__global__ void _kInitialArray(int* thearray)
{
    int iX = threadIdx.x + blockDim.x * blockIdx.x;
    int iY = threadIdx.y + blockDim.y * blockIdx.y;
    int iZ = threadIdx.z + blockDim.z * blockIdx.z;
    thearray[iX * 16 + iY * 4 + iZ] = iX * 16 + iY * 4 + iZ;
}

extern "C" {
    void _cInitialArray(int* thearray)
    {
        dim3 block(1, 1, 1);
        dim3 th(4, 4, 4);

        _kInitialArray << <block, th >> > (thearray);
        checkCudaErrors(cudaGetLastError());
    }
}

class B
{
public:
    B()
    {
        checkCudaErrors(cudaMalloc((void**)&m_pDevicePtr, sizeof(int) * 64));
        _cInitialArray(m_pDevicePtr);
    }
    ~B()
    {
        cudaFree(m_pDevicePtr);
    }
    __device__ int GetNumber(int index)
    {
        m_pDevicePtr[index] = m_pDevicePtr[index] + 1;
        return m_pDevicePtr[index];
    }
    int* m_pDevicePtr;
};

__global__ void _kAddArray(int* thearray1, B* pB)
{
    int iX = threadIdx.x + blockDim.x * blockIdx.x;
    int iY = threadIdx.y + blockDim.y * blockIdx.y;
    int iZ = threadIdx.z + blockDim.z * blockIdx.z;
    thearray1[iX * 16 + iY * 4 + iZ] = thearray1[iX * 16 + iY * 4 + iZ] + pB->GetNumber(iX * 16 + iY * 4 + iZ);
}

extern "C" {
    void _cAddArray(int* thearray1, B* pB)
    {
        dim3 block(1, 1, 1);
        dim3 th(4, 4, 4);
        _kAddArray << <block, th >> > (thearray1, pB);
        checkCudaErrors(cudaGetLastError());
    }
}

class A
{
public:
    A()
    {
        checkCudaErrors(cudaMalloc((void**)&m_pDevicePtr, sizeof(int) * 64));
        _cInitialArray(m_pDevicePtr);
    }
    ~A()
    {
        checkCudaErrors(cudaFree(m_pDevicePtr));
    }
    void Add(B* toAdd/*this should be a device ptr(new on device function or created by cudaMalloc)*/)
    {
        _cAddArray(m_pDevicePtr, toAdd);
    }
    int* m_pDevicePtr;
};



int main(int argc, char * argv[])
{
    B* pB = new B();
    A* pA = new A();
    B* pDeviceB;
    checkCudaErrors(cudaMalloc((void**)&pDeviceB, sizeof(B)));
    checkCudaErrors(cudaMemcpy(pDeviceB, pB, sizeof(B), cudaMemcpyHostToDevice));
    pA->Add(pDeviceB);
    int* res = (int*)malloc(sizeof(int) * 64);
    checkCudaErrors(cudaMemcpy(res, pA->m_pDevicePtr, sizeof(int) * 64, cudaMemcpyDeviceToHost));
    printf("----------- A=");
    for (int i = 0; i < 8; ++i)
    {
        printf("\n");
        for (int j = 0; j < 8; ++j)
            printf("res %d=%d  ", i * 8 + j, res[i * 8 + j]);
    }
    printf("\n");
    //NOTE: We are getting data from pB, not pDeviceB, this is OK, ONLY because m_pDevicePtr is a pointer
    checkCudaErrors(cudaMemcpy(res, pB->m_pDevicePtr, sizeof(int) * 64, cudaMemcpyDeviceToHost));
    printf("----------- B=");
    for (int i = 0; i < 8; ++i)
    {
        printf("\n");
        for (int j = 0; j < 8; ++j)
            printf("res %d=%d  ", i * 8 + j, res[i * 8 + j]);
    }
    printf("\n");
    delete pA;
    delete pB;
    return 0;
}
\end{lstlisting}

\textcolor[rgb]{1,0,0}{Note: this is a copy of the original instance! It is ONLY OK to change the content of $pDevicePtr->m\_pOtherPtr$, NOT $pDevicePtr->somevalue$}

\subsubsection{\label{device virtual member function}device virtual member function}

According to \href{https://stackoverflow.com/questions/26812913/how-to-implement-device-side-cuda-virtual-functions}{https://stackoverflow.com/questions/26812913/how-to-implement-device-side-cuda-virtual-functions}

To call a device virtual member function, unlike Sec.~\ref{device member function}, the pointer to the virtual function table should also be on device,

\begin{itemize}
  \item First, cudaMalloc a sizeof(void*), for the device pointer.
  \item Then, use a kernel function to new the instance on device, and assign it to the device pointer created by cudaMalloc.
  \item One can copy the pointer, by using cudaMemcpy(void**, void**, sizeof(void*), devicetodevice).
  \item When copy it to elsewhere, one need to copy it back to host, then copy it again to device. The example shows how to copy it to constant.
\end{itemize}

in other words, it will work as
\begin{lstlisting}

class CA
{
public:
    __device__ CA() { ; }
    __device__ ~CA() { ; }
    __device__ virtual void CallMe() { printf("This is A\n"); }
};

class CB : public CA
{
public:
    __device__ CB() : CA() { ; }
    __device__ ~CB() { ; }
    __device__ virtual void CallMe() { printf("This is B\n"); }
};

__global__ void _kernelCreateInstance(CA** pptr)
{
    (*pptr) = new CB();
}

__global__ void _kernelDeleteInstance(CA** pptr)
{
    delete (*pptr);
}

extern "C" {
    void _kCreateInstance(CA** pptr)
    {
        _kernelCreateInstance << <1, 1 >> >(pptr);
    }

    void _kDeleteInstance(CA** pptr)
    {
        _kernelDeleteInstance << <1, 1 >> >(pptr);
    }
}

__constant__ CA* m_pA;

__global__ void _kernelCallConstantFunction()
{
    m_pA->CallMe();
}


extern "C" {
    void _cKernelCallConstantFunction()
    {
        _kernelCallConstantFunction << <1, 1 >> > ();
    }
}

int main()
{
    CA** pptr;
    cudaMalloc((void**)&pptr, sizeof(CA*));
    _kCreateInstance(pptr);

    //I can NOT use a kernel to set m_pA = (*pptr), because it is constant.
    //I can NOT use cudaMemcpyToSymbol(m_pA, (*pptr)), because * operator on host is incorrect when pptr is a device ptr.
    //I can NOT use cudaMemcpyToSymbol(m_pA, (*pptr)) in kernel, because cudaMemcpyToSymbol is a __host__ function
    //I have to at first copy it back to host, then copy it back back again to constant
    CA* pptrHost[1];
    cudaMemcpy(pptrHost, pptr, sizeof(CA**), cudaMemcpyDeviceToHost);
    cudaMemcpyToSymbol(m_pA, pptrHost, sizeof(CA*));
    _cKernelCallConstantFunction();

    _kDeleteInstance(pptr);
    cudaFree(pptr);
    return 0;
}

\end{lstlisting}
