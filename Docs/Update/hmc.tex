\subsection{\label{sec:hmc}\index{hmc}HMC}

HMC is abbreviation for hybrid Monte Carlo.

\subsubsection{\label{sec:hmc_action}The Fermion action}

Cooperating with HMC, the fermion is usually the \index{pseudofermions}'Pseudofermions`.

We begin with Eq.~(1.85) and Eq.~(1.86) of Ref.~\cite{latticeqcdbook2017}.
\begin{equation}
\begin{split}
&Z=\int \mathcal{D}[U] \prod _{f=1}^{N_f} \mathcal {D}[\bar{\psi}_f]\mathcal {D}[\psi _f]\exp \left(-S_G[U]\textcolor[rgb]{1,0,0}{-}\sum _{f=1}^{N_f}\bar{\psi} _f\left(\hat{D}_f\right)\psi _f\right)
\end{split}
\end{equation}
where $\hat{D}_f=D+m_f$. (Note that, there seems a typo in Eq.~(1.85) of Ref.~\cite{latticeqcdbook2017} and Eq.~(8.31) of Ref.~\cite{latticeqcdbook2010}, we have $S_F=\textcolor[rgb]{1,0,0}{+}\bar{\psi}D\psi$, see also Eqs.~(2.5) and (8.39) of Ref.~\cite{latticeqcdbook2010} and Eq.~(7.6) of Ref.~\cite{latticeqcdbook1998}, Eq.~(3.75) of Ref.~\cite{latticeqcdbook2017}, etc.)

It can be evaluated as Eq.~(1.86) of Ref.~\cite{latticeqcdbook2017}~(or Eq.~(4.19) of Ref.~\cite{condensedmatterbookAltland}) (Note, there is another minus sign in Eq.~(5.28) of Ref.~\cite{latticeqcdbook2010})
\begin{equation}
\begin{split}
&\int \mathcal{D}{\bar{\psi}}\psi \exp \left(\textcolor[rgb]{1,0,0}{-}\bar{\psi}A\psi\right)=\det \left(A\right),\\
&Z=\prod _{f=1}^{N_f} \det \left(\hat{D}_f\right)\int \mathcal{D}[U] \exp \left(-S_G[U]\right).\\
\end{split}
\end{equation}

On the other hand, with the help of Gaussian integral of complex vectors Eq.~(3.17) of Ref.~\cite{condensedmatterbookAltland}
\begin{equation}
\begin{split}
&\int d{\bf v}^{\dagger}d{\bf v} \exp (-{\bf v}^{\dagger} {\bf A} {\bf v})=\pi ^N \left(\det {\bf A}\right)^{-1}
\end{split}
\end{equation}
which is (3.31) of Ref.~\cite{latticeqcdbook2017}
\begin{equation}
\begin{split}
&\frac{1}{ \det ({\bf A}) }=\int \mathcal{D}[\eta] \exp (-\eta^{\dagger} {\bf A} \eta)
\end{split}
\end{equation}
where $\eta$ now is a complex Bosonic field, and the normalization
\begin{equation}
\begin{split}
&\mathcal{D}[\eta] = \prod \frac{d{\rm Re}(\eta _i)d{\rm Im}(\eta _i)}{\pi},\;\;1=\int \mathcal{D}[\eta] \exp (-\eta^{\dagger} \eta)
\end{split}
\end{equation}
is assumed. With the condition such that
\begin{equation}
\begin{split}
&\lambda ({\bf A}+{\bf A}^{\dagger}) > 0.
\end{split}
\end{equation}
where $\lambda ({\bf M})$ denoted as eigen-values of ${\bf M}$.

We now, concentrate on two degenerate fermion flavours. i.e. considering
\begin{equation}
\begin{split}
&S_F=\bar{\psi}_u \hat{D} \psi _u+\bar{\psi}_d \hat{D} \psi _d.
\end{split}
\end{equation}

Using $\det (DD^{\dagger})=\det (D)\det (D^{\dagger})$ and $\det(M^{-1})=\left(\det (M)\right)^{-1}$ and \textcolor[rgb]{1,0,0}{$\det (D)=\det (D^{\dagger})$} (\textbf{Only for Wilson Fermions or $\gamma _5$-hermiticity fermions, $\hat{D}^{\dagger}=\gamma _5 D \gamma _5 + m=\gamma _5 (D+m) \gamma _5=\gamma _5 \hat{D} \gamma _5$, and $\det(\hat{D}^{\dagger})=\det(\gamma _5)\det(\hat{D})\det(\gamma _5)=\det (\hat{D})$.} See also Ref.~\cite{weingarten1981}.), one can show Eq.~(8.9) of Ref.~\cite{latticeqcdbook2010} (Eq.~(2.77) of Ref.~\cite{latticeqcdreview2009})
\begin{equation}
\begin{split}
&\int \mathcal{D}[\bar{\psi}]\mathcal{D}[\psi]\exp\left(-\bar{\psi}_u \hat{D} \psi _u-\bar{\psi}_d \hat{D} \psi _d\right)=\det (\hat{D}\hat{D}^{\dagger})=\int \mathcal{D}[\phi] \exp (\textcolor[rgb]{1,0,0}{-}\phi ^{\dagger}\left(\hat{D}\hat{D}^{\dagger}\right)^{-1} \phi)
\end{split}
\end{equation}
where $\phi$ now is a complex Bosnic field. (Note that, there is a sign typo in Eq.~(8.31) of Ref.~\cite{latticeqcdbook2010}, see also Eqs.~(8.38) and (8.39) of Ref.~\cite{latticeqcdbook2010} )

So, generally, we are using HMC to evaluate the action with 'Pseudofermions`, or in other words, we are working with an action including only gauge and bosons.
\begin{equation}
\begin{split}
&S=S_G+S_{pf}=S_G+\phi ^{\dagger}\left(\hat{D}\hat{D}^{\dagger}\right)^{-1} \phi\\
\end{split}
\end{equation}
where $pf$ is short for pseudofermion.

\subsubsection{\label{sec:hmc_scheme}Basic idea, force from gauge field}

The basic idea is to use a \index{molecular dynamics}molecular dynamics simulation, i.e, it is a integration of \index{Langevin equation}Langevin equation.

Treating $SU(N)$ matrix $U$ on links as coordinate, HMC will generate a pair of configurations, $(P,U)$, where $P$ is momentum and $P\in \mathfrak{su}(N)$.

One can:

\begin{enumerate}
  \item Create a random $P=i\sum _a\omega _a T_a$, where $\omega _a\in \mathbb{R}$.
  \item Obtain $\dot{P}$, $\dot{U}$. Note that, dot is $d/d\tau$, where $\tau$ is `Markov time'.
  \item Numerically evaluate the differential equation, and use a Metropolis accept / reject to update.
\end{enumerate}

\begin{itemize}
  \item About the randomized $P$
\end{itemize}

The randomized $P$ is chosen according to normal distribution $\exp \left(-P^2 /2\right)$

Note that, here $P$ corresponds to $Q$, not $U$, for $U=\exp\left(i \sum q_a T^a\right)$, there are $8$ \textbf{real} variables denoting as $\omega _i$.

Using $P=\sum \omega _a T^a$, $tr((T^a)\cdot (T^b))=\frac{1}{2}\delta _{ab}$. So one have $\frac{1}{2}\sum _a\omega _a^2=tr[P^2]$.

It is usually written as distribution $\exp \left(-tr(P^2)\right)$ (where $P$ is a matrix, and $tr[P^2]=\frac{1}{2}p^2$ where $p=(\omega_1,\omega_2,\ldots ,\omega _8)$).

Using the property of normal distribution
\begin{equation}
\begin{split}
&if\; \{X\}\sim N(\mu _X, \sigma _X^2),\;\; \{Y\}\sim N(\mu _Y, \sigma _Y^2),\\
&\{X+Y\}\sim N(\mu _X+\mu _Y, \sigma _X^2+\sigma _Y^2).\\
\end{split}
\end{equation}

One can randomize $\omega _a$ using $\exp \left(-\omega_a\omega_a\right)$. Then using $P=\frac{1}{\sqrt{8N}}\sum \omega _a T^i$, where $N$ is the number of links.

\textcolor[rgb]{1,0,0}{\textbf{Note: Here is a difference between Refs.~\cite{latticeqcdbook2010} and \cite{latticeqcdbook2017} and Bridge++~\cite{bridge}}}

Note, by Eq.~(8.16) of Ref.~\cite{latticeqcdbook2010}, $P^2=\sum _{n\in \Lambda}P^2(n)$, so when the lattice is large, $P$ become very small. See also the definition of $\langle P,P\rangle$ below Eq.~(2.42) of Ref.~\cite{latticeqcdbook2017}.

However, in Bridge++, it uses distribution $\exp \left(-tr(P^2)/DOF\right)$, where `DOF' is the degrees of freedom, i.e., number of links.

We use the distribution same as in Bridge++. Imagining that for a very small (hot) $\beta \to 0$, the force is also almost $0$ so momentum is unchanged when evolution. Considering a very large lattice such that the momentum is very small when using distribution $\exp \left(-tr(P^2)\right)$, the gauge field will stay near the initial value rather then becoming hot (randomized). So we think it should be $\exp \left(-tr(P^2)/DOF\right)$.

\begin{itemize}
\item \index{force}Force
\end{itemize}

Defined by Newton, $dp/dt$ is a force, so $\dot{P}$ is called `force'. See Eqs.~(2.53), (2.56) and (2.57) of Ref.~\cite{latticeqcdbook2017}, for $SU(N)$,
\begin{equation}
\begin{split}
&S_G[U_{\mu}(n)]=-\frac{\beta}{N}{\rm Re}{\rm tr}\left[U_{\mu}(n)\Sigma ^{\dagger}_{\mu}(n)\right]\\
&\Sigma _{\mu}(n)=\sum _{\mu \textcolor[rgb]{1,0,0}{\neq}\nu}\left(U_{\nu}(n)U_{\mu}(n+a\nu)U^{-1}_{\nu}(n+a\mu)+U^{-1}_{\nu}(n-a\nu)U_{\mu}(n-a\nu)U_{\nu}(n-a\nu+a\mu)\right)\\
\end{split}
\label{eq.hmc.staple}
\end{equation}
\textbf{Note that $S_G\neq \sum _{\mu,n} S_G[U_{\mu}(n)]$. $S_G[U_{\mu}(n)]$ is convenient for derivate which collecting all terms related to the specified bond. For plaquettes with 4 edges, $S_G=\frac{1}{4} \sum _{\mu,n} S_G[U_{\mu}(n)]$. }

$S_G$ the action for a particular $U_{\mu}(n)$. $\Sigma$ is the \index{staple}`staple'(see Eq.~(\ref{eq.plaqutteEnergy.staple})). The staple for $U_{\mu}(n)$ is independent of $U_{\mu}(n)$, denoting
\begin{equation}
\begin{split}
&U_{\mu}(n)=\exp \left(i \sum _a\omega _a(\mu,n) T_a\right)U_{\mu}^0(n)
\end{split}
\end{equation}
so
\begin{equation}
\begin{split}
&\frac{\partial}{\partial \omega _a (\mu,n)}S_G=-\frac{\beta}{N}{\rm Re}{\rm tr}\left[\frac{\partial}{\partial \omega _a}U_{\mu}(n)\Sigma ^{\dagger}_{\mu}(n)\right]=-\frac{\beta}{2N}{\rm tr}\left[\frac{\partial}{\partial \omega _a}\left(U_{\mu}(n)\Sigma ^{\dagger}_{\mu}(n)+\Sigma _{\mu}(n)U^{\dagger}_{\mu}(n)\right)\right]\\
&=-i\frac{\beta}{2N}{\rm tr}\left[T_aU_{\mu}(n)\Sigma ^{\dagger}_{\mu}(n)-\Sigma _{\mu}(n)T^{\dagger}_aU^{\dagger}_{\mu}(n)\right]=-i\frac{\beta}{2N}{\rm tr}\left[T_a\left(U_{\mu}(n)\Sigma ^{\dagger}_{\mu}(n)-\Sigma _{\mu}(n)U^{\dagger}_{\mu}(n)\right)\right]\\
&=\frac{\beta}{N} {\rm Im\;tr}\left[T_aU_{\mu}(n)\Sigma ^{\dagger}_{\mu}(n)\right]\\
\end{split}
\end{equation}
This is the Eq.~(8.41) of Ref.~\cite{latticeqcdbook2010}.

Using (Checked by Mathematica that Eq.~(8.42) of Ref.~\cite{latticeqcdbook2010} is incompatable with our notation, but replacing the $UA-A^{\dagger}U^{\dagger}$ of Eq.~(8.42) with $\{UA\}_{TA}$ is correct. Also, Eq.~(2.58) of Ref.~\cite{latticeqcdbook2017} is different from ours, in our formulism, it is correct by replacing $2T_a{\rm Re}[tr[T_a\cdot W]]$ of Eq.~(2.58) with $2iT_a{\rm Im}[tr[T_a\cdot W]]$)
\begin{equation}
\begin{split}
&\sum _a {\rm tr}\left[T_a\left(U_{\mu}(n)\Sigma ^{\dagger}_{\mu}(n)-\Sigma _{\mu}(n)U^{\dagger}_{\mu}(n)\right)\right]T_a=2i\sum _a{\rm Im}\left[T_aU_{\mu}(n)\Sigma ^{\dagger}_{\mu}(n)\right]T_a=\{U_{\mu}(n)\Sigma ^{\dagger}_{\mu}(n)\}_{TA}\\
&\{W\}_{TA}=\frac{W-W^{\dagger}}{2}-{\rm tr}\left(\frac{W-W^{\dagger}}{2N}\right)\mathbb{I}\\
\end{split}
\end{equation}
where $\mathbb{I}$ is identity matrix. Therefor
\begin{equation}
\begin{split}
&\dot{\omega} _a=-\frac{\partial}{\partial \omega _a (\mu,n)}S_G\\
&F_{\mu}(x)=\dot{P}_{\mu}(x)=i\sum \dot{\omega}_a T_a=-i\frac{\partial}{\partial \omega _a (\mu,n)}S_G T_a=-\frac{\beta}{2N}\{U_{\mu}(n)\Sigma ^{\dagger}_{\mu}(n)\}_{TA}\\
\end{split}
\label{eq.hmc.force}
\end{equation}
Note that, $\dot{\omega}_a=\frac{\beta}{N}{\rm Im}[tr[T_a\cdot W]]$ is still a \textbf{real} number.

Eq.~(\ref{eq.hmc.force}) is same as Eqs.~(2.53), (2.56) and (2.57) of Ref.~\cite{latticeqcdbook2017}.

\begin{itemize}
\item \index{Integrator}Integrator
\end{itemize}

Knowing $\dot{P}$, and $\dot {U}$, to obtain $U$ and $P$ is simply
\begin{equation}
\begin{split}
&U(\tau+d\tau)\approx \dot{U}d\tau + U(\tau),\;\;P(\tau+d\tau)\approx \dot{P}d\tau + P(\tau)\\
\end{split}
\end{equation}

A more accurate calculation is done by integrator, for example, the leap frog integrator, the $M$ step leap frog integral is described in Ref.~\cite{latticeqcdbook2010},
\begin{subequations}
\begin{eqnarray}
&\epsilon = \frac{\tau}{M}\\
&U_{\mu}(x,(n+1)\epsilon)=U_{\mu}(x,n\epsilon)+\epsilon P_{\mu}(x,n\epsilon)+\frac{1}{2}F_{\mu}(x,n\epsilon)\epsilon ^2\\
&P_{\mu}(x,(n+1)\epsilon)=P_{\mu}(x,n\epsilon)+\frac{1}{2}\left(F_{\mu}(x,(n+1)\epsilon)+F_{\mu}(x,n\epsilon)\right)\epsilon
\end{eqnarray}
\label{eq.hmc.update_basic}
\end{subequations}

So, knowing $U(n\epsilon)$ we can calculate $F(n\epsilon)$ using Eq.~(\ref{eq.hmc.force}).
Knowing $U(n\epsilon),P(n\epsilon),F(n\epsilon)$, we can calculate $U((n+1)\epsilon)$ using Eq.~(\ref{eq.hmc.update_basic}).b.
Then we are able to calculate $F((n+1)\epsilon)$ again using Eq.~(\ref{eq.hmc.force}).
Then we can calculate $P((n+1)\epsilon)$ using Eq.~(\ref{eq.hmc.update_basic}).c.

\subsubsection{\label{sec:forceOfPseudofermions}Force of pseudofermions}

For important sampling, one can generate both $U$ and $\phi$ by $e^{-S}$. In molecular dynamics simulation, it can be simplified as:

\begin{enumerate}
  \item Evaluate $U$ use force of $U$ and $\phi$ on $U$.
  \item Evaluate $\phi$ use force of $U$ and $\phi$ on $\phi$.
\end{enumerate}

The second step can be simplified as, generating random complex numbers $\phi$ according to $\exp (-\phi^{\dagger} \left(\hat{D}\hat{D}^{\dagger}\right)^{-1}\phi)=\exp (-\phi^{\dagger}(\hat{D}^{\dagger})^{-1} \hat{D}^{-1}\phi)$. $D[U]$ is a function of $U$.

\textbf{How to get randomized $\phi$?} Let $\chi$ be random \textbf{complex} numbers according to $\exp (-\chi ^{\dagger}\chi)$. Let $\hat{D}^{-1}\phi=\chi$, $\phi$ is the random \textbf{complex} number satisfying distribution we want ($\exp (-\phi^{\dagger}(\hat{D}^{\dagger})^{-1} \hat{D}^{-1}\phi)$). So, first get $\chi$ and then let $\phi = D\chi$.

Using the Wilson Fermion action
\begin{equation}
\begin{split}
&\hat{D}=C(D+1)\\
&D=-\kappa\sum _{\mu}\left((1-\gamma _{\mu})U_{\mu}(x _L)\delta _{x _L,(x+\mu)_R}+(1+\gamma _{\mu})U_{\mu}^{-1}(x _L-\mu)\delta _{x_L,(x-\mu)_R}\right)\\
\end{split}
\end{equation}
with $C=m_f+\left(4/a\right)=1/2a\kappa$ and $\kappa=1/(2am_f+8)$. One can rescale the field and set $C=1$.

The force of $\phi$ on $U$ is obtained as $\partial _{\omega_a}S_{pf}$. The result for Wilson Fermion action is shown in Eqs.~(8.39), (8.44) and (8.45) of Ref.~\cite{latticeqcdbook2010} as
\begin{equation}
\begin{split}
&F=i\sum _a\dot{\omega}_aT_a=i\sum _a\left(-\partial _{\omega _a}\left(S_G[U_{\mu}(n)]+S_{pf}[U_{\mu}(n)]\right)\right)T_a=F_G+F_{pf}.\\
&F_{pf}=i\sum _a\left(-\partial _{\omega _a}S_{pf}[U_{\mu}(n)]\right)T_a=-i\sum_aT^a \frac{\partial}{\partial \omega_a} \left(\phi ^{\dagger}\left(\hat{D}\hat{D}^{\dagger}\right)^{-1}\phi\right).\\
&\frac{\partial}{\partial \omega_a} \left(\phi ^{\dagger}\left(\hat{D}\hat{D}^{\dagger}\right)^{-1}\phi\right) = -\left(\left(\hat{D}\hat{D}^{\dagger}\right)^{-1}\phi\right)^{\dagger}\left(\frac{\partial D}{\partial \omega _{\mu}^a}\hat{D}^{\dagger}+\hat{D}\frac{\partial D^{\dagger}}{\partial \omega _{\mu}^a}\right)\left(\left(\hat{D}\hat{D}^{\dagger}\right)^{-1}\phi\right).\\
&\frac{\partial \hat{D}}{\partial {\omega _{\mu}^a}}=\left(\frac{\partial D}{\partial {\omega _{\mu}^a}}\right)_{x_L.x_R}=-i\kappa \left\{(1-\gamma _{\mu})T^aU_{\mu}(x)\delta_{x,x_L}\delta _{x,(x+\mu)_R}-(1+\gamma _{\mu})U_{\mu}^{-1}(x)T^a\delta _{x,(x\textcolor[rgb]{1,0,0}{+}\mu)_L}\delta _{x,x_R} \right\} \\
&\hat{D}^{\dagger} = \gamma _5 \hat{D} \gamma _5, \;\;\frac{\partial D^{\dagger}}{\partial \omega _{\mu}^i}=\gamma _5 \frac{\partial D}{\partial \omega _{\mu}^a} \gamma _5\\
\end{split}
\end{equation}
where $F_G$ is force from $U$ introduced in Sec.~\ref{sec:hmc_scheme}, $T^a$ are $SU(3)$ generators. $x_L,x_R$ are coordinate index of the left and right pseudofermion field. And
\begin{equation}
\begin{split}
&U_{\mu}=\exp (i\sum _a \omega _{\mu}^a T^a)U_0,\;\;\frac{\partial U_{\mu}}{\partial \omega_{\mu}^a}=iT^aU_{\mu},\;\;\frac{\partial U^{\dagger}_{\mu}}{\partial \omega_{\mu}^a}=-iU^{\dagger}_{\mu}T^a,\\
&\left(T^a\right)^{\dagger}=T^a,\;\;\frac{\partial M^{-1}}{\partial \omega _{\mu}^a}=-M^{-1}\frac{\partial M}{\partial \omega _{\mu}^a}M^{-1}\\
\end{split}
\end{equation}
are used. (Note that, Eq.~(8.45) of Ref.~\cite{latticeqcdbook2010} has a sign typo, see also Eq.~(2.82) of Ref.~\cite{latticeqcdreview2009})

We can simplify it further by $\left(\hat{D}^{\dagger}(\hat{D}\hat{D}^{\dagger})^{-1}\phi\right)^{\dagger}=\left((\hat{D}\hat{D}^{\dagger})^{-1}\phi\right)^{\dagger}\hat{D}$, so
\begin{equation}
\begin{split}
&\phi _1=\left(\left(\hat{D}\hat{D}^{\dagger}\right)^{-1}\phi\right),\;\;
 \phi _2=\hat{D}^{\dagger}\left(\left(\hat{D}\hat{D}^{\dagger}\right)^{-1}\phi\right)=D^{-1}\phi,\;\;\phi _1^{\dagger}D=\phi _2^{\dagger},\\
&\frac{\partial}{\partial \omega_a} \left(\phi ^{\dagger}\left(\hat{D}\hat{D}^{\dagger}\right)^{-1}\phi\right)=-\left(\left(\hat{D}\hat{D}^{\dagger}\right)^{-1}\phi\right)^{\dagger}\left(\frac{\partial D}{\partial \omega _{\mu}^a}\hat{D}^{\dagger}+\hat{D}\frac{\partial D^{\dagger}}{\partial \omega _{\mu}^a}\right)\left(\left(\hat{D}\hat{D}^{\dagger}\right)^{-1}\phi\right)\\
&=-\left(\phi _1^{\dagger} \frac{\partial D}{\partial \omega _{\mu}^a} \phi _2+\phi _2^{\dagger} \frac{\partial D^{\dagger}}{\partial \omega _{\mu}^a} \phi _1\right)=-2{\rm Re}\left[\left(\phi _1 ^{\dagger} \frac{\partial D}{\partial \omega _{\mu}^a} \phi _2\right)\right]\\
\end{split}
\end{equation}
and
\begin{equation}
\begin{split}
&\frac{\partial D}{\partial \omega _{\mu}^a}=-i\kappa M_a,\\
&\left(M_a\right)_{x_L,x_R}=\left\{(1-\gamma _{\mu})T^aU_{\mu}\delta _{x_L,(x+\mu)_R}-(1+\gamma _{\mu})U_{\mu}^{-1}T^a\delta _{(x+\mu)_L,x_R}\right\}\\
&\frac{\partial}{\partial \omega_a} \left(\phi ^{\dagger}\left(\hat{D}\hat{D}^{\dagger}\right)^{-1}\phi\right)=-2\kappa{\rm Im}\left[\left(\phi _1 ^{\dagger} M \phi _2\right)\right]\\
\end{split}
\end{equation}
Again, $\dot{\omega}$ is a \textbf{real} number, and
\begin{equation}
\begin{split}
&F_{pf}=-i\sum _a T^a \frac{\partial}{\partial \omega_a} \left(\phi ^{\dagger}\left(\hat{D}\hat{D}^{\dagger}\right)^{-1}\phi\right)=2i\kappa \sum _a {\rm Im}\left[\left(\phi _1 ^{\dagger} M_a \phi _2\right)\right] T_a\\
\end{split}
\end{equation}

So we can calculate $\phi _1$ first, then $\phi _2 = \hat{D}^{\dagger}\phi _1$. Then contract the spinor and color space with $\partial D / \partial \omega$.

Note that, $D$ is changing when integrating the Langevin equation.

The last part is how to calculate $(\hat{D}\hat{D}^{\dagger})^{-1}$.

\begin{itemize}
  \item Anti-Hermitian traceless of the force
\end{itemize}

See from Eq.~(\ref{eq.hmc.force}), the force from the gauge field is an anti-Hermitian traceless matrix.

The result above can be further simplified. Note that
\begin{equation}
\begin{split}
&\phi_{L1}(n)=\phi _1(n),\;\;\phi _{R1}(n)=(1-\gamma _{\mu})\phi _2(n+\mu),\\
&\phi_{L2}(n)=\phi _1(n+\mu),\;\;\phi _{R1}(n)=(1+\gamma _{\mu})\phi _2(n),\\
\end{split}
\end{equation}
One have
\begin{equation}
\begin{split}
&{\rm Im}\left[\phi _1^{\dagger}M\phi _2\right]_{\mu}^a(n)={\rm Im}\left[\phi _{L1}^{\dagger}T^aU_{\mu}(n)\phi _{R1}\right]-{\rm Im}\left[\phi _{L2}^{\dagger}U_{\mu}^{\dagger}(n)T^a\phi _{R2}\right]\\
&={\rm Im}\left[\phi _{L1}^{\dagger}T^aU_{\mu}(n)\phi _{R1}\right]+{\rm Im}\left[\phi _{R2}^{\dagger}T^aU_{\mu}(n)\phi _{L2}\right]\\
\end{split}
\end{equation}

For any vector
\begin{equation}
\begin{split}
&{\rm Im}\left[L^{\dagger}TUR\right]={\rm Im}\left[\sum _{\alpha,\beta,\rho}L_{\alpha}^*T_{\alpha\beta}U_{\beta\rho}R_{\rho}\right]={\rm Im}\left[\sum _{\alpha,\beta,\rho}T_{\alpha\beta}U_{\beta\rho}R_{\rho}L_{\alpha}^*\right]={\rm Im}\left[{\rm tr}\left[TU(RL^{\dagger})\right]\right]\\
\end{split}
\end{equation}

So
\begin{equation}
\begin{split}
&F^{pf}_{\mu}(n)=2i\kappa{\rm Im}\left[\phi _1^{\dagger}M\phi _2\right]_{\mu}(n)=\kappa \left(2i\sum _a{\rm Im}{\rm tr}\left[T^a U_{\mu}(n)\left(\phi _{R1}\phi _{L1}^{\dagger}+\phi _{R2}\phi _{L2}^{\dagger}\right)\right]T^a\right)\\
&=\kappa \left.\left\{U_{\mu}(n)\left(\phi _{R1}\phi _{L1}^{\dagger}+\phi _{R2}\phi _{L2}^{\dagger}\right)\right\}\right|_{TA}
\end{split}
\end{equation}
which is also an anti-Hermitian traceless matrix.

\textbf{So, the momentum is always anti-Hermitian traceless.}.

For anti-Hermitian traceless matrix $M$, the $\exp(M)$ can be simplified as Appendix. A of Ref.~\cite{luscher2005}.

\subsubsection{\label{sec:Solver_In_HMC}\index{solver}Solver in HMC}

To calculate $(\hat{D}\hat{D}^{\dagger})^{-1}$, we need a solver. The detail of solvers will be introduced in Sec.~\ref{sec:solver}. Here we establish a simple introduction.

Let $M$ be a matrix operating on a vector, for example, $M=(\hat{D}\hat{D}^{\dagger})$, the goal of the solver is to find $x$ such $b=M\cdot x$, and therefor $x=(\hat{D}\hat{D}^{\dagger})^{-1}b$.

We first introduce the CG algorithm for real vector and real matrix, define
\begin{equation}
\begin{split}
&Q({\bf x})=\frac{1}{2}{\bf x}^T \cdot A \cdot {\bf x}-{\bf x}^T {\bf b}.
\end{split}
\end{equation}
so that one can try to find the minimum of $Q$, and at the minimum
\begin{equation}
\begin{split}
&\frac{\partial }{\partial {\bf x}}Q({\bf x})=0=A\cdot {\bf x}-{\bf b}.
\end{split}
\end{equation}

To find the minimum, one can use gradient. Starting from a random point on a curve, calculate the falling speed and move it until it is stable.

For complex vector, one can use \index{BiCGStab}BiCGStab in Table.~6.2 in Ref.~\cite{latticeqcdbook2010}. It can be described as

\begin{algorithm}[H]
\begin{algorithmic}
\State ${\bf x}={\bf b}$
\Comment{Use ${\bf b}$ as trail solution and start.}
\For{$i=0$ to $r$}
    \State ${\bf r}={\bf b}-A{\bf x}$
    \Comment {Restart $r$ times}
    \State ${\bf r}_h={\bf r}$
    \For{$j=0$ to $itera$}
        \State $\rho = {\bf r}_h^* \cdot {\bf r}_j$
        \If {$j=0$}
            \State ${\bf p}={\bf r}$
        \Else
            \State $\beta = \alpha \times \rho / (\omega \times \rho_p)$
            \State ${\bf p}={\bf r}+\beta\left({\bf p}-\omega {\bf v}\right)$
        \EndIf
        \State ${\bf v}=A{\bf p}$
        \State $\alpha = \rho / \left({\bf r}_h^*\cdot {\bf v}\right)$
        \State ${\bf s}={\bf r}-\alpha {\bf v}$
        \If {$0\neq j$ and $0=\mod(j,5)$}
            \State $er=\|{\bf s}\|$
            \Comment {Check deviation every 5 steps}
            \If {$er<\epsilon$}

                \Return ${\bf x}$
            \EndIf
        \EndIf
        \State ${\bf t}=A{\bf s}$
        \State $\omega = {\bf s}^*\cdot {\bf t}/\|{\bf t}\|$
        \State ${\bf r}={\bf s}-\omega {\bf t}$
        \State ${\bf x}={\bf x}+\alpha {\bf p}+\omega {\bf s}$
        \State $\rho _p=\rho$
        \Comment {Preserve the last calculated $\rho$ become we still need it}
    \EndFor
\EndFor
\end{algorithmic}
\caption{\label{alg.BiCGStab}BiCGStab, note that, the numbers are \textbf{complex} number.}
\end{algorithm}

\subsubsection{\label{sec:Leap frog}\index{leap frog}Leap frog integrator}

In Sec.~\ref{sec:hmc_scheme}, the basic idea is introduced. However, the implementation is slightly different.
\begin{subequations}
\begin{eqnarray}
&U_{\mu}(0,x)=gauge(x),\;\;P_{\mu}(0,x)=\sum _{a}r_a(\mu,x)T_a\\
&F_{\mu}(n\epsilon,x)=-\frac{\beta}{2N}\{U_{\mu}(n\epsilon,x)\Sigma _{\mu}(n\epsilon,x)\}_{TA}\\
&P_{\mu}(\frac{1}{2}\epsilon,x)=P_{\mu}(0,x)+\frac{\epsilon}{2}F_{\mu}(0,x)\\
&U_{\mu}((n+1)\epsilon,x)=\exp \left(i\epsilon P_{\mu}((n+\frac{1}{2})\epsilon,x)\right)U_{\mu}(n\epsilon,x)\\
&P_{\mu}((n+\frac{1}{2})\epsilon,x)=P_{\mu}((n-\frac{1}{2})\epsilon,x)+\epsilon F_{\mu}(n\epsilon,x)
\end{eqnarray}
\label{eq.hmc.update_leapfrog}
\end{subequations}

Note that, the sign of $F$ is `+' here which is different from Ref.~\cite{latticeqcdbook2010}, because in Ref.~\cite{latticeqcdbook2010}, $F=\partial _{\mu,n}S = -\dot{P}$. Here we define $F=\dot{P}=-\partial _{\mu,n}S$.

Or simply written as
\begin{equation}
\begin{split}
&P_{\epsilon}\circ U_{\epsilon}\circ P_{\frac{1}{2}\epsilon}\left(P_0,U_0\right)
\end{split}
\label{eq.hmc.update_leapfrog2}
\end{equation}
The pseudo code can be written as

\begin{algorithm}[H]
\begin{algorithmic}
\State ${\bf f}=CalculateForce(actions,{\bf U})$
\State ${\bf p}={\bf p}+0.5 \times \epsilon {\bf f}$
\For{$i=1$ to $n$}
    \State ${\bf U}=\exp (\epsilon {\bf p})U$
    \State ${\bf f}=CalculateForce(actions,{\bf U})$
    \If {$i=n$}
        \State ${\bf p}={\bf p}+0.5 \times \epsilon {\bf f}$
        \Comment {We still need to update ${\bf p}$ for the Metropolis step.}
    \Else
        \State ${\bf p}={\bf p}+\epsilon {\bf f}$
    \EndIf
\EndFor
\end{algorithmic}
\caption{leap-frog integration}
\end{algorithm}

\subsubsection{\label{sec:summaryOfHMC}A summary of HMC with pseudofermions}

Now, every part is ready. We summary the HMC following the Sec.8.2.3 in Ref.~\cite{latticeqcdbook2010}. The HMC with fermions can be divided into 6 steps.

\begin{enumerate}
  \item Generate a complex Bosonic field with $\chi \sim \exp (-\chi ^{\dagger}\chi)$, and $\phi = \hat{D} \chi$.
  \item Generate a momentum field $P$ by $\exp (-tr(P^2))$.
  \item Calculate $E=tr(P^2)+S_G(U)+S_{pf}(U,\phi)$.
  \item Use $U_0$ to calculate $F$, evaluate $P$ and $U$ using integrator. Here, $\phi$ is treated as a constant field.
  \item Finally, use $P',U'$ to calculate Calculate $E'=tr({P'}^2)+S_G(U')+S_{pf}(U',\phi)$. Use a Metropolis to accept or reject the result~(configurations) \textbf{Note, by Refs.~\cite{latticeqcdbook2010} and ~\cite{latticeqcdreview2009} `reject' means add a duplicated old configuration.}.
  \item Iterate from 1 to 5, until the number of configurations generated is sufficient.
\end{enumerate}

\begin{itemize}
 \item More on \index{Metropolis}Metropolis step:
\end{itemize}

If the hybrid Monte Carlo can be implemented exactly, then, when \index{equilibrium}equilibrium is reached, $H$ should be unchanged, so, in some implementation, the Metropolis step can be ignored to archive a better accept rate. The parameter \verb"Metropolis" of parameter \verb"Updator" can be set to $1$ if Metropolis step is enabled and $0$ otherwise.

\subsection{\label{sec:optimized_hmc}Optimization of HMC}

\subsubsection{\label{sec:Omelyan}\index{Omelyan}Omelyan integrator}

The Omelyan integrator can be simply written as (c.f. Eq.~(2.80) of Ref.~\cite{latticeqcdbook2017})
\begin{equation}
\begin{split}
&P_{\lambda\epsilon}\circ U_{\frac{1}{2}\epsilon}\circ P_{(1-2\lambda)\epsilon}\circ U_{\frac{1}{2}\epsilon}\circ P_{\lambda\epsilon}\left(P_0,U_0\right)
\end{split}
\label{eq.hmc.update_Omelyan}
\end{equation}
with
\begin{equation}
\begin{split}
&\lambda = \frac{1}{2}-\frac{\left(2\sqrt{326}+36\right)^{\frac{1}{3}}}{12}+\frac{1}{6\left(2\sqrt{326}+36\right)^{\frac{1}{3}}}\approx 0.19318332750378364
\end{split}
\label{eq.hmc.update_Omelyan2}
\end{equation}

In practical, the $\lambda$ is a tunable parameter, and usually, $2\lambda = 0.3 \sim 0.5$~\cite{latticeqcdreview2009}. The \verb"Omelyan2Lambda" parameter of \verb"Updator" is a input parameter to set $2\lambda$, which if left blank is set to be $0.38636665500756728$ by default.

Usually, for each sub-step, it is $2$ times slower then leap-frog, and for one trajectory, it is $1.5$ time faster~\cite{latticeqcdreview2009}, implying the number of sub-step needed is about $1/3$ of leap-frog.

\subsubsection{\label{sec:OmelyanForceGradient}\index{force-gradient integrator}Omelyan force-gradient integrator}

Start from the approximation
\begin{equation}
\begin{split}
&\log\left(\exp(\frac{\epsilon S}{6})\exp(\frac{\epsilon T}{2})\exp\left(\frac{2}{3}\epsilon S+\frac{\epsilon^3}{72}[S,[S,T]]\right)\exp(\frac{\epsilon T}{2})\exp(\frac{\epsilon S}{6})\right)=S+T+\mathcal{O}(\epsilon ^4)+\mathcal{O}(\epsilon ^6)\\
\end{split}
\end{equation}
with~\cite{forcegradient1}
\begin{equation}
\begin{split}
&\mathcal{O}(\epsilon^4)\sim 10^{-4}\epsilon^4
\end{split}
\end{equation}

The other 4 steps are the usual ones, except for $\exp\left(\frac{2}{3}\epsilon S+\frac{\epsilon^3}{72}[S,[S,T]]\right)$ which correspond to
\begin{equation}
\begin{split}
&\omega_i\to \omega_i-\frac{2}{3}\tau \frac{\partial }{\partial \omega _i}S+\frac{1}{36}\tau ^3 \sum _j \left(\frac{\partial }{\partial \omega _j}S\right)\frac{\partial }{\partial \omega _j}\frac{\partial }{\partial \omega _i}S\\
&p_i=\sum _a \omega ^a_iT^a\\
\end{split}
\end{equation}
Use the approximation~\cite{forcegradient2}
\begin{equation}
\begin{split}
&\frac{2}{3}\tau \frac{\partial }{\partial \omega _i}S-\frac{1}{36}\tau ^3 \sum _j \left(\frac{\partial }{\partial \omega _j}S\right)\frac{\partial }{\partial \omega _j}\frac{\partial }{\partial \omega _i}S\\
&=\frac{2}{3}\tau \exp\left(-\frac{1}{24}\tau ^2 \sum _j\left(\frac{\partial}{\partial \omega _j}S\right)\frac{\partial}{\partial \omega _j}\right)\frac{\partial}{\partial \omega_i}S+\mathcal{O}(\tau ^5)\\
\end{split}
\end{equation}

Let $U'$ be a function of $U$, solving
\begin{equation}
\begin{split}
&\frac{2}{3}\tau \exp\left(-\frac{1}{24}\tau ^2 \sum _j\left(\frac{\partial}{\partial \omega _j}S\right)\frac{\partial}{\partial \omega _j}\right)\frac{\partial}{\partial \omega_i}S(U)=\frac{2}{3}\tau\frac{\partial}{\partial \omega _i}S(U')\\
&\exp\left(-\frac{1}{24}\tau ^2 \sum _j\left(\frac{\partial}{\partial \omega _j}S\right)\frac{\partial}{\partial \omega _j}\right)\frac{\partial}{\partial \omega_i}U\textcolor[rgb]{0,0,1}{\frac{\partial S(U)}{\partial U}}=\frac{\partial}{\partial \omega _i}U'\textcolor[rgb]{0,0,1}{\frac{\partial S(U')}{\partial U'}}\\
&\textcolor[rgb]{0,0,1}{\frac{\partial }{\partial \omega _i}}\left(\exp\left(-\frac{1}{24}\tau ^2 \sum _j\left(\frac{\partial}{\partial \omega _j}S\right)\frac{\partial}{\partial \omega _j}\right)U\right)=\textcolor[rgb]{0,0,1}{\frac{\partial }{\partial \omega _i}}U'\\
&\exp\left(-\frac{1}{24}\tau ^2 \sum _j\left(\frac{\partial}{\partial \omega _j}S\right)\frac{\partial}{\partial \omega _j}\right)U=U',\\
&U'=\exp\left(-\frac{1}{24}\tau ^2 \sum _j\left(\frac{\partial}{\partial \omega _j}S\right)\frac{\partial}{\partial \omega _j}\right)U=\exp\left(-\frac{1}{24}\tau ^2 \sum _j\left(\frac{\partial}{\partial \omega _j}S\right)T_i\right)U\\
\end{split}
\end{equation}

This approximation can be divided into 3 steps:

\begin{enumerate}
  \item Calculate $U'=\exp\left(-\frac{1}{24}\tau ^2 \sum _j\left(\frac{\partial}{\partial \omega _j}S\right)T_i\right)U$.
  \item Use $S[U']$ and $\frac{2}{3}\tau$ to update $P$.
  \item Restore $U$.
\end{enumerate}

It is easy to implement, we do not need to calculate second derivative, and $\sum _j\left(\frac{\partial}{\partial \omega _j}S\right)T_i$ is already implemented, it is nothing but the \textbf{force}.

It is almost as accurate as force-gradient, see the compare in Ref.~\cite{forcegradient3}

\subsubsection{\label{sec:MultiRateUpdator}\index{multi-rate integrator}\index{nested integrator}Multi-rate integrator (nested integrator)}

Following Ref.~\cite{forcegradient3}

Assuming the action is $S=S_F+S_G$ with $S_G\gg S_F$, one can evaluate $S_G$ more often than $S_F$. In the case of lattice QCD, often, the $S_G$ is the cheap gauge force, and $S_F$ is the expansive fermion force.

The nested scheme is different for leap-frog Omelyan and force-gradient integrator, but they are similar

\begin{itemize}
  \item Nested leap-frog
\end{itemize}

\begin{equation}
\begin{split}
&\Delta (h) =\exp (\frac{h}{2}S_F)\Delta _m (h) \exp (\frac{h}{2}S_F),\\
&\Delta _m(h)=\left(\exp(\frac{h}{2m}S_G)\exp(\frac{h}{m}T)\exp(\frac{h}{2m}S_G)\right)^m.\\
\end{split}
\end{equation}

\begin{itemize}
  \item Nested Omelyan
\end{itemize}

\begin{equation}
\begin{split}
&\Delta (h) =\exp (\lambda h S_F)\Delta _m (\frac{h}{2})\exp((1-2\lambda)hS_F)\Delta _m(\frac{h}{2}) \exp (\epsilon h S_F),\\
&\Delta _m(\textcolor[rgb]{1,0,0}{h})=\left(\exp(\frac{\lambda h}{m} S_G)\exp(\frac{h}{2m}T)\exp(\frac{1-2\lambda}{m}hS_G)\exp(\frac{h}{2m}T)\exp(\frac{\lambda h}{m}S_G)\right)^m.\\
\end{split}
\end{equation}

\textbf{Note, it is $\Delta _m (\frac{h}{2})$ in the first line.}

\begin{itemize}
  \item Nested force-gradient
\end{itemize}

\begin{equation}
\begin{split}
&\Delta (h) =\exp (\frac{h}{6} S_F)\Delta _m (\frac{h}{2})\exp(\frac{2}{3}h S_F+\frac{1}{72}h^3 C_F)\Delta _m(\frac{h}{2}) \exp (\frac{h}{6} S_F),\\
&\Delta _m(\textcolor[rgb]{1,0,0}{h})=\left(\exp(\frac{h}{6 m} S_G)\exp(\frac{h}{2m}T)\exp\left(\frac{2}{3}\frac{h}{m} S_G+\frac{1}{72}\left(\frac{h}{m}\right)^3 C_G\right)\exp(\frac{h}{2m}T)\exp(\frac{h}{6m}S_G)\right)^m.\\
\end{split}
\end{equation}


\textbf{Note about the integrator: when analyzing the error of the integrators, it is assumed $e^T$, $e^{S_G}$ and $e^{S_F}$ can be calculated accurately. It is almost true for $e^{S_G}$, and almost true for $e^T$ as long as $\epsilon$ is not too large, but it is not true for $e^{S_F}$. Typically, using an optimized integrator, it needs more accurate criterion for solvers.}

\subsubsection{\label{sec:CachedSolution}Cached solution}

The pseudo fermion field is generate only once for a trajectory and is not changed. Also, the gauge field is changing slowly in one trajectory, this make the solutions for ${\bf x}_1=D^{-1}{\bf b}$ or ${\bf x}_2=\left(DD^{\dagger}\right)^{-1}{\bf b}$, where $D$ depends on $U$ and ${\bf b}$ is the pseudo fermion field, only change slowly.

So, once ${\bf x}_{1,2}$ is obtained, in the same trajectory, ${\bf x}_{1,2}$ can be set as the initial trail solution for the solver.

\begin{comment}
\subsubsection{\label{sec:Z4_fermion}Z4 pseudo fermion}

In the pseudo-fermion approach, $\phi$ is a complex scalar field. However, it can also been viewed as a mathematical tool. To see this, what we are in fact evaluating is
\begin{equation}
\begin{split}
&\int \mathcal{D}\phi \mathcal{D}\phi^{\dagger} \exp (-\phi ^{\dagger}\left(DD^{\dagger}\right)^{-1}\phi )\\
\end{split}
\end{equation}
Let $\chi = D^{-1}\phi$, $\phi$ is a linear function of $\chi$, one have
\begin{equation}
\begin{split}
&\phi=D\chi,\;\;\frac{\partial \phi _i}{\partial \chi _j}=\left(D\right)_{ji},\\
&\mathcal{D}\phi \to \det \left[D\right]\mathcal{D}\chi,\;\;\mathcal{D}\phi^{\dagger} \to \det \left[D^{\dagger}\right]\mathcal{D}\chi^{\dagger}\\
\end{split}
\end{equation}
It becomes
\begin{equation}
\begin{split}
&\det[DD^{\dagger}]\int \mathcal{D}\chi \mathcal{D}\chi^{\dagger} \exp (-\chi ^{\dagger}\chi )\\
\end{split}
\end{equation}
\end{comment}

\subsection{\label{sec:staggered_fermion}Staggered Fermion}

\subsubsection{\label{sec:jordanWigner1D}The D=1 staggered fermion and Jordan-Wigner transformation}

One way to introduce fermion is to use the Ising action (which is particularly useful in quantum computer simulation of $\mathbb{Z}_2$ lattice gauge). The Jordan-Wigner transformation of $D=1$ Ising model is famous, now considering $D=1$ $XY$ model
\begin{equation}
\begin{split}
&H=\frac{1}{4a}\sum _n \left(\sigma _x(n)\sigma _x(n+1)+\sigma _y(n)\sigma _y(n+1)\right)
\end{split}
\end{equation}
with
\begin{equation}
\begin{split}
&\phi (n)=\left\{\prod _{m<n}(i\sigma _z(n))\right\}\sigma ^+(n),\;\;\phi (n)=\left\{\prod _{m<n}(-i\sigma _z(n))\right\}\sigma ^-(n),\;\;\sigma ^{\pm}=\sigma _x\pm \sigma _y
\end{split}
\end{equation}
one can verify that $\phi$ is fermion and
\begin{equation}
\begin{split}
&H=\frac{i}{2a}\sum _n \left(\phi ^{\dagger}(n)\phi(n+1) -\phi ^{\dagger}(n+1)\phi(n)\right)
\end{split}
\end{equation}
with EoM same as naive discretized fermion
\begin{equation}
\begin{split}
&\dot{\phi}(n)=-i[\phi (n), H]=\frac{1}{2a} \left(\phi (n+1)-\phi (n-1)\right)
\end{split}
\end{equation}
