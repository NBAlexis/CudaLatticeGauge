\subsection{\label{sec:rhmc}\index{RHMC}The RHMC for staggered fermion}

From now on, we consider $N_f=2+1$. Two key quantities should be calculated.

$\bullet$ Calculate the action

\begin{equation}
\begin{split}
&\exp(-S)=\left(\det [D_{st}(m_{ud})]\right)^{\frac{1}{2}}\left(\det [D_{st}(m_s)]\right)^{\frac{1}{4}} \exp (-S_G)\\
\end{split}
\end{equation}

The problem is how to evaluate $\left(\det [D]\right)^{\alpha}$. Before that, we have to prepare some tools.

\subsubsection{\label{sec:rationalapproximation}The \index{rational approximation}rational approximation and Remes algorithm}

The first thing to do is to find a rational approximation of a function, here we use \index{Remes algorithm}Remes algorithm as proposed by Ref.~\cite{staggeredRHMC}

The Remes algorithm is implemented in Mathematica, which finds
\begin{equation}
\begin{split}
&g(x)=c+\sum _i \frac{a_i}{x+b_i}\approx f(x),
\end{split}
\end{equation}
in an interval. This have been implemented in Mathematica so we use the result of Mathematica directly. An example is
\begin{lstlisting}
<< FunctionApproximations\`; \\
res1 = Apart[MiniMaxApproximation[1/Sqrt[x], \{x, \{0.003, 1\}, 3, 3\}][[2]][[1]]] \\
Join[\{Part[res1, 1]\}, Table[Numerator[Part[res1, n]], {n, 2, 4}], Table[Denominator[Part[res1, n]] /. x -> 0, \{n, 2, 4\}]]
\end{lstlisting}

\subsubsection{\label{sec:hmcofrhmc}The RHMC with \texorpdfstring{$N_f=2$}{Nf=2}}

We mainly follow Ref.~\cite{staggeredRHMC}.

Similar as the HMC, the first step is to rewrite the action use a Boson field~(pseudo-fermion field), still starting with
\begin{equation}
\begin{split}
&Z=\int \mathcal{D}[U] \prod _{f=1}^{N_f} \mathcal {D}[\bar{\psi}_f]\mathcal {D}[\psi _f]\left(\det D_{st}\right)^{\frac{1}{4}}\exp \left(-S_G[U]\right)\\
\end{split}
\end{equation}
\textbf{Note that the $1/4$ dose NOT come from the $1/4$ in Eq.~(\ref{eq.naivestaggeredaction}), it is to deal with the $4$ doublers.} With
\begin{equation}
\begin{split}
&D_{st}(n|m)=2m\delta _{m,n} \textcolor[rgb]{1,0,0}{+} \sum _{\mu}\eta _{\mu}(n)\left(U_{\mu}(n) \delta _{n,n+\mu} - U_{-\mu}(n)\delta _{n,n-\mu}\right)\\
\end{split}
\end{equation}

It is very interesting that, although stated in many literatures, this is \textbf{\textcolor[rgb]{1,0,0}{not}} the true \index{fourth root}fourth root method. The true fourth root method is after the `pseudofermion' step. First, just like the usual HMC to evaluate
\begin{equation}
\begin{split}
&Z=\int \mathcal{D}[U] \prod _{f=1}^{N_f} \left(\det D_{st} \right)\exp \left(-S_G[U]\right)\\
\end{split}
\end{equation}
For $N_f=2$, before the 4th root, it is
\begin{equation}
\begin{split}
&Z=\int \mathcal{D}[U] \left(\det D_{st}^{\dagger}D_{st} \right)\exp \left(-S_G[U]\right)\\
\end{split}
\end{equation}
Note that because $D_{st0}$ for massless is anti-Hermitian, $D_{st}^{\dagger}D_{st}=-\frac{1}{4a^2}D_{st0}^2+m^2$, for simplicity we denote $A=-D_{st0}^2+4a^2m^2$ (with a rescale of $\psi$ assumed), using
\begin{equation}
\begin{split}
&\det A = \frac{1}{ \det A^{-1} }=\int \mathcal{D}[\phi] \exp (-\phi^{\dagger} A^{-1} \phi).\\
\end{split}
\end{equation}
For $N_f=2$ it is
\begin{equation}
\begin{split}
&Z=\int \mathcal{D}[U]  \mathcal {D}[\phi]\exp \left(-S_G[U]-\phi ^{\dagger} A^{-1} \phi \right)
\end{split}
\end{equation}

The true fourth root method is taken here as
\begin{equation}
\begin{split}
&Z=\int \mathcal{D}[U] \left(\det A \right)^{\frac{N_f}{4}}\exp \left(-S_G[U]\right)\\
&Z=\int \mathcal{D}[U]  \mathcal {D}[\phi]\exp \left(-S_G[U]-\phi ^{\dagger} A^{-\frac{N_f}{4}} \phi \right)\\
\end{split}
\end{equation}
Note that the fourth root is just to remove doublers, so for $N_f=2$, it is $A^{1/2}$ in spite that $A=D^{\dagger}D$ already considered $N_f=2$.

In the following, we concentrate on
\begin{equation}
\begin{split}
&Z=\int \mathcal{D}[U]  \mathcal {D}[\phi]\exp \left(-S_G[U]-\phi ^{\dagger} A^{-\frac{1}{2}} \phi \right)\\
\end{split}
\end{equation}

$\bullet$ Pseudofermion refreshment of staggered fermion

\textbf{\textcolor[rgb]{1,0,0}{Note! The refreshment of staggered fermion use a different action!} It can be summarized as $Z_{MC}\approx Z_{MD}$, where $Z_{MC}$ is to refresh the pseudofermion, $Z_{MD}$ is for molecular dynamic.}
\begin{equation}
\begin{split}
&Z_{MC}=\int \mathcal{D}[U]  \mathcal {D}[\phi]\exp \left(-S_G[U]-\phi ^{\dagger} \left(A^{-\frac{1}{4}}\right)^2 \phi \right)\\
&Z_{MD}=\int \mathcal{D}[U]  \mathcal {D}[\phi]\exp \left(-S_G[U]-\phi ^{\dagger} r_2(A) \phi \right)\\
\end{split}
\end{equation}
where $r_2(A)\approx A^{-\frac{1}{2}}$.

Then, the refreshment step is:

Let $\chi$ be random \textbf{complex} numbers according to $\exp (-\chi ^{\dagger}\chi)$. Let $A^{-\frac{1}{4}}\phi=\chi$, $\phi$ is the random \textbf{complex} number satisfying distribution we want ($\exp (-\phi^{\dagger} \left(A^{-\frac{1}{4}}\right)^{\dagger} A^{-\frac{1}{4}} \phi)$). So, first get $\chi$ and then let $\phi = A^{\textcolor[rgb]{1,0,0}{+}\frac{1}{4}}\chi$.

$\bullet$ MD step of staggered fermion

Let $A^{-\frac{1}{2}}=c_0+\sum _i \frac{\alpha _i }{A+\beta _i} $, it is
\begin{equation}
\begin{split}
&Z=\int \mathcal{D}[U]  \mathcal {D}[\phi]\exp \left(-S_G[U]-c_0\phi ^{\dagger} \phi -\sum_i \phi ^{\dagger}\frac{\alpha _i }{A+\beta _i} \phi\right)
\end{split}
\end{equation}
then
\begin{equation}
\begin{split}
&F=-iT_a \frac{\partial }{\partial \omega _a}\left(\sum_i \phi ^{\dagger}\frac{\alpha _i }{A+\beta _i} \phi\right)\\
&=-iT_a \sum_i \left(\phi ^{\dagger}\frac{\alpha _i }{(A+\beta _i)^2} \frac{\partial }{\partial \omega _a} A \phi\right)\\
\end{split}
\end{equation}
\textbf{The order of matrix product is wrong}, it is in fact
\begin{equation}
\begin{split}
&F=-iT_a \sum_i \left(\phi ^{\dagger}\frac{\sqrt{\alpha _i} }{(A+\beta _i)} \left(\frac{\partial }{\partial \omega _a} A \right) \frac{\sqrt{\alpha _i} }{(A+\beta _i)} \phi\right)\\
\end{split}
\end{equation}
Now, let $\phi _{i} =  \frac{\sqrt{\alpha _i} }{(A+\beta _i)} \phi$, it is
\begin{equation}
\begin{split}
&F=-iT_a \sum _i \phi _{i}^{\dagger} \left(\frac{\partial }{\partial \omega _a} A \right) \phi _{i}\\
\end{split}
\end{equation}
\textbf{Note it is not $\phi ' =  \sum _i \frac{\sqrt{\alpha _i} }{(M+\beta _i)} \phi$ because generally, $\phi _{i}^{\dagger} A' \phi _{j} \neq 0$}

To calculate the $A'$ we use
\begin{equation}
\begin{split}
&\frac{\partial }{\partial \omega _a}A = \left(\frac{\partial }{\partial \omega _a}D^{\dagger}_{st0}\right)D_{st0}+D^{\dagger}_{st0}\left(\frac{\partial }{\partial \omega _a}D_{st0}\right)\\
&\phi _i^{\dagger}\left(\frac{\partial }{\partial \omega _a}A\right)\phi _i=\phi _i^{\dagger}\left(\frac{\partial }{\partial \omega _a}D^{\dagger}_{st0}\right)\phi _{i,d}+\phi _{i,d}^{\dagger}\left(\frac{\partial }{\partial \omega _a}D_{st0}\right)\phi _i\\
&=\left(\phi _{i,d}^{\dagger}\left(\frac{\partial }{\partial \omega _a}D_{st0}\right)\phi _i\right)^{\dagger}+\left(\phi _{i,d}^{\dagger}\left(\frac{\partial }{\partial \omega _a}D_{st0}\right)\phi _i\right)\\
&=2{\rm Re}\left[\phi _{i,d}^{\dagger}\left(\frac{\partial }{\partial \omega _a}D_{st0}\right)\phi _i\right]
\end{split}
\end{equation}
with $\phi _{i,d}=D_{st0}\phi _i$.

\begin{equation}
\begin{split}
&\frac{\partial }{\partial \omega _a}  D_{st0}=\sum _{\mu}\left(\eta _{\mu}(n)\left(\frac{\partial }{\partial \omega _a}U_{\mu}(n)\right) \delta _{n,n+\mu} - \textcolor[rgb]{1,0,0}{\eta _{\mu}(n+\mu)}\left(\frac{\partial }{\partial \omega _a} U^{\dagger}_{\mu}(n)\right)\delta _{n+\mu,n}\right)\\
&=i\eta _{\mu}(n)\left(\left(T_aU_{\mu}(n)\right) \delta _{n,n+\mu} + \left(U^{\dagger}_{\mu}(n)T_a\right)\delta _{n+\mu,n}\right)\\
\end{split}
\end{equation}
note that the $\eta _{\mu} (n+\mu)$ follows $\delta _{\textcolor[rgb]{1,0,0}{n+\mu},n}$. The last step we use $\eta _{\mu}(n+\mu)=\eta_{\mu}(n)$.
The $T_a=T_a^{\dagger}$, let
\begin{equation}
\begin{split}
&\phi _A = \phi _i, \phi _B = T_aU_{\mu}(n)\phi _i\\
&\phi _C = \phi _{i,d}, \phi _D = T_aU_{\mu}(n)\phi _{i,d}\\
\end{split}
\end{equation}
it is
\begin{equation}
\begin{split}
&\phi _{i,d}^{\dagger}\left(\frac{\partial }{\partial \omega _a}D_{st0}\right)\phi _i = i\eta(n)\left(\phi _C^{\dagger}(n)\phi _B(n+\mu) + \phi _D^{\dagger}(n+\mu) \phi _A(n)\right)\\
&\phi _i^{\dagger}\left(\frac{\partial }{\partial \omega _a}A\right)\phi _i=i\eta(n)\left(\phi _C^{\dagger}(n)\phi _B(n+\mu) + \phi _D^{\dagger}(n+\mu)\phi _A(n)\right)-i\eta(n)\left(\phi _B^{\dagger}(n+\mu)\phi _C(n) + \phi _A^{\dagger}(n)\phi _D(n+\mu)\right)\\
&=i\eta (n)\left\{\left(\phi _C^{\dagger}(n)\phi _B(n+\mu)-\phi _B^{\dagger}(n+\mu)\phi _C(n)\right)+\left( \phi _D^{\dagger}(n+\mu)\phi _A(n)-\phi _A^{\dagger}(n)\phi _D(n+\mu)\right)\right\}\\
&=-2\eta (n) {\rm Im}\left[\phi _C^{\dagger}(n)\phi _B(n+\mu) + \phi _D^{\dagger}(n+\mu)\phi _A(n)\right]\\
&=-2\eta (n) {\rm Im}\left[\phi _{i,d}^{\dagger}T_aU_{\mu}(n) \delta _{n,n+\mu}\phi _i + \phi _{i,d}^{\dagger}(n+\mu)U^{\dagger}_{\mu}(n)T_a\delta _{n+\mu,n}\phi _i\right]\\
\end{split}
\end{equation}
let $M(n|m)=U_{\mu}(n) \delta _{n,n+\mu}$, it is
\begin{equation}
\begin{split}
&\phi _i^{\dagger}\left(\frac{\partial }{\partial \omega _a}A\right)\phi _i = -2\eta (n) {\rm Imtr}\left\{T_a  \left[\left(U_{\mu}(n)\phi _i(n+\mu)\right)\phi _{i,d}^{\dagger}(n)\right]+T_a \left[\phi _i(n)(U_{\mu}(n)\phi _{i,d}(n+\mu))^{\dagger}\right]\right\}\\
\end{split}
\end{equation}
One can verify that for any matrix \textcolor[rgb]{0,0,1}{
\begin{equation}
\begin{split}
&2i \sum _a T_a {\rm Im tr}[T_a M] = -2 i \sum _a T_a {\rm Re tr}[i T_a M] = \{M\}_{TA}\\
&2i \sum _a T_a {\rm Im tr}[i T_a M] = 2 i \sum _a T_a {\rm Re tr}[T_a M] = \{i M\}_{TA}\\
\end{split}
\end{equation}
}
so
\begin{equation}
\begin{split}
&F_{\mu}(n)=-iT_a \sum _i \phi _{i}^{\dagger} \left(\frac{\partial }{\partial \omega _a} A \right) \phi _{i}\\
&=\eta _{\mu}(n) \sum _a \sum _i 2i T_a{\rm Imtr}\left\{T_a  \left[\left(U_{\mu}(n)\phi _i(n+\mu)\right)\phi _{i,d}^{\dagger}(n)\right]+T_a \left[\phi _i(n)(U_{\mu}(n)\phi _{i,d}(n+\mu))^{\dagger}\right]\right\}\\
&=\eta _{\mu}(n) \sum _i \left\{\left[\left(U_{\mu}(n)\phi _i(n+\mu)\right)\phi _{i,d}^{\dagger}(n)\right]+\left[\phi _i(n)(U_{\mu}(n)\phi _{i,d}(n+\mu))^{\dagger}\right]\right\}_{TA}
\end{split}
\label{eq.staggeredformnf2}
\end{equation}
with
\begin{equation}
\begin{split}
&\phi _i = \frac{\sqrt{\alpha _i} }{(D_{st}^{\dagger}D_{st})+\beta _i} \phi,\;\;\alpha _i,\beta _i \in A^{-\frac{1}{2}}\\
&\phi _{i,d}=\textcolor[rgb]{1,0,0}{D_{st0}}\phi _i\\
\end{split}
\end{equation}

Here is the summary of $N_f=2$ RHMC for staggered fermion
\begin{enumerate}
  \item Refresh $\phi$ using $\phi = (D_{st}^{\dagger}D_{st})^{\frac{1}{4}}\chi$, where $\chi$ is random Gaussian complex field.
  \item Calculate force using Eq.~(\ref{eq.staggeredformnf2}).
  \item Metropolis step.
\end{enumerate}

Some tips on RHMC of staggered fermions:

\begin{itemize}
  \item Let $r_1(M)\approx M^{a}$, $r_2(M)\approx M^{b}$, $r_1(M)$ is generally less accurate than $(r_2(M))^{1/b}$ when $b>1$.
  \item For $N_f=2+1$, there are other optimizations.
\end{itemize}

\subsubsection{\label{sec:hmcofrhmcnf2p1}The RHMC with \texorpdfstring{$N_f=2+1$}{Nf=2+1}}

Before we continue, it is necessary to extend the nested integrator, here we use Force gradient because it is the best one, a two level integrator is introduced in Sec.~\ref{sec:MultiRateUpdator}, which is modified as
\begin{equation}
\begin{split}
&\Delta (h) =\exp (\frac{h}{6} S_F)\Delta _m (\frac{h}{2m})\exp(\frac{2}{3}h S_F+\frac{1}{72}h^3 C_F)\Delta _m(\frac{h}{2m}) \exp (\frac{h}{6} S_F),\\
&\Delta _m(\textcolor[rgb]{1,0,0}{h})=\left(\exp(\frac{h}{6 } S_G)\exp(\frac{h}{2}T)\exp\left(\frac{2}{3}h S_G+\frac{1}{72}h^3 C_G\right)\exp(\frac{h}{2}T)\exp(\frac{h}{6}S_G)\right)^m.\\
\end{split}
\end{equation}

The Lagrangian of $N_f=2+1$ is
\begin{equation}
\begin{split}
&Z=\int \mathcal{D}[U]  \mathcal {D}[\phi]\exp \left(-S_G[U]-\phi _1^{\dagger} A_{ud}^{-\frac{1}{2}} \phi _1 -\phi _2 ^{\dagger} A_s^{-\frac{1}{4}} \phi _2\right)\\
\end{split}
\end{equation}
where $A=D^{\dagger}D$~(Note that it is always $A^{-\frac{N_f}{4}}$). Which is also
\begin{equation}
\begin{split}
&Z= \left(\det \left[A_{ud}\right]\right)^{\frac{1}{2}}\left(\det \left[A_{s}\right]\right)^{\frac{1}{4}}\int \mathcal{D}[U]  \exp \left(-S_G[U]\right)\\
& = \frac{\left(\det \left[A_{ud}\right]\right)^{\frac{1}{2}}}{\left(\det \left[A_{ud}\right]\right)^{\frac{1}{2}}}\left(\det \left[A_{s}\right]\right)^{\frac{3}{4}}\int \mathcal{D}[U]  \exp \left(-S_G[U]\right)\\
& = \left(\det \left[\frac{A_{ud}}{A_{ud}}\right]\right)^{\frac{1}{2}}\left(\det \left[A_{s}\right]\right)^{\frac{3}{4}}\int \mathcal{D}[U]  \exp \left(-S_G[U]\right)\\
&=\int \mathcal{D}[U]  \mathcal {D}[\phi]\exp \left(-S_G[U]-\phi _1^{\dagger} \left(\frac{A_{s}}{A_{ud}}\right)^{\frac{1}{2}} \phi _1 -\phi _2 ^{\dagger} A_s^{-\frac{3}{4}} \phi _2\right)\\
\end{split}
\end{equation}
using $A_{ud,s}=-D_{st0}^2+4a^2m_{ud,s}^2$, so $A_s=A_{ud}+4a^2(m_s^2-m_{ud}^2)$. Defining $\delta m^2 = m_s^2-m_{ud}^2$, it is
\begin{equation}
\begin{split}
&Z= \int \mathcal{D}[U]  \mathcal {D}[\phi]\exp \left(-S_G[U]-\phi _1^{\dagger} \left(\frac{A_{ud}+4a^2\delta m^2}{A_{ud}}\right)^{\frac{1}{2}} \phi _1 -\phi _2 ^{\dagger} A_s^{-\frac{3}{4}} \phi _2\right)\\
\end{split}
\end{equation}
This is called \index{mass precondition}`mass precondition', the force from the $D_{ud}$ is much smaller and one can have a three level nested integrator. Defining
\begin{equation}
\begin{split}
&r_{1,MC}(x) =  \left(\frac{x+4a^2\delta m^2}{x}\right)^{-\frac{1}{4}},\;\;r_{2,MC}(x)=  x^{\frac{3}{8}}\\
&r_{1,MD}(x) =  \left(\frac{x+4a^2\delta m^2}{x}\right)^{\frac{1}{2}},\;\;r_{2,MD}(x)=  x^{-\frac{3}{4}}\\
\end{split}
\end{equation}
All the other steps are same as $N_f=2$.

\subsubsection{\label{sec:multifieldOptimization}Multi field Optimization}

The multi-level nested integrator is useless compared with multi field optimization. Instead of evaluating $\bar{\psi} A \psi$, one use $\sum _i^{\alpha} \bar{\psi _i} A^{\frac{1}{\alpha}} \psi _i$. Typically, the force from the $s$ quark is much larger than the mass preconditioned $u,d$ quarks, so we can use $\sum _{i=1,2,3} \bar{\psi _i} A^{-\frac{1}{4}} \psi _i$ instead of $\bar{\psi} A^{-\frac{3}{4}} \psi$. And then use a nested integrator.

\textbf{\textcolor[rgb]{0,0,1}{I don't know why but it is tested that Omelyan nested with leapfrog is the best choice, better than approximate force gradient.}}

\subsubsection{\label{sec:TipToImplementHigherOrdersStaggeredFermion}Tip to implement higher orders staggered fermion}

As required by anti-Hermiticity and gauge invarience, for higher orders, we always have
\begin{equation}
\begin{split}
&\bar{\chi}(n_1)U(n_1,n_2)\chi (n_2) - \bar{\chi}(n_2)U^{\dagger}(n_1,n_2)\chi (n_1)\\
\end{split}
\end{equation}
